\chapter{Theoretical Background}

\section{Molecular Dynamics}

Molecular dynamics (MD) is a computer simulation technique where the time
evolution of a set of interacting atoms is calculated using equations of
motion. The laws of classical mechanics, such as Newton's second law, is
followed in this technique. For example, in a closed system consisting of $N$
atoms, the force acting on an atom $i$ is
\begin{align}
	\bm{F}_i = m \bm{a}_i = m \frac{d^2 \bm{r}_i}{dt^2}
\end{align}
This force is a result of the interactions of atom $i$ with all the other atoms
in the system. If the force acting on an atom is known, its movement can be
predicted given the initial position and velocity \cite{ercolessi}.

In \textit{ab initio} MD, quantum mechanics is used for finding the interatomic
forces. Although calculations from first principle are accurate, they are
computationally expensive and sometimes infeasible. Another way to derive these
forces is to compute the gradients of the potential energy of the system with
respect to atomic displacements.
\begin{align}
	\bm{F}_i = - \nabla_{\bm{r}_i} V(\bm{r}_1,\ldots,\bm{r}_N)
\end{align}
This approach is called classical MD (referred as only MD from here on).
Finding a practical and efficient description of the potential energy is the
crux of MD. The interatomic potentials, in this case, tries to approximate the
true nature of atomic interactions at the expense of accuracy, the advantage
being reduced computational cost \cite{marx}.

The simplest interatomic potentials for MD only consider pairwise interactions
between atoms. These two-body (pairwise) potentials are poor for metal systems.
Few different schemes to construct many-body potentials for metals exist, based
fundamentally on the following form.
\begin{align}
	V = \frac{1}{2} \sum_{i=1}^N \sum_{\substack{j=1\\j\ne i}}^N
		\phi(r_{ij}) + \sum_{i=1}^N F_i(n_i) = \sum_{i=1}^N V_i
\end{align}
where $\phi(r_{ij})$ is the pairwise interaction potential between atoms $i$
and $j$, $F_i$ is the embedding energy of atom $i$ as a function of a
"generalized coordination" $n_i$. This coordination $n_i$ is a superposition of
contributions from the neighboring atoms of atom $i$, and it can be expressed
as $n_i = \sum_{j\ne i} \rho(r_{ij})$ \cite{ercolessi, carlsson}.

% TODO: Finnis-Sinclair to be replaced with ADP

The Finnis-Sinclair potentials follow the above many-body potential model, and
the potential energy of an atom $i$ in these potentials can be expressed as
\begin{align}
	V_i = \frac{1}{2} \sum_{\substack{j=1\\j \ne i}}^N \phi_{ij} (r_{ij}) +
	F_i \bigg( \sum_{\substack{j=1\\j \ne i}}^N \rho_{ij} (r_{ij}) \bigg)
\end{align}
where $\rho_{ij}$ refers to the density contributed by a neighbor atom $j$ at
the site of atom $i$. This density is dependent on both atom $j$ and atom $i$,
meaning the neighboring atoms will contribute differently to the electron
density at an atomic site depending on the identity of the atom present at that
atomic site \cite{fs, mendelev}.

After defining the interactions among atoms through a potential, new positions
and velocities of atoms can be calculated using time integration.
Velocity-Verlet is the most commonly used integration scheme in MD
\cite{verlet, tuckerman}.

MD simulations are mainly done in the microcanonical ensemble. However, in many
situations, the simulated system needs to have a constant temperature. Among
many methods introduced to control the temperature in a MD simulation, the
Nos\'{e}-Hoover thermostat is the most popular one \cite{hoover}.

\subsection{Angular Dependent Potential}

\begin{align}
	E_{tot}
	= \frac{1}{2} \sum_{i,j (j \ne i)} \phi_{s_i s_j} (r_{ij})
		+ \sum_i F_{s_i} (\bar{\rho_i})
		+ \frac{1}{2} \sum_{i, \alpha} (\mu_i^{\alpha})^2
		+ \frac{1}{2} \sum_{i, \alpha, \beta} (\lambda_i^{\alpha\beta})^2
		- \frac{1}{6} \sum_i \nu_i^2
\end{align}

\begin{align}
	\bar{\phi_i}
	= \sum_{j \ne i} \rho_{s_j} (r_{ij})
\end{align}

\begin{align}
	\mu_i^{\alpha}
	= \sum_{j \ne i} u_{s_i s_j} (r_{ij}) r_{ij}^{\alpha}
\end{align}

\begin{align}
	\lambda_i^{\alpha\beta}
	= \sum_{j \ne i} w_{s_i s_j} (r_{ij}) r_{ij}^{\alpha} r_{ij}^{\beta}
\end{align}

\section{\textit{Ab Initio} Molecular Dynamics}

\section{Machine-Learned Interatomic Potential}

\subsection{Moment Tensor Potential}

\section{Inverse Uncertainty Quantification}

Inverse uncertainty quantification (IUQ) is a process
to quantify uncertainties in the input parameters of a computer model,
such as DART, given experimental data.
IUQ is an essential step in computational model validation
because it provides a concrete and quantifiable measure of uncertainty
in model predictions
\cite{wu2018inverse, nagel2019bayesian, wu2021comprehensive}.
The concept of IUQ is in contrast to forward uncertainty quantification (FUQ),
which quantifies the uncertainty in the output of a computer model
given the input parameters.
FUQ is often used to predict the uncertainty in the results of a simulation,
while IUQ is used to quantify the uncertainty in the model itself.
IUQ is often used in engineering and scientific applications
where there is a high degree of uncertainty in the input parameters.
By quantifying the uncertainty in the input parameters,
IUQ can help to improve the accuracy and reliability
of the model predictions \cite{wu2021comprehensive, xie2024functional}.
