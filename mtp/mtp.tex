\chapter{U-Mo Moment Tensor Potential}

\section{Computational Details}

Various U and Mo phases and alloys are simulated using AIMD.
The resulting data is then used to generate an MTP for the two elements
with $\lev_{max} = 10$.
To validate this potential,
MD simulations of $\alpha$U, $\gamma$U, bcc Mo, U$_2$Mo, and U-10Mo
are performed with the airm of extracting
lattice constants, radial distribution fucntions (RDFs),
and other properties as a function of temperature.
$\alpha$U has an orthorombic crystal structure,
whereas U$_2$Mo has a tetragonal structure.
The three other simulated systems have a bcc structure.

The LAMMPS software package is utilized for MD simulations \cite{lammps}.
For all simulations,
supercells of $20 \times 20 \times 20$ in lattice units are used.
Each simulations runs for a total of 100 ps.
NPT ensembles are employed,
with anisotropic control of the x, y, and z dimensions.
The length of dimensional length are averaged every 1 ps,
and data from the last 1 ps is used for analysis.
RDFs are computed using the atomic positions
in the last timestep of the simulations.

\section{Results}

\subsection{Lattice Constants}

Figure \ref{fig:latConst} shows
the lattice constants of the five simulated systems.
The lattice constants computed using the ADP potential \cite{beelerADP}
are also shown in the figure.
The lattice constants from the MTP agree with the ADP calculations.
Even for the noncubic systems, such as $\alpha$U and U$_2$Mo,
the MTP is capable of producing comparable results.
Noncubic systems are generally harder to replicate than cubic systems
for interatomic potentials.

\begin{figure}[!ht]
\begin{subfigure}{0.495\textwidth}
	\caption{}
	\includegraphics[width=0.8\textwidth]{mtp/images/latConst_alphaU.pdf}
\end{subfigure}
\begin{subfigure}{0.495\textwidth}
	\caption{}
	\includegraphics[width=0.8\textwidth]{mtp/images/latConst_gammaU.pdf}
\end{subfigure}
\begin{subfigure}{0.495\textwidth}
	\caption{}
	\includegraphics[width=0.8\textwidth]{mtp/images/latConst_bccMo.pdf}
\end{subfigure}
\begin{subfigure}{0.495\textwidth}
	\caption{}
	\includegraphics[width=0.8\textwidth]{mtp/images/latConst_u2mo.pdf}
\end{subfigure}
\begin{subfigure}{0.495\textwidth}
	\caption{}
	\includegraphics[width=0.8\textwidth]{mtp/images/latConst_U-10Mo.pdf}
\end{subfigure}
\caption{
	Lattice constants of
	(a) $\alpha$U,
	(b) $\gamma$U,
	(c) bcc Mo,
	(d) U$_2$Mo, and
	(e) $\gamma$U-10Mo
	using ADP and MTP potentials.
}
\label{fig:latConst}
\end{figure}

\subsection{Radial Distributions Functions}

Figure \ref{fig:rdfs} depicts the RDFs of a select few pair distances
from the simulated systems.
The U-U pair RDFs in $\alpha$U, $\gamma$U, and U$_2$Mo are reported
along with the Mo-Mo pair RDF in bcc Mo.
The RDFs computed using both MTP and ADP display similar characteristics.
The most noticeable discrepancy is observed in the RDFs of U$_2$Mo.
The ADP predicts slightly higher density for the first nearest neighbor
than the second nearest neighbor for U-U pairs,
whereas MTP predicts the same density
for both first and second nearest neighbors.

\begin{figure}[!ht]
\begin{subfigure}{0.495\textwidth}
	\caption{}
	\includegraphics[width=0.8\textwidth]{mtp/images/rdf_alphaU_500K.pdf}
\end{subfigure}
\begin{subfigure}{0.495\textwidth}
	\caption{}
	\includegraphics[width=0.8\textwidth]{mtp/images/rdf_gammaU_950K.pdf}
\end{subfigure}
\begin{subfigure}{0.495\textwidth}
	\caption{}
	\includegraphics[width=0.8\textwidth]{mtp/images/rdf_bccMo_1600K.pdf}
\end{subfigure}
\begin{subfigure}{0.495\textwidth}
	\caption{}
	\includegraphics[width=0.8\textwidth]{mtp/images/rdf_u2mo_500K.pdf}
\end{subfigure}
\caption{
	Radial distribution functions for
	(a) U-U pairs in $\alpha$U,
	(b) U-U pairs in $\gamma$U,
	(c) Mo-Mo pairs in bcc Mo, and
	(d) U-U pairs in U$_2$Mo.
}
\label{fig:rdfs}
\end{figure}

\section{Discussion}

The MTP demonstrates considerable promise in its capabilities
to predict cystal structures involving U and Mo.
However, further testing is required to assess to accurately predict
elastic constants, melting points, and other thermodynamic properties.
Following the validation of the two-element potential,
our next step will be to incorporate a third element (Al or Zr) to the MTP.
