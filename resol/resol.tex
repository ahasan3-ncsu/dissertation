\chapter{Xe gas bubble re-solution}

\section{PKA simulations}

Limited re-solution was observed in the PKA simulations.
Figure \ref{fig:pka} showcases snapshots from one such simulation,
with the black spheres indicating Xe atoms
and the red and blue spheres representing U and Mo atoms, respectively.
The energy introduced into the system by the PKA
propagates as a shock wave toward the supercell boundary.
This shock wave generates numerous point defects,
most of which eventually recombine.
The Xe gas bubble undergoes deformation immediately after initiating the PKA,
but swiftly reverts to a stable configuration.
In the PKA simulations, only one re-solved Xe atom at most is observed,
with most simulations showing no re-solution at all.
Thus, homogeneous re-solution is considered negligible
within the examined energy range of the PKAs (up to 500 keV).
This holds even though the re-solution from PKA simulations can be
a combination of both homogeneous and heterogeneous re-solution.

The homogeneous re-solution is expected to be a few orders of magnitude smaller
than the heterogeneous re-solution,
as the collision cross section of a PKA is substantially smaller
than the interaction cross section between a bubble and a thermal spike
\cite{olander2006re}.
Govers el al. \cite{govers2012} performed PKA simulations for UO$_2$
and observed, at most, three re-solved Xe atoms
out of gas bubbles containing hundreds.
Thus, they deemed homogeneous re-solution too insignificant
for calculating overall re-solution rates.
The PKA simulations performed in our work on $\gamma$U-10Mo
show similar results.
Also, in the case of $\gamma$U-Mo, only 5\% of the energy from a fission event
is deposited ballistically \cite{beeler2021rad}.
Given that a limited number of re-solved atoms
were observed in the simulations,
and only a small fraction of the energy from fission reactions
go to ballistic collisions,
it is reasonable to assume that the homogeneous re-solution is negligible.

\begin{figure}[ht]
	\centering
	\begin{subfigure}{0.45\textwidth}
		\centering
		\caption{}
		\includegraphics[width=\textwidth]{resol/images/pka1.png}
	\end{subfigure}
	\begin{subfigure}{0.45\textwidth}
		\centering
		\caption{}
		\includegraphics[width=\textwidth]{resol/images/pka2.png}
	\end{subfigure}

	\begin{subfigure}{0.45\textwidth}
		\centering
		\caption{}
		\includegraphics[width=\textwidth]{resol/images/pka3.png}
	\end{subfigure}
	\begin{subfigure}{0.45\textwidth}
		\centering
		\caption{}
		\includegraphics[width=\textwidth]{resol/images/pka4.png}
	\end{subfigure}
	\caption{
		Snapshots of a PKA (500 keV) simulation
		at (a) 0, (b) 1.5, (c) 44.5, and (d) 114.5 ps.
		The Xe atoms are shown in black,
		along with the U (red) and Mo (blue) atoms.
	}
	\label{fig:pka}
\end{figure}


\section{Thermal spike simulations}

% On-centered thermal spike

In contrast, the thermal spike model demonstrated
significant re-solution of Xe atoms.
Figure \ref{fig:spike} showcases several snapshots
of an on-centered thermal spike simulation.
The thermal spike formed a cylindrical volume of liquid
that engulfed the entire Xe gas bubble, leading to its disintegration.
However, the region cooled down significantly after a few ps.
Despite the fact that many Xe atoms
coalesced into a large gas bubble upon cooling,
a substantial number remained in the fuel matrix
or formed other, smaller bubbles.
The thermal spikes also generated shock waves
that propagated radially to the boundary
and created point defects in the system.
The number of point defects decreased with cooling through the sink.
Since substantially greater re-solution was observed in the thermal spike model
than in the PKA model,
we only considered heterogeneous re-solution of Xe atoms
in subsequent calculations.

\begin{figure}[ht]
	\centering
	\begin{subfigure}{0.45\textwidth}
		\centering
		\caption{}
		\includegraphics[width=\textwidth]{resol/images/spike1.png}
	\end{subfigure}
	\begin{subfigure}{0.45\textwidth}
		\centering
		\caption{}
		\includegraphics[width=\textwidth]{resol/images/spike2.png}
	\end{subfigure}

	\begin{subfigure}{0.45\textwidth}
		\centering
		\caption{}
		\includegraphics[width=\textwidth]{resol/images/spike3.png}
	\end{subfigure}
	\begin{subfigure}{0.45\textwidth}
		\centering
		\caption{}
		\includegraphics[width=\textwidth]{resol/images/spike4.png}
	\end{subfigure}
	\caption{
		Snapshots of a thermal spike (30 keV/nm) simulation
		at (a) 0, (b) 1.5, (c) 44.5, and (d) 114.5 ps.
		The Xe atoms are shown in black,
		along with the U (red) and Mo (blue) atoms.
	}
	\label{fig:spike}
\end{figure}

Re-solution data obtained from the on-centered thermal spike simulations
for different bubble sizes are plotted in Figure \ref{fig:frac}(a)
as a fraction of re-solved Xe atoms
over the effective energy transferred to the lattice, $S_{e,eff}$.
Thermal spikes re-solve a greater fraction of Xe atoms in the smaller bubbles
than in the larger bubbles.
Furthermore, the fraction of re-solved Xe atoms seems to saturate
with increasing deposited energy.
Thus, an exponentially saturating function
was used to model the available data:
\begin{align}
	\chi_0 &= 1 - \exp[-\alpha S_{e,eff}]
\end{align}
where $\chi_0$ is the fraction of re-solved atoms
due to an on-centered thermal spike,
and $\alpha$ is the saturation factor.
The saturation factors for different bubble sizes
are plotted in Figure \ref{fig:frac}(b) against the bubble radius.
The graph suggests there is an inverse proportionality
between the saturation factor and the bubble size,
meaning that larger bubbles are more difficult to re-solve completely.
To express this relationship, the saturation factor was made
an inverse power function of bubble radius, as follows:
\begin{align}
	\alpha &= \frac{5.1}{R_{bubble}^{2.2}}
\end{align}

\begin{figure}[ht]
	\centering
	\begin{subfigure}{0.69\textwidth}
		\centering
		\caption{}
		\includegraphics[height=6cm]{resol/images/resolutionVsRadius.pdf}
	\end{subfigure}
	\begin{subfigure}{0.3\textwidth}
		\centering
		\caption{}
		\includegraphics[height=6cm]{resol/images/saturationFactor.pdf}
	\end{subfigure}
	\caption{
		(a) Fraction of re-solved Xe atoms
		as a function of the energy deposited to the lattice.
		(b) Saturation factor as a function of bubble radius.
	}
	\label{fig:frac}
\end{figure}

% Off-centered thermal spike

The results discussed so far have pertained to on-centered thermal spikes.
Off-centered thermal spikes
are expected to have diminishing effects on re-solution,
as a smaller portion of the Xe gas bubble would be enveloped
by the initial thermal spike region.
Furthermore, all re-solution is anticipated to cease
when the initial cylindrical region of the thermal spike
no longer contacts the bubble surface.
Let us define the off-centered distance $r$ as the distance
between the Xe gas bubble center and the cylindrical axis of the thermal spike.
The farthest distance at which the thermal spike contacts the bubble
can be denoted as $r_{c} \equiv R_{bubble} + R_{spike}$.
Simulations were conducted with off-centered distances
over a range of 0 to $r_{c}$, at intervals of 10 \r{A}.

To compare the results obtained from different bubble sizes,
both the Xe re-solution fraction and the off-centered distance were normalized.
The fraction of re-solved Xe atoms $\chi$ was normalized
using the fraction of re-solved Xe atoms
from the on-centered thermal spike $\chi_0$,
and the off-centered distance $r$ was normalized using $r_{c}$,
which varies depending on bubble size.
The normalization enabled comparison of re-solution data
stemming from different bubble sizes.
Figure \ref{fig:off} depicts the normalized fraction of re-solved atoms
$\chi/\chi_0$ as a function of $r/r_c$,
highlighting how the effect of the relative off-centered distance
of the thermal spike remained similar for bubbles of different sizes.
The data were fitted to a logistic function,
due to the plateaus observed at both ends of the off-centered distance data.
The fitted equation was as follows:
\begin{align}
	\label{eq:off}
	\frac{\chi}{\chi_0}
	&= \frac{1.058}{1 + \exp \big[8.168 \big(\frac{r}{r_c}\big) - 3.331 \big]}
\end{align}

\begin{figure}[ht]
	\centering
	\includegraphics[height=7cm]{resol/images/offcentered.pdf}
	\caption{
		Re-solution caused by 15 keV/nm off-centered thermal spikes.
	}
	\label{fig:off}
\end{figure}

% Bubble pressure

To understand the effect of bubble pressure,
bubbles with various Xe/vacancy ratios were simulated.
Figure \ref{fig:pres}(a) shows the number of re-solved Xe atoms
within a 25 \r{A} radius gas bubble as a function of Xe/vacancy ratio.
The number of re-solved atoms is apparently invariant
with respect to Xe/vacancy ratio,
which is positively correlated with bubble pressure.
Similar trends were observed in bubbles with radii of 15 \r{A} and 35 \r{A}.
Consequently, the number of re-solved Xe atoms
from identical-radius bubbles with different pressures were averaged.
The results are presented in Figure \ref{fig:pres}(b).
The number of re-solved Xe atoms also appears consistent
across bubbles of various sizes.
Thus, the number of re-solved Xe atoms does not seem to depend on
either bubble pressure or size for sufficiently large bubbles.
Only the thermal spike energy seems to have a relationship
with the number of re-solved atoms,
as is also apparent in Figure \ref{fig:frac}(a).
One possible explanation for this behavior is that
thermal spikes create low-density regions around the Xe gas bubble,
and these regions facilitate the separation of Xe atoms from the bubble.
Since the thermal spike energy dictates
the volume of these low-density regions,
it also affects the total number of Xe atoms
that can remain in a low-energy configuration outside the original bubble.

\begin{figure}[ht]
	\centering
	\begin{subfigure}{0.49\textwidth}
		\centering
		\caption{}
		\includegraphics[height=6cm]{resol/images/xevac.pdf}
	\end{subfigure}
	\begin{subfigure}{0.49\textwidth}
		\centering
		\caption{}
		\includegraphics[height=6cm]{resol/images/r2dep.pdf}
	\end{subfigure}
	\caption{
		(a) Number of re-solved Xe atoms from 25 \r{A} radius bubbles
		as a function of Xe/vacancy ratio.
		(b) Number of re-solved Xe atoms against bubble radius.
		Data from identical-radius bubbles
		with different Xe/vacancy ratios were averaged.
	}
	\label{fig:pres}
\end{figure}


\section{Re-solution rate}

The re-solution rate can be defined
as the probability of a Xe atom in a gas bubble getting re-solved
into the fuel matrix, per unit time.
The heterogeneous re-solution rate can thus be calculated
based on the fraction of re-solved atoms in a bubble (due to a thermal spike)
and the number of thermal spikes that occur per second.
Both bubble size and pressure are considered
when calculating the bubble re-solution rate.
As stated in the previous section, the number of re-solved Xe atoms
remains constant regardless of Xe/vacancy ratio.
Therefore, the fraction of re-solved Xe atoms (the number of re-solved Xe atoms
divided by the total number of Xe atoms in the system)
is inversely proportional to the Xe/vacancy ratio.
This observation can be used
to decouple the re-solution calculation as follows:
\begin{align}
	b_{het}(R_{bubble}, \dot{F}, \phi)
		&= f(R_{bubble}, \dot{F}) \cdot g(\phi)
\end{align}
where $b_{het}$ is the heterogeneous re-solution rate,
$\dot{F}$ is the fission rate,
and $\phi$ is the Xe/vacancy ratio in the bubble.
The Xe/vacancy ratio, $\phi$, is closely related to the bubble pressure, $P$.
(This relation is discussed later.)
The function $f(R_{bubble}, \dot{F})$ calculates
the re-solution rate of Xe atoms from gas bubbles with $\phi=0.2$,
whereas the function $g(\phi)$ corrects the re-solution rate
for bubbles deviating from the nominal $\phi$ value of 0.2.
We also define $f(R_{bubble}, \dot{F})$
as the nominal heterogeneous re-solution rate.
The subsequent formulation of $f(R_{bubble}, \dot{F})$
draws significant inspiration from the work of Setyawan et al.
\cite{setyawan2018}.

\begin{figure}[ht]
	\centering
	\includegraphics[height=7cm]{resol/images/coordSystem.pdf}
	\caption{
		Cylindrical coordinate system
		for calculating the heterogeneous re-solution rate.
	}
	\label{fig:coord}
\end{figure}

To represent the overall re-solution behavior,
the contributions from all the fission products,
originating at different distances from a bubble
and oriented toward a random direction,
need to be summed up by means of a volume integral.
Consider a cylindrical coordinate system in which the gas bubble
is at the origin (Figure \ref{fig:coord}).
For ease of calculation, fission product origins can be rotated
around the coordinate system origin so that all fission tracks
(as well as the thermal spikes they create) point in the same direction.
Given that the fission product generation
is uniform and isotropic in the material,
the aforementioned rotational transformation will lead to
a uniform distribution of unidirectional fission products.
The axial coordinate $x$ is then defined as parallel to the fission tracks,
and the radial coordinate $r$ as perpendicular to the tracks.
The behavior with respect to the azimuth is constant
because the fission products are unidirectional after the transformation.
If we denote the fission rate as $\dot{F}$,
the number of fission events per second in an infinitesimal volume
$dV = 2 \pi dr dx$ would be $\dot{F} dV$.
Also, if a fission track produced by species $i$
interacts with a Xe gas bubble,
it re-solves $\chi_i$ fraction of the Xe atoms.
Each fission product species $i$ from fission events
thus contributes $\chi_i \dot{F} dV$ to the total re-solution of the bubble.
The nominal heterogeneous re-solution rate can then be expressed as:
\begin{align}
	f(R_{bubble}, \dot{F})
	&= \sum_{i=1}^2 \int_V \chi_i \dot{F} dV \\
	&= \sum_{i=1}^2 \int_{x=0}^{\infty} \int_{r=0}^{\infty}
		\chi_i \dot{F} 2 \pi r dr dx \\
	&= \dot{F} \sum_{i=1}^2 \int_{r=0}^{\infty}
		\bigg( \frac{\chi_i}{\chi_{0,i}} \bigg)
		2 \pi r dr \int_{x=0}^{\infty} \chi_{0,i} dx
\end{align}
where it is assumed that a single fission reaction creates
two fission products, ignoring ternary fission reactions in the process.
Furthermore, the double integral is decoupled into two 1-D integrals,
since $\Big(\frac{\chi_i}{\chi_{0,i}}\Big)$ is only dependent on $r$,
and $\chi_{0,i}$ is dependent on $x$.

We can now compute the $r$ integral by using Equation \ref{eq:off}.
The upper limit of the integral can be changed to $r=r_c$,
since any possibility of re-solution is presumed nonexistent for $r > r_c$.
\begin{align}
	\int_{r=0}^{r_c} \bigg( \frac{\chi_i}{\chi_{0,i}} \bigg) 2 \pi r dr
	&= 2 \pi r_c^2 \int_{r/r_c=0}^{1}
		\frac{1.058}{1 + \exp \big[8.168 \big(\frac{r}{r_c}\big) - 3.331 \big]}
		\bigg(\frac{r}{r_c}\bigg) d\bigg(\frac{r}{r_c}\bigg) \\
	&\approx 0.225 \pi r_c^2
\end{align}

Thus, the nominal heterogeneous re-solution rate
$f(R_{bubble}, \dot{F})$ can be written as:
\begin{align}
	f(R_{bubble}, \dot{F})
	&= \dot{F} \sum_{i=1}^2 \int_{r=0}^{\infty}
		\bigg( \frac{\chi_i}{\chi_{0,i}} \bigg)
		2 \pi r dr \int_{x=0}^{\infty} \chi_{0,i} dx \\
	&= \dot{F} \sum_{i=1}^2 0.225 \pi r_c^2 \int_{x=0}^{\infty}
		[1 - \exp(-\alpha S_{e,eff,i})] dx \\
	\label{eq:dx}
	&= 0.225 \pi r_c^2 \dot{F} \sum_{i=1}^2 \int_{x=0}^{\infty}
		[1 - \exp(-5.1 \zeta S_{e,i} / R_{bubble}^{2.2})] dx
\end{align}
where $S_{e,i}$ is the electronic stopping power of the fission product $i$,
and $\zeta = S_{e,eff} / S_{e}$ is the fraction of fission product energy
imparted to the lattice.
In this work, $\zeta$ is assumed to be an unknown parameter
ranging somewhere between 0.55 and 0.95.
Introducing this parameter is necessary to account for
the fraction of the fission product energy
that is not transferred to the thermal spike.

% Electronic stopping of fission products

We chose to evaluate the $S_e$ values for Xe-140 and Sr-94 in $\gamma$U-10Mo,
based on the fission reaction
$_0^1n + _{92}^{235}U \rightarrow _{54}^{140}Xe + _{38}^{94}Sr + Q$.
Despite this being one of many fission reactions that may occur,
Xe-140 and Sr-94 can still be considered representative of
heavy and light fission products.
This is because the two aforementioned species
appear close to the two peaks in a typical fission product yield graph
\cite{setyawan2018, mills1995}.
Figure \ref{fig:elec} shows $S_{e,Xe}$ and $S_{e,Sr}$
as a function of the distance traversed by the fission product.
These data were generated using the SRIM software.
The following equations, similar to the ones used in \cite{setyawan2018},
were utilized to represent the $S_e$ data:
\begin{align}
	\label{eq:xe}
	S_{e,Xe} &= 21.3 \exp(-0.239 x^{1.78})
		+ 5.23 \exp(-4.67 \times 10^{-8} x^{11}) \\
	\label{eq:sr}
	S_{e,Sr} &= 19.7 \exp(-0.00273 x^{3.71})
		+ 6.8 \exp(-0.424 x^{1.45})
\end{align}

\begin{figure}[ht]
	\centering
	\includegraphics[height=7cm]{resol/images/elec_stopping.pdf}
	\caption{
		Total electronic stopping power ($S_e$) of Xe-140 and Sr-94
		in $\gamma$U-10Mo, as calculated by the SRIM software
		as a function of distance traversed by the fission product
		from the location of the fission reaction.
	}
	\label{fig:elec}
\end{figure}

Finally, Equations \ref{eq:dx}, \ref{eq:xe}, and \ref{eq:sr} can be used
to calculate the re-solution rate for different values of $\zeta$
and bubble radius $R_{bubble}$.
Figure \ref{fig:res} displays the results for a fission rate of
$10^{14}$ fiss/cm$^3$/s since usual fission rates in the $\gamma$U-10Mo fuel
are of this order \cite{annualreport2021}.
The re-solution rates were computed using numerical integration,
and the data can be extracted from the associated graphs
or by recalculating the integrals.
However, an approximate analytical function
might better serve the higher-length-scale models.
To that end, we propose the following function
for fitting the computed re-solution rate:
\begin{align}
	\label{eq:param}
	f(R_{bubble}, \dot{F})
	&= \bigg[ \frac{a}{1 + (R_{bubble} / c)^d} \bigg]
		(10^{-14} \: \dot{F})
\end{align}
where $R_{bubble}$ is in \r{A},
$\dot{F}$ is in fissions per cubic centimeter per second (fiss/cm$^3$/s),
and $a$, $c$, and $d$ are adjustable parameters.
The parameter values for different $\zeta$ are listed in Table 1.
For all $\zeta$ values, the parameter $d$ remains consistent
up to three decimal places.
To determine the re-solution rate of any arbitrary $\zeta$
of between 0.55 and 0.95, linear interpolation would be sufficient.

\begin{figure}[ht]
	\centering
	\includegraphics[height=7cm]{resol/images/resRate.pdf}
	\caption{
		Xe gas bubble re-solution rate in $\gamma$U-10Mo,
		as a function of bubble radius,
		at a fission rate of $10^{14}$ fiss/cm$^3$/s
		and for a Xe/vacancy ratio of 0.2.
	}
	\label{fig:res}
\end{figure}

\begin{table}[ht]
\centering
\caption{Equation \ref{eq:param} parameter fits for different $\zeta$ values.}
\label{tab:zeta}
\begin{tabular}{llll}
\toprule
$\zeta$     & $a$        & $c$     & $d$      \\
\midrule
0.55        & 0.0167     & 2.717   & 1.225    \\
0.65        & 0.0163     & 3.184   & 1.225    \\
0.75        & 0.0161     & 3.635   & 1.225    \\
0.85        & 0.0159     & 4.073   & 1.225    \\
0.95        & 0.0158     & 4.502   & 1.225    \\
\bottomrule
\end{tabular}
\end{table}

% Multiply the pressure factor

Now consider the effect of $\phi$ (and $P$) on re-solution.
Since the number of re-solved atoms is constant with respect to $\phi$
for a specific bubble size (Figure \ref{fig:pres}(a)),
bubbles with more Xe atoms (and thus more pressure)
will have a lower fraction of re-solved atoms.
This leads to the following simple expression:
\begin{align}
	g(\phi) &= \frac{0.2}{\phi}
\end{align}

The Xe/vacancy ratio, $\phi$, can be used
to obtain the molar volume of Xe, $v$.
At 400 K, the equilibrium volume of $\gamma$U-Mo is $19.7$ \r{A}$^3$/atom.
Thus, $v$ can be expressed as:
\begin{align}
	v
	&= \frac{N_A \cdot 19.7}{\phi} \text{ \r{A}}^3/\text{mol} \\
	\label{eq:v}
	&= \frac{11.86}{\phi} \text{ cm}^3/\text{mol}
\end{align}
where $N_A$ is the Avogadro constant.

With the molar volume now known, the pressure, P,
can be computed via the Xe bubble equation of state (EOS),
as provided by Beeler et al. \cite{beelerADP}.
The Virial EOS is expanded to the third order with respect to volume,
and to the second order with respect to temperature.
The EOS is as follows:
\begin{align}
	\label{eq:eos}
	P &= \frac{RT}{v}
		\bigg( A + \frac{B}{v} + \frac{C}{v^2} + \frac{D}{v^3} \bigg)
\end{align}
where $R$ is the gas constant ($8.3145$ J/mol-K),
and $A=1$, $B=151.12 \text{ cm}^3/\text{mol}$,
$C=2976 \text{ cm}^6/\text{mol}^2$, and $D=705527 \text{ cm}^9/\text{mol}^3$
at temperature $T=400$ K.
Combining Equations \ref{eq:v} and \ref{eq:eos} provides
a direct relationship between $\phi$ and $P$:
\begin{align}
	\label{eq:pres}
	P &= (280.4 \: \phi + 3573 \: \phi^2 + 5933 \: \phi^3
		+ 118600 \: \phi^4) \: \text{ MPa} \\
	  &\approx (5894.417 \: \phi^2 + 123341.14 \: \phi^4) \: \text{ MPa},
		\quad 0.05 \leq \phi \leq 0.50
\end{align}
where $P$ is approximated as a quadratic function of $\phi^2$
(by assuming an equation of the form $P = c_1 \phi^2 + c_2 \phi^4$
and fitting it to a set of data generated using Equation \ref{eq:pres})
so as to make algebraic manipulation easier.
Consequently, $g(\phi)$ can be expressed as follows:
\begin{align}
	g(\phi)
	= \frac{0.2}{\phi}
	= \frac{k}{(l+\sqrt{l^2+mP})^{1/2}}
	\coloneq h(P)
\end{align}
where $P$ is in MPa, $k=99.334$ (MPa)$^{1/2}$, $l=-5894.417$ MPa,
and $m=493364.56$ MPa.
Both $g(\phi)$ and $h(P)$ are unitless.

Now that we have two equivalent functions of $\phi$ and $P$,
the heterogeneous re-solution rate takes the following forms:
\begin{align}
	b_{het}(R_{bubble}, \dot{F}, \phi)
		&= f(R_{bubble}, \dot{F}) \cdot g(\phi)
		= f(R_{bubble}, \dot{F}) \cdot h(P) \\
		&= \bigg[ \frac{a}{1+(R_{bubble}/c)^d} 10^{-14} \dot{F} \bigg]
			\bigg( \frac{0.2}{\phi} \bigg) \label{eq:g} \\
		&= \bigg[ \frac{a}{1+(R_{bubble}/c)^d} 10^{-14} \dot{F} \bigg]
			\bigg[ \frac{k}{(l + \sqrt{m+nP})^{1/2}} \bigg] \label{eq:h}
\end{align}
Equations \ref{eq:g} and \ref{eq:h} entirely define the model
developed in this work to predict the Xe gas bubble re-solution rate.
Figure \ref{fig:3d} depicts the re-solution rate as a function of
both bubble radius and Xe/vacancy ratio calculated using equation \ref{eq:g}.
It is evident from the figure that the re-solution rate
decreases with both these variables.

\begin{figure}[ht]
	\centering
	\includegraphics[height=7cm]{resol/images/3d.pdf}
	\caption{
		Xe gas bubble re-solution rate in $\gamma$U-10Mo
		as a function of bubble radius and Xe/vacancy ratio
		at a fission rate of $10^{14}$ fiss/cm$^3$/s for $\zeta=0.75$.
	}
	\label{fig:3d}
\end{figure}


\section{Discussion}

Re-solution rate is a crucial parameter
for calculating fission gas swelling in $\gamma$U-10Mo.
The mesoscale software program DART incorporates
this parameter for such calculations \cite{ye2023}.
The currently used re-solution rate $b_{dart}$ in DART is defined as follows:
\begin{align}
	b_{dart} &= b_0 \cdot \dot{F} \cdot G \\
	b_0 &= R_{spike}^2 \cdot \mu_{ff} \\
	G &=
	\begin{cases}
		1 & ,R_{bubble} \leq \lambda \\
		1 - (\frac{R_{bubble}-R_{resol}}{R_{bubble}})^3
		  & ,R_{bubble} > \lambda
	\end{cases}
\end{align}
where $b_0$ is the bubble destruction probability,
$\dot{F}$ is the fission rate,
and $G$ is a piecewise function representing different re-solution modes
for small and large gas bubbles.
The parameter $b_0$ can be estimated based on the interaction volume
of a thermal spike with bubbles,
per the formula $b_o = R_{spike}^2 \times \mu_{ff}$,
where $R_{spike}$ is the radius of a thermal spike
and $\mu_{ff}$ is the recoil length of fission fragments.
In the piecewise function G, $R_{bubble}$ is the bubble radius,
$\lambda$ is the gas-atom knock out distance,
and $R_{resol}$ is the thickness of the annulus
within which all gas-atoms are knocked out.
The parameters $b_0$, $\lambda$, and $R_{resol}$ are considered adjustable,
and the optimized values are $b_0 = 2 \times 10^{-18}$ cm$^3$,
$\lambda = 5 \times 10^{-7}$ cm,
and $R_{resol} = 3 \times 10^{-9}$ cm.
Since the parameters $\lambda$ and $R_{resol}$ are not coupled,
the re-solution rate is discontinuous at a bubble radius $\lambda$.
The rationale behind the step function-like implementation
of the re-solution rate is to delineate intragranular bubbles
from intergranular bubbles.
According to Ye et al. \cite{ye2023}, this is to account for
the strong trapping effects of grain boundaries.
However, they also asserted that a smoother transition
between the two different bubble regimes is warranted.

The re-solution rate calculated in this work is based solely
on the simulation of intragranular bubbles.
We assumed the trapping effect of the grain boundaries to be insignificant
over the timescale of the thermal spikes, thus it was not considered.
Also, accounting for trapping in the re-solution rate
contradicts the definition of re-solution.
To model gas bubble evolution, the re-solution rate and trapping rate
should be implemented separately in higher-length-scale programs,
thereby affording greater freedom in
modeling complex behavior of gas bubbles in the fuel, under different contexts.
For example, the grain boundary diffusion coefficient of Xe in $\gamma$U-Mo
can be 15 orders of magnitude higher than the intrinsic diffusion coefficient
of Xe in $\gamma$U-Mo at around 600 K \cite{hasan2024gb}.
Thus, the Xe trapping rate for intergranular bubbles can be set 15 times higher
than that for intragranular bubbles.
Comparison between the re-solution rate computed in the present work
and the rate used in DART can elucidate their differences.
Figure \ref{fig:dart} shows the re-solution rates
for bubble radii up to 100 \r{A}.
Unlike the DART model prediction, the re-solution rate computed
in the present work is a smooth decaying function of bubble radius.
For very small bubbles, the calculated re-solution rate
is about two orders of magnitude higher than the DART prediction.
For bubbles with a radius of around 50 \r{A},
the difference is less than one order of magnitude.
This difference suddenly increases to about three orders of magnitude
for bubbles having greater than 50 \r{A} radii
due to the discontinuous nature of the rate in DART.
The $\zeta$ value does not impact this comparison,
as the re-solution rate at $\zeta=0.95$ is, at most,
double that at $\zeta=0.55$.
The large bubble size regime ($R_{bubble} > 50$ \r{A}) is where
DART adds a strong trapping effect to the re-solution rate,
and thus the rate does not represent a pure re-solution rate.

\begin{figure}[ht]
	\centering
	\includegraphics[height=7cm]{resol/images/resRate_withDart.pdf}
	\caption{
		Comparison of the calculated nominal re-solution rate
		against the DART model prediction \cite{ye2023}.
	}
	\label{fig:dart}
\end{figure}
