\begin{abstract}

The United States High-Performance Research Reactor (USHPRR) program
seeks to replace highly enriched uranium (HEU) fuels in research reactors
with low enriched uranium (LEU) fuels to reduce nuclear proliferation risks.
This transition necessitates the development of high-density uranium fuels,
among which $\gamma$U-Mo alloys have emerged as a promising candidate
due to their favorable thermal and mechanical properties under irradiation.
Despite their advantages, significant knowledge gaps exist
in the material properties of $\gamma$U-Mo,
limiting the accuracy of engineering-scale fuel performance models.

This dissertation aims to address these gaps using computational techniques,
focusing on a few key aspects of $\gamma$U-Mo fuel behavior.
First, classical molecular dynamics (MD) simulations
are employed to determine the diffusion coefficients
of uranium, molybdenum, and xenon in grain boundaries (GBs) of $\gamma$U-Mo.
These diffusion parameters are crucial for improving
models of fuel swelling, gas bubble evolution, and irradiation creep.
Second, the re-solution mechanisms of fission gas bubbles in $\gamma$U-Mo
are being investigated through MD simulations,
considering both homogeneous and heterogeneous re-solution pathways.
This study will quantify the re-solution rate
as a function of bubble size, pressure, and fission rate.
Lastly, inverse uncertainty quantification (IUQ) methods are applied
to estimate unknown fission-gas-behavior parameters
used in the Dispersion Analysis Research Tool (DART),
a meso-scale fuel performance modeling code.
Bayesian inference techniques and surrogate modeling are employed
to derive probability distributions of these parameters,
ensuring improved accuracy in fuel swelling predictions.

The insights gained from this research will enhance
the predictive capabilities of fuel performance models,
supporting the qualification and deployment of $\gamma$U-Mo
as a viable LEU fuel for high-performance research reactors.
By providing a comprehensive computational framework
for understanding diffusion behavior, fission gas interactions,
and parameter uncertainties,
this study will contribute to the broader effort of ensuring
the safety and efficiency of next-generation nuclear fuels.

\end{abstract}
