\documentclass[10pt]{beamer}
\usepackage{booktabs, subcaption, multirow, xspace}
\usepackage{amsmath, amssymb, bm, siunitx}

% Ion shorthand
\usepackage[version=4]{mhchem}
\newcommand{\Y}{\ce{_{39}^{97}Y}\xspace}
\newcommand{\I}{\ce{_{53}^{136}I}\xspace}

\usetheme{CambridgeUS}
\usecolortheme{beaver}
\setbeamertemplate{caption}[numbered]
\setbeamerfont{caption}{size=\scriptsize}

\AtBeginSection[]
{
  \begin{frame}
    \frametitle{Table of Contents}
    \tableofcontents[currentsection]
  \end{frame}
}

\title[Evaluation of U-Mo Microstructural Properties]
{Evaluation of Microstructural Properties
of U-Mo Monolithic Fuel using Computational Methods}

\author{ATM Jahid Hasan}
\institute[NCSU]{Department of Nuclear Engineering\\
North Carolina State University}
\date{January 28, 2026}

\begin{document}
\frame{\titlepage}

\begin{frame}{Table of Contents}
	\tableofcontents
\end{frame}

\section{Introduction}

\begin{frame}{Introduction}
	\begin{columns}
	\column{0.5\textwidth}
		\begin{itemize}
			\item Research reactors operate at low temperature and pressure
				but at high fission density and specific power.
				Thus, RRs historically needed highly enriched uranium (HEU).
			\item With the inception of RERTR in 1978,
				there has been an ongoing effort
				to change HEU into low enriched uranium (LEU).
			\item U$_3$Si$_2$ is the only qualified LEU for RRs.
			\item High-performance RRs need even higher uranium density.
		\end{itemize}
	\column{0.5\textwidth}
		\begin{figure}[ht]
			\centering
			\includegraphics[width=6cm]{sfigs/historic_fuels.png}
			\caption{
				U-density of research reactor fuel vs time of first use.
				(Jamison et al.
				{\color{blue}\url{https://doi.org/10.2172/1890226}}).
			}
		\end{figure}
	\end{columns}
\end{frame}

\begin{frame}{Introduction (cont.)}
	\begin{columns}
	\column{0.5\textwidth}
		\begin{itemize}
			\item US High-Performance Research Reactor (USHPRR) program
				selected U-10Mo fuel for qualification.
			\item U-10Mo monolithic fuel design involves
				a solid U-10Mo fuel meat monded to Al-6061 cladding
				with a zirconium diffusion barrier.
			\item The U-Mo dispersion fuel design has also been tested.
				The dispersion fuel shows breakaway swelling,
				associated with the formation of Mo-stabilized high-Al
				intermetallic phase.
		\end{itemize}
	\column{0.5\textwidth}
		\begin{figure}[ht]
			\centering
			\includegraphics[width=5cm]{sfigs/umo_design.jpg}
			\caption{
				Depiction of monolithic fuel cross-section
				(Meyer et al.
				{\color{blue}\url{https://doi.org/10.5516/NET.07.2014.706}}).
			}
		\end{figure}
	\end{columns}
\end{frame}

\begin{frame}{Swelling in $\gamma$U-Mo}
	\begin{columns}
	\column{0.5\textwidth}
		\begin{itemize}
			\item Excessive swelling during operation is a major concern
				for $\gamma$U-Mo fuels.
			\item With increasing bunrup,
				the bubble population increases in the grain boundaries
				and newer grain boundaries form.
			\item This grain refinement is accompanied by an increase
				in bubble size and number,
				which increases fuel swelling rate.
			\item To account for it in the fuel design,
				fuel swelling needs to be predicted
				for different operational conditions.
		\end{itemize}
	\column{0.5\textwidth}
		\begin{figure}[ht]
			\centering
			\includegraphics[width=4cm]{sfigs/umo_burnup.png}
			\caption{
				U-Mo microstructure as a function of burnup
				(Kim et al. {\color{blue}
				\url{https://doi.org/10.1016/j.jnucmat.2011.08.018}}).
			}
		\end{figure}
	\end{columns}
\end{frame}

\begin{frame}{Dispersion Analysis Research Tool (DART)}
	\begin{columns}
	\column{0.5\textwidth}
		\begin{itemize}
			\item DART is a mechanistic rate-theory-based meso-scale model
				for the calculation of fuel swelling.
			\item The code uses material properties,
				such as gas atom diffusivity, recrystallization kinetics,
				and gas re-solution rate.
			\item Many of material properties used by DART are currently unknown.
			\item These properties can be gleaned
				from lower-length-scale studies
				or through parameter optimization using experimental data.
		\end{itemize}
	\column{0.5\textwidth}
		\begin{figure}[ht]
			\centering
			\includegraphics[width=6cm]{sfigs/dart_schematic.png}
			\caption{
				DART schematic.
				(Ye et al. {\color{blue}
				\url{https://doi.org/10.1016/j.jnucmat.2023.154542}}).
			}
		\end{figure}
	\end{columns}
\end{frame}

\begin{frame}{Grain boundary (GB) diffusion in $\gamma$U-Mo}
	\begin{itemize}
		\item One of the unknown properties is the grain boundary (GB)
			diffusion enhancement factor.
		\item Accurate calculation of fuel swelling requires
			diffusion coefficients of the related species in the fuel
			since these coefficients determine
			the fission gas (Xe) trapping rate of the bubbles.
		\item Creep modeling also requires diffusion coefficients
			to determine creep rates and evaluate the evolving microstructure.
		\item The diffusion coefficients of the relevant species
			in $\gamma$U-Mo grain boundaries (GBs) are yet unknown.
	\end{itemize}
\end{frame}

\begin{frame}{Xe gas bubble re-solution}
\begin{itemize}
	\item Xe gas bubbles act as a sink for individual Xe atoms,
		trapping them and causing the bubbles to grow after absorption.
	\item Under irradiation, the Xe atoms in the gas bubble are reintroduced
		into the fuel matrix through fission-product-induced cascades
		and thermal spikes---a process known as re-solution.
	\item The relative rates of re-solution
		affect the overall size and number density of the bubbles,
		in turn impacting bubble evolution and subsequent fuel swelling.
	\item There is no physical model for Xe re-solution rate in U-Mo yet.
\end{itemize}
\end{frame}

\begin{frame}{Other fission-gas-behavior parameters}
	\begin{itemize}
		\item There are many other unknown parameters, such as
			the bubble nucleation probability,
			the average grain size of high burnup structures
			and the threshold areal coverage needed for GB interconnection.
		\item For the parameters that do not have measurement data
			or atomic-scale simulation results,
			they are usually estimated by either fitting to measured bubble morphology
			or by borrowing from other similar fuel systems where the data is available.
		\item The fission-gas-behavior parameters used in
			the gas release and swelling subroutine (GRASS) of DART
			were calibrated in a previous study,
			using the bubble size distributions
			measured from irradiated U-10Mo dispersion fuel particles.
		\item However, this set of parameters needs recalibration
			because new atomic-scale data have become available
			since the previous calibration.
	\end{itemize}
\end{frame}

\begin{frame}{Objectives}
	The objectives of this dissertation are:
	\begin{itemize}
		\item Computing the diffusivities of U, Mo, and Xe in $\gamma$U-Mo GBs.
		\item Investigating the Xe re-solution behavior in the U-10Mo fuel.
		\item Performing inverse uncertainty quantification (IUQ)
			of unknown fission-gas-behavior parameters used in DART.
	\end{itemize}
\end{frame}

\section{Grain boundary diffusion in \texorpdfstring{$\gamma$}{gamma}U-Mo}

\begin{frame}{Simulation setup}
	\begin{columns}
	\column{0.5\textwidth}
		\begin{itemize}
			\item Bicrystals for molecular dynamics (MD) simulations were created
				by rotating two halves of the system around a tilt axis.
			\item Two GBs: one in the middle and one at the edge.
			\item GB region is defined using atomic trajectories.
			\item We tracked lattice point jumps.
				GB region can be differentiated from the bulk
				by determining lattice points associated with jumps.
		\end{itemize}
	\column{0.5\textwidth}
		\begin{figure}[ht]
			\centering
			\begin{subfigure}{1.0\textwidth}
				\centering
				\includegraphics[width=\textwidth]
					{diffusion/images/configuration.png}
			\end{subfigure}

			\begin{subfigure}{0.7\textwidth}
				\centering
				\includegraphics[width=0.7\textwidth]
					{diffusion/images/gb_def.png}
			\end{subfigure}
			\caption{
				(Top) $\Sigma 5$ Symmetric tilt GB initial configuration.
				(Bottom) Definition of GB based on atomic trajectories.
			}
		\end{figure}
	\end{columns}
\end{frame}

\begin{frame}{Explored GBs}
	\begin{itemize}
		\item Three compositions: $\gamma$U-7Mo, $\gamma$U-10Mo, and $\gamma$U-12Mo
			were examined.
		\item The temperature of the simulated GBs ranged from 600 K to 1200 K
			with an interval of 100 K.
			The dimensions of the supercells were at least
			$50 \times 200 \times 50$ \AA$^3$.
			This corresponds to at least 35,000 atoms.
		\item The following symmetric tilt GBs were studied:
			\begin{itemize}
				\item (120)
				\item (130)
				\item (150)
				\item (190)
				\item (340)
				\item (350)
			\end{itemize}
		\item Other simulated systems were:
			\begin{itemize}
				\item asymmetric (110)
				\item asymmetric (130)
				\item asymmetric (190)
				\item asymmetric (350)
				\item twist (110)
				\item twist (230)
			\end{itemize}
	\end{itemize}
\end{frame}

\begin{frame}{GB validation}
	\begin{columns}
	\column{0.5\textwidth}
		\begin{itemize}
			\item For validation, we computed GB energies
				with the configurations first.
			\item The energies are within 2$\sigma$ of the values
				reported by Beeler et al.
			\item Beeler et al. employed roughly 10 times fewer atoms.
			\item The average GB energy for the examined systems
				is about 0.68 Jm$^{-2}$.
		\end{itemize}
	\column{0.5\textwidth}
		\begin{figure}[ht]
			\centering
			\includegraphics[width=6cm]{diffusion/images/gbe.pdf}
			\caption{
				GB energies as a function of misorientation angle.
			}
		\end{figure}
	\end{columns}
\end{frame}

\begin{frame}{GB width}
	\begin{columns}
	\column{0.5\textwidth}
		\begin{itemize}
			\item GB width in this work means the structural width of a GB.
			\item This is different from the diffusional width,
				where width is calculated using the number of mobile GB atoms.
			\item GB width increases linearly with temperature.
			\item The width is about 6 \r{A} at 600 K
				and increases to about 12 \r{A} at 1200 K.
			\item In literature, GB width assumptions fall between
				5 \r{A} to 15 \r{A}.
		\end{itemize}
	\column{0.5\textwidth}
		\begin{figure}[ht]
			\centering
			\includegraphics[width=6cm]{diffusion/images/d_gb_sym.pdf}
			\caption{
				Widths of symmetric tilt GBs as a function of temperature.
			}
		\end{figure}
	\end{columns}
\end{frame}

\begin{frame}{Symmetric tilt GB diffusivities}
	\begin{figure}[ht]
		\centering
		\begin{subfigure}{0.325\textwidth}
			\centering
			\includegraphics[width=\textwidth]
				{diffusion/images/u10mo_U_Dz.pdf}
		\end{subfigure}
		\begin{subfigure}{0.325\textwidth}
			\centering
			\includegraphics[width=\textwidth]
				{diffusion/images/u10mo_Mo_Dz.pdf}
		\end{subfigure}
		\begin{subfigure}{0.325\textwidth}
			\centering
			\includegraphics[width=\textwidth]
				{diffusion/images/u10mo_Xe_Dz.pdf}
		\end{subfigure}
		\caption{
			Diffusion coefficients parallel to the GB tilt axis
			for symmetric tilts GBs.
		}
	\end{figure}

	\begin{itemize}
		\item Diffusion behavior is generally Arrhenius.
			The spread in diffusion coefficients due to tilt angles
			is about one order or magnitude.
		\item Diffusivity ranges from $10^{-14}$
			to $10^{-11}$ m$^2$s$^{-1}$.
			The activation energy is about 0.4--0.7 eV for U.
		\item At high temperatures,
			$D_{\parallel}^{Xe} > D_{\parallel}^U > D_{\parallel}^{Mo}$.
			At low temperatures, all three species have similar coefficients.
	\end{itemize}
\end{frame}

\begin{frame}{Asymmetric tilt \& Twist GB diffusivities}
	\begin{figure}[ht]
		\centering
		\begin{subfigure}{0.45\textwidth}
			\centering
			\includegraphics[width=\textwidth]
				{diffusion/images/asym_twist_U_Dz.pdf}
		\end{subfigure}
		\begin{subfigure}{0.45\textwidth}
			\centering
			\includegraphics[width=\textwidth]
				{diffusion/images/asym_twist_Mo_Dz.pdf}
		\end{subfigure}
		\caption{
			Diffusion coefficients parallel to the GB tilt axis
			for asymmetric tilt and twist GBs.
		}
	\end{figure}

	\begin{itemize}
		\item The diffusion behavior is similar
			for asymmetric tilt and twist GBs.
		\item U diffusivity is higher than that of Mo.
		\item Overall impact of orientation of GBs on diffusivity
			appears to be minimal.
	\end{itemize}
\end{frame}

\begin{frame}{Orientation-averaged GB diffusivities}
	\begin{figure}[ht]
		\centering
		\begin{subfigure}{0.45\textwidth}
			\centering
			\includegraphics[width=\textwidth]
				{diffusion/images/comp_U_Dz.pdf}
		\end{subfigure}
		\begin{subfigure}{0.45\textwidth}
			\centering
			\includegraphics[width=\textwidth]
				{diffusion/images/comp_Mo_Dz.pdf}
		\end{subfigure}
		\caption{
			Orientation-averaged GB diffusivities parallel to the tilt axis
			for $\gamma$U-7Mo, $\gamma$U-10Mo, and $\gamma$U-12Mo.
		}
	\end{figure}

	\begin{itemize}
		\item To compare diffusion of different compositions,
			diffusivities parallel to the tilt axis of all symmetric tilt GBs
			are averaged.
		\item GB diffusivity is negatively correlated with the Mo content.
	\end{itemize}
\end{frame}

\begin{frame}{Comparison with literature}
	\begin{columns}
	\column{0.5\textwidth}
		\begin{itemize}
			\item The difference between GB diffusivity and self-diffusivity
				grows larger with decreasing temperature.
			\item With GB diffusion enhancement factors and GB widths,
				we can calculate effective diffusion coefficient $D_{eff}$:
				$ D_{eff} = f D_{gb} + (1-f) D_{\ell} $,
				where $D_{\ell}$ is the bulk diffusion coefficient
				and $f$ is the volume fraction of GBs.
			\item Assuming 10 $\mu$m grain size,
				we find that $D_{eff} \approx 1.5 \times 10^3 D_{\ell}$ at 600 K
				and $D_{eff} \approx 1.1 D_{\ell}$ at 1200 K.
			\item Overall diffusion is dominated by GB diffusion
				at lower temperatures.
		\end{itemize}
	\column{0.5\textwidth}
		\begin{figure}[ht]
			\centering
			\includegraphics[width=0.9\textwidth]
				{diffusion/images/newLitComp.pdf}
			\caption{
				GB vs bulk diffusivities.
			}
		\end{figure}
	\end{columns}
\end{frame}

\begin{frame}{Effect of the misorientation angle}
	\begin{figure}[ht]
		\centering
		\begin{subfigure}{0.45\textwidth}
			\centering
			\includegraphics[width=\textwidth]
				{diffusion/images/DvsTilt_600K.pdf}
		\end{subfigure}
		\begin{subfigure}{0.45\textwidth}
			\centering
			\includegraphics[width=\textwidth]
				{diffusion/images/DvsTilt_1200K.pdf}
		\end{subfigure}
		\caption{
			Diffusivites against misorientation angle at 600 K and 1200 K.
		}
	\end{figure}

	\begin{itemize}
		\item U GB diffusivity is higher than that of Mo and Xe at 600 K.
			At 1200 K, U GB diffusivity is still higher than that of Mo.
			However, Xe diffusivity becomes larger than that of U.
		\item In general, the greater the misorientation angle,
			the greater the diffusivity.
	\end{itemize}
\end{frame}

\begin{frame}{Anisotropic nature of diffusion}
	\begin{columns}
	\column{0.5\textwidth}
		\begin{itemize}
			\item For some GBs like (190), diffusion is similar to
				dislocation pipe diffusion.
			\item A symmetric tilt low-angle GB is essnetially
				an array of parallel edge dislocations.
			\item Even at high temperatures,
				the anisotropic nature of diffusion remains.
			\item Isotropic behavior of (130) is shown for comparison.
		\end{itemize}
	\column{0.5\textwidth}
		\begin{figure}[ht]
			\centering
			\begin{subfigure}{0.48\textwidth}
				\centering
				\caption{}
				\includegraphics[width=\textwidth]
					{diffusion/images/130at700cs.png}
			\end{subfigure}
			\begin{subfigure}{0.48\textwidth}
				\centering
				\caption{}
				\includegraphics[width=\textwidth]
					{diffusion/images/130at1100cs.png}
			\end{subfigure}
			\begin{subfigure}{0.48\textwidth}
				\centering
				\caption{}
				\includegraphics[width=\textwidth]
					{diffusion/images/190at700cs.png}
			\end{subfigure}
			\begin{subfigure}{0.48\textwidth}
				\centering
				\caption{}
				\includegraphics[width=\textwidth]
					{diffusion/images/190at1100cs.png}
			\end{subfigure}
			\caption{
				Cross-sectional view of atomic trajectories at
				(a) 700 K and (b) 1100 K of (130),
				and (c) 700 K and (d) 1100 K of (190).
			}
		\end{figure}
	\end{columns}
\end{frame}

\begin{frame}{Anisotropic nature (cont.)}
	\begin{columns}
	\column{0.5\textwidth}
		\begin{itemize}
			\item To add to the visual example,
				mean squared displacements (MSDs) are shown here.
			\item In the isotropic case,
				MSDs parallel and perpendicular to the tilt axis
				are quite similar.
			\item In the anisotropic case, MSD perpendicular to the tilt axis
				is similar to MSD perpendicular to the GB plane,
				indicating an almost one dimensional diffusion.
		\end{itemize}
	\column{0.5\textwidth}
		\begin{figure}[ht]
			\centering
			\begin{subfigure}{0.95\textwidth}
				\centering
				\includegraphics[height=3cm]
					{diffusion/images/130at1100xyz.pdf}
			\end{subfigure}
			\begin{subfigure}{0.95\textwidth}
				\centering
				\includegraphics[height=3cm]
					{diffusion/images/190at1100xyz.pdf}
			\end{subfigure}
			\caption{
				Mean squared displacements of
				(Top) (130) and (Bottom) (190) at 1100 K.
			}
		\end{figure}
	\end{columns}
\end{frame}

\begin{frame}{Conclusions}
	\begin{itemize}
		\item It was observed that the GB width increases almost linearly
			from around 6 \r{A} to around 12 \r{A} with temperature
			in the range $600$ K -- $1200$ K.
		\item The GB diffusion coefficients are typically
			on the order of $10^{-14}$ to $10^{-11}$ m$^2$s$^{-1}$.
		\item The U GB diffusivity is always higher than the Mo GB diffusivity
			for any GB at a given temperature.
			There is also a significant negative correlation between
			the GB diffusivities and the concentration of Mo in the alloy.
		\item The orientation-averaged diffusion coefficients of U, Mo, and Xe
			are between three to fifteen orders of magnitude faster
			than the intrinsic/self-diffusion coefficients
			of the species in the bulk,
			and GB diffusion dominates the effective diffusion of the material
			at lower temperatures.
		\item Low angle GBs show anisotropic diffusion.
	\end{itemize}
\end{frame}

\section{Re-solution of Xe gas bubbles in U-10Mo}

\begin{frame}{Fission gas bubble re-solution}
	\begin{columns}
	\column{0.5\textwidth}
		\begin{itemize}
			\item Fission strongly limits bubble size in the fuel lattice
				by reintroducing the atoms in the bubble
				into the fuel matrix---a process known as re-solution.
			\item In homogeneous re-solution,
				individual gas atoms are ejected from the gas bubble
				through collisions with energetic fission fragments
				or recoil atoms (PKAs).
			\item In heterogeneous re-solution,
				bubbles are destroyed by the passage of fission tracks
				where local temperature can be higher than
				the fuel melting temperature.
		\end{itemize}
	\column{0.5\textwidth}
		\begin{figure}[ht]
			\centering
			\begin{subfigure}{0.6\textwidth}
				\centering
				\includegraphics[width=\textwidth]{sfigs/homogeneous.png}
			\end{subfigure}

			\begin{subfigure}{0.6\textwidth}
				\centering
				\includegraphics[width=\textwidth]{sfigs/heterogeneous.png}
			\end{subfigure}
			\caption{
				Illustration of
				(Top) homogeneous and (Bottom) heterogeneous re-solution.
			}
		\end{figure}
	\end{columns}
\end{frame}

\begin{frame}{Energy loss of fission fragments}
	\begin{columns}
	\column{0.5\textwidth}
		\begin{itemize}
			\item Binary collision approximation (BCA) simulations
				of two representative fission fragments (FFs) \Y and \I
				were performed using RustBCA.
		\end{itemize}
		\begin{align}
			\ce{
				_0^1n + _{92}^{235}U -> _{39}^{97}Y + _{53}^{136}I + Q
			}
		\end{align}
		\begin{itemize}
			\item Peak electronic stopping power is just shy of $20$ keV/nm
				for both FFs.
			\item Nuclear stopping power has a peak at the end of FF trajectories.
		\end{itemize}
	\column{0.5\textwidth}
		\begin{figure}[ht]
			\centering
			\begin{subfigure}{0.6\textwidth}
				\centering
				\includegraphics[width=\textwidth]{resol2/images/Y_stopping.pdf}
			\end{subfigure}

			\begin{subfigure}{0.6\textwidth}
				\centering
				\includegraphics[width=\textwidth]{resol2/images/I_stopping.pdf}
			\end{subfigure}
			\caption{
				Nuclear and electronic stopping powers of (Top) \Y and (Bottom) \I.
			}
		\end{figure}
	\end{columns}
\end{frame}

\begin{frame}{Heterogeneous re-solution}
	\begin{columns}
	\column{0.5\textwidth}
		\begin{itemize}
			\item MD simulations with the two-temperature model (TTM)
				were performed to simulate thermal spikes
				due to electronic stopping.
			\item No re-solution has been observed in any of the simulations.
			\item Since U-10Mo is a metallic system
				with high thermal conductivity,
				the energy in the electronic subsystem dissipates quickly.
		\end{itemize}
	\column{0.5\textwidth}
		\begin{figure}[ht]
			\centering
			\begin{subfigure}{0.49\textwidth}
				\centering
				\caption{}
				\includegraphics[width=\textwidth]{resol2/images/ttm1.png}
			\end{subfigure}
			\begin{subfigure}{0.49\textwidth}
				\centering
				\caption{}
				\includegraphics[width=\textwidth]{resol2/images/ttm2.png}
			\end{subfigure}
			\begin{subfigure}{0.49\textwidth}
				\centering
				\caption{}
				\includegraphics[width=\textwidth]{resol2/images/ttm3.png}
			\end{subfigure}
			\begin{subfigure}{0.49\textwidth}
				\centering
				\caption{}
				\includegraphics[width=\textwidth]{resol2/images/ttm4.png}
			\end{subfigure}
			\caption{
				Snapshots of a thermal spike ($30$ keV/nm) simulation at
				(a) $0$, (b) $2.4$, (c) $10.1$, and (d) $29.1$ ps.
				U, Mo and Xe atoms are shown in red, blue and black.
			}
		\end{figure}
	\end{columns}
\end{frame}

\begin{frame}{Model for homogeneous re-solution}
	\begin{columns}
	\column{0.5\textwidth}
		\begin{itemize}
			\item For homogeneous re-solution,
				FFs originating at different distances from a bubble
				need to be accounted for.
			\item A particular rotation of fission events around the bubble
				makes the calculation tenable.
		\end{itemize}
		\begin{align}
			b &= \sum_{k = Y, I} \int_V \xi_k(x, w) \dot{F} dV \\
			&= \dot{F} \sum_{k = Y, I} \int_{x=0}^{\infty} \int_{w=0}^{\infty}
				\xi_k(x, w) 2 \pi w dw dx
		\end{align}
	\column{0.5\textwidth}
		\begin{figure}[ht]
			\centering
			\begin{subfigure}{\textwidth}
				\centering
				\includegraphics[width=0.8\textwidth]{resol2/images/rotation.pdf}
			\end{subfigure}

			\begin{subfigure}{\textwidth}
				\centering
				\includegraphics[width=0.5\textwidth]{resol2/images/coord.pdf}
			\end{subfigure}
			\caption{
				(Top) Rotation of fission events around the origin.
				(Bottom) A coordinate system with a bubble at the origin
				and all FFs pointing to the $-x$ direction.
			}
		\end{figure}
	\end{columns}
\end{frame}

\begin{frame}{Reference fission fragment simulations}
	\begin{columns}
	\column{0.5\textwidth}
		\begin{itemize}
			\item The brute-force approach is computationally prohibitive.
			\item If the probability of a FF going through a certain point
				with a certain energy is known,
				only local re-solution behavior needs to be simulated.
			\item RustBCA was used to perform BCA simulations of ions in U-10Mo.
			\item A discretization scheme is used to gather FF information.
		\end{itemize}
	\column{0.5\textwidth}
		\begin{figure}[ht]
			\centering
			\begin{subfigure}{\textwidth}
				\centering
				\includegraphics[width=0.6\textwidth]{resol2/images/ff_track.png}
			\end{subfigure}

			\begin{subfigure}{\textwidth}
				\centering
				\includegraphics[width=0.5\textwidth]{resol2/images/surf_grid.pdf}
			\end{subfigure}
			\caption{
				(Top) Trajectories of $100$ \Y ions in U-10Mo.
				(Bottom) Surface discretization
				of fission fragment information across volume.
			}
		\end{figure}
	\end{columns}
\end{frame}

\begin{frame}{Ion profiles}
	\begin{figure}[ht]
		\centering
		\begin{subfigure}{0.32\textwidth}
			\centering
			\includegraphics
				[width=\textwidth, trim={0.8cm 0 1.5cm 0.4cm}, clip]
				{resol2/images/Y_p.pdf}
		\end{subfigure}
		\begin{subfigure}{0.32\textwidth}
			\centering
			\includegraphics
				[width=\textwidth, trim={0.8cm 0 1.5cm 0.4cm}, clip]
				{resol2/images/Y_e.pdf}
		\end{subfigure}
		\begin{subfigure}{0.32\textwidth}
			\centering
			\includegraphics
				[width=\textwidth, trim={0.8cm 0 1.5cm 0.4cm}, clip]
				{resol2/images/Y_a.pdf}
		\end{subfigure}
		\begin{subfigure}{0.32\textwidth}
			\centering
			\includegraphics
				[width=\textwidth, trim={0.8cm 0 1.5cm 0.4cm}, clip]
				{resol2/images/I_p.pdf}
		\end{subfigure}
		\begin{subfigure}{0.32\textwidth}
			\centering
			\includegraphics
				[width=\textwidth, trim={0.8cm 0 1.5cm 0.4cm}, clip]
				{resol2/images/I_e.pdf}
		\end{subfigure}
		\begin{subfigure}{0.32\textwidth}
			\centering
			\includegraphics
				[width=\textwidth, trim={0.8cm 0 1.5cm 0.4cm}, clip]
				{resol2/images/I_a.pdf}
		\end{subfigure}
		\caption{
			(Top, left to right) \Y ion
			incidence probability per unit surface area, energy, and angle.
			(Bottom, left to right) \I ion
			incidence probability per unit surface area, energy, and angle.
		}
	\end{figure}
\end{frame}

\begin{frame}{Fission fragment and bubble interactions}
	\begin{columns}
	\column{0.5\textwidth}
		\begin{itemize}
			\item A specific 3D geometry has been implemented in RustBCA
				to allow the simulation of gas bubbles in solid materials.
			\item A constant mean-free-path was used for solid U-10Mo,
				and an exponentially distributed mean-free-path
				was used for the Xe gas bubble.
			\item Xe recoils were analyzed
				to count the number of re-solved Xe atoms.
		\end{itemize}
	\column{0.5\textwidth}
		\begin{figure}[ht]
			\centering
			\begin{subfigure}{\textwidth}
				\centering
				\includegraphics[width=0.6\textwidth]{resol2/images/ff_bubble.png}
			\end{subfigure}

			\begin{subfigure}{0.49\textwidth}
				\centering
				\includegraphics[width=\textwidth]{resol2/images/xe_dr.pdf}
			\end{subfigure}
			\begin{subfigure}{0.49\textwidth}
				\centering
				\includegraphics[width=\textwidth]{resol2/images/xe_hist.pdf}
			\end{subfigure}
			\caption{
				(Top) BCA simulation of a $64$ nm radius bubble
				with a $5$ MeV \Y ion.
				(Bottom) Xe recoil distribution.
			}
		\end{figure}
	\end{columns}
\end{frame}

\begin{frame}{Local re-solution behavior}
	\begin{columns}
	\column{0.5\textwidth}
		\begin{itemize}
			\item In BCA simulations, ion energy ($E$)
				and off-centered distance ($\ell$) were varied.
			\item $5,000$ simulations were performed for each configuration
				with the \texttt{SPHEREINCUBOID} geometry in RustBCA.
			\item The results are denoted by $\chi(E, \ell)$.
			\item Interpolation can be used to obtain local FF-bubble behavior
				for any energy and off-centered distance.
		\end{itemize}
	\column{0.5\textwidth}
		\begin{figure}[ht]
			\centering
			\begin{subfigure}{0.49\textwidth}
				\centering
				\includegraphics[width=\textwidth]{resol2/images/chi_2nm_Y.pdf}
			\end{subfigure}
			\begin{subfigure}{0.49\textwidth}
				\centering
				\includegraphics[width=\textwidth]{resol2/images/chi_2nm_I.pdf}
			\end{subfigure}
			\begin{subfigure}{0.49\textwidth}
				\centering
				\includegraphics[width=\textwidth]{resol2/images/chi_64nm_Y.pdf}
			\end{subfigure}
			\begin{subfigure}{0.49\textwidth}
				\centering
				\includegraphics[width=\textwidth]{resol2/images/chi_64nm_I.pdf}
			\end{subfigure}
			\caption{
				Re-solved bubble fraction
				as a function of energy and off-centered distance.
				Error bars show $2\sigma$ deviations.
			}
		\end{figure}
	\end{columns}
\end{frame}

\begin{frame}{Re-solution due to any fission fragment}
	\begin{columns}
	\column{0.5\textwidth}
		\begin{itemize}
			\item With the ion profiles and local re-solution behavior,
				it is possible to calculate the re-solved bubble fraction
				due to a FF originating at an arbitrary $(x, w)$.
			\item Probability, energy and angle are combined with
				a probabilistic interpretation of a FF-bubble interaction.
		\end{itemize}
		{\scriptsize
			\begin{align}
				\xi(x, w) &= \sum_{m \in S}
					p(r_m) \frac{A_m}{\cos \alpha(x, w)}
					\chi(E(r_m), ||r_m - r_c||)
			\end{align}
		}
	\column{0.5\textwidth}
		\begin{figure}[ht]
			\centering
			\includegraphics[width=\textwidth]{resol2/images/surf_mesh.pdf}
			\caption{
				Illustration of an arbitrary
				fission fragment and bubble interaction.
			}
		\end{figure}
	\end{columns}
\end{frame}

\begin{frame}{Re-solution profiles}
	\begin{columns}
	\column{0.5\textwidth}
		\begin{itemize}
			\item Re-solution of a bubble at $(x, w)$ due to a FF at the origin
				can be calculated for all $(x, w)$.
			\item If the positions of the bubble and the FF are swapped,
				we would get the same $\xi$ profiles.
			\item The re-solution rate can then be calculated
				using a discretized version of the model we described before.
		\end{itemize}
		{\scriptsize
			\begin{align}
				b / \dot{F}
					&= \sum_{k=Y,I} \sum \xi_k \Delta V
			\end{align}
		}
	\column{0.5\textwidth}
		\begin{figure}
			\centering
			\begin{subfigure}{0.9\textwidth}
				\centering
				\includegraphics
					[width=\textwidth, trim={0.8cm 0 1.4cm 0.7cm}, clip]
					{resol2/images/64nm_Y_xi.pdf}
			\end{subfigure}
			\begin{subfigure}{0.9\textwidth}
				\centering
				\includegraphics
					[width=\textwidth, trim={0.8cm 0 1.4cm 0.7cm}, clip]
					{resol2/images/64nm_Y_db.pdf}
			\end{subfigure}
			\caption{
				(Top) $\xi$ and (Bottom) $\xi \Delta V$
				for \Y incident on a bubble of radius $64$ nm.
			}
		\end{figure}
	\end{columns}
\end{frame}

\begin{frame}{Homogeneous re-solution rate}
	\begin{columns}
	\column{0.5\textwidth}
		\begin{itemize}
			\item Summing up $\xi \Delta V$ for both \Y and \I
				provide the overall re-solution rate.
			\item The probability of an interaction between a FF
				and a larger bubble is higher.
		\end{itemize}
		\begin{align}
			b / \dot{F} &= a R_b^k + c \\
			a &= \num{8.43e-25} \\
			k &= -0.926 \\
			c &= \num{3.46e-26}
		\end{align}
	\column{0.5\textwidth}
		\begin{figure}
			\centering
			\includegraphics[width=\textwidth]{resol2/images/bhom.pdf}
			\caption{
				Homogeneous re-solution rate at equilibrium Xe number density.
			}
		\end{figure}
	\end{columns}
\end{frame}

\begin{frame}{Effect of bubble pressure}
	\begin{columns}
	\column{0.5\textwidth}
		\begin{itemize}
			\item The number of re-solved Xe atoms is almost an invariant
				with respect to the Xe number density in the bubble.
			\item $\chi$ and $n$ are thus inversely proportional:
				$\chi/\chi_{eq} = n_{eq}/n$.
			\item Pressure effects can be incorporated as follows.
		\end{itemize}
		\begin{align}
			b &\propto \xi \propto \chi \\
			b / b_{eq} &= n_{eq} / n \\
			b &= \left( a R_b^k + c \right)
				\left( \frac{n_{eq}}{n} \right) \dot{F}
		\end{align}
	\column{0.5\textwidth}
		\begin{figure}
			\centering
			\includegraphics[width=\textwidth]{resol2/images/pressure.pdf}
			\caption{
				Effect of Xe number density on homogeneous re-solution rate.
			}
		\end{figure}
	\end{columns}
\end{frame}

\begin{frame}{Comparison with other fuels}
	\begin{columns}
	\column{0.5\textwidth}
		\begin{itemize}
			\item The re-solution in UO$_2$ is dominated
				by the heterogeneous mechanism
				and the re-solution in UC happens entirely through
				the homogeneous mechanism.
			\item The re-solution rates in UO$_2$, UC and U-10Mo
				are within one order of magnitude.
			\item The re-solution rate used in DART is higher for smaller bubbles
				and much lower for larger bubbles than the rate we calculated.
		\end{itemize}
	\column{0.5\textwidth}
		\begin{figure}
			\centering
			\includegraphics[width=\textwidth]{resol2/images/comp.pdf}
			\caption{
				Comparison of the re-solution rates in different nuclear fuels.
			}
		\end{figure}
	\end{columns}
\end{frame}

\begin{frame}{Conclusions}
	\begin{itemize}
		\item Both the homogeneous and heterogeneous re-solution mechanisms
			were investigated.
		\item Thermal spikes initiated by electronic stopping
			was found to be ineffective for re-solving Xe atoms.
			Thus, heterogeneous re-solution does not occur in U-10Mo.
		\item Homogeneous re-solution,
			which is brought about by nuclear stopping,
			was found to be the only mechanism of re-solution in U-10Mo.
		\item The re-solution rate was calculated
			by first profiling FF behavior in the fuel,
			then evaluating the FF interactions with Xe gas bubbles,
			and finally putting all the information together in a physical model.
		\item The computed re-solution rate $b$ is
			$\num{4.4e-26} \dot{F}$ s$^{-1}$ for the largest intergranular bubble
			and $\num{8.8e-25} \dot{F}$ s$^{-1}$ for the smallest intragranular bubble,
			where the unit of $\dot{F}$ is fission/m$^3$/s.
		\item The re-solution rate is inversely proportional to the Xe number density.
	\end{itemize}
\end{frame}

\section{IUQ of U-10Mo fission-gas-behavior parameters}

\begin{frame}{Inverse uncertainty quantification (IUQ)}
	\begin{columns}
	\column{0.5\textwidth}
		\begin{itemize}
			\item IUQ is a process to quantify uncertainties
				in the input parameters of a computer model
				given experimental data.
			\item It's the opposite of
				forward uncertainty quantification (FUQ),
				which quantifies the uncertainty in the output given the input.
			\item IUQ quantifies the uncertainty in the model itself,
				and thus can help to improve
				the accuracy of the model predictions.
			\item It's helpful in situations where the underlying parameters
				of the model are unknown.
		\end{itemize}
	\column{0.5\textwidth}
		\begin{figure}[ht]
			\centering
			\includegraphics[width=6cm]{sfigs/wu_iuq.png}
			\caption{
				Some essential parts of modeling and simulation
				(Wu et al. {\color{blue}
				\url{https://doi.org/10.1016/j.nucengdes.2018.06.004}}).
			}
		\end{figure}
	\end{columns}
\end{frame}

\begin{frame}{Experimental observation}
	\begin{columns}
	\column{0.5\textwidth}
		\begin{itemize}
			\item \textit{RERTR5-V6018G} had a burnup of $36 \%$
				after $116$ effective full power days (EFPD).
				The fission density was $F_d = \num{2.31e21}$ fsn/cm$^3$ (low $F_d$),
				with a fission rate density of $\num{2.3e14}$ fsn/cm$^3$/s
				and a fuel temperature of $116.3^{\circ}$ C.
			\item \textit{RERTR12-L1P755} had
				$F_d = \num{5.34e21}$ fsn/cm$^3$ (high $F_d$),
				with a life-averaged fission rate and temperature of
				$\num{6.86e14}$ fsn/cm$^3$/s and $133.6^{\circ}$ C.
			\item DART simulations were performed to simulate
				these operational conditions.
		\end{itemize}
	\column{0.5\textwidth}
		\begin{figure}
			\centering
			\includegraphics[width=0.7\textwidth]{iuq2/images/mvn_expt.pdf}
			\caption{
				Experimentally observed fuel swelling
				of \textit{RERTR5-V6018G} at $F_d = \num{2.32e21}$ fsn/cm$^3$ (low $F_d$)
				and \textit{RERTR12-L1P755} at $F_d = \num{5.34e21}$ fsn/cm$^3$ (high $F_d$)
				with a measurement uncertainty of $2.64 \%$.
				(Kim et al.
				{\color{blue}\url{https://doi.org/10.1016/j.jnucmat.2013.01.291}}.
				Rice et al.
				{\color{blue}\url{https://doi.org/10.2172/1173078}}.
				Robinson et al.
				{\color{blue}\url{https://doi.org/10.1016/j.jnucmat.2020.152703}}.)
			}
		\end{figure}
	\end{columns}
\end{frame}

\begin{frame}{Fission-gas-behavior parameters}
	\begin{table}[ht]
	\centering
	\caption{
		Fission-gas-behavior parameters and their ranges.
	}
	\begin{tabular}{llccc}
	\toprule
	Parameter             & Unit
		& Minimum value & Maximum value & Reference value \\
	\midrule
	\texttt{dGrainHBS}    & cm
		& \num{1e-5 }   & \num{1e-4 }   & \num{4e-5 }   \\
	\texttt{FaceCovMax}   & N/A
		& \num{0.5  }   & \num{0.907}   & \num{0.907}   \\
	\texttt{SwellLink}    & N/A
		& \num{0.01 }   & \num{0.08 }   & \num{0.025}   \\
	\texttt{rResolBulk}   & cm
		& \num{6e-10}   & \num{5e-9 }   & \num{3e-9 }   \\
	\texttt{DatomFissGBx} & N/A
		& \num{6e3  }   & \num{5e4  }   & \num{3e4  }   \\
	\texttt{StickProb}    & N/A
		& \num{2e-10}   & \num{4e-7 }   & \num{2e-7 }   \\
	\texttt{fNucleate}    & N/A
		& \num{3e-10}   & \num{3e-9 }   & \num{6e-10}   \\
	\texttt{vResol}       & cm$^3$
		& \num{1e-18}   & \num{4e-18}   & \num{2e-18}   \\
	\texttt{rResolGBB}    & cm
		& \num{2e-7 }   & \num{1e-6 }   & \num{5e-7 }   \\
	\bottomrule
	\end{tabular}
	\end{table}

	\begin{itemize}
		\item 3200 different combinations of the parameters
			were used in DART to simulate swelling
			for the operational conditions of the experiments.
			We performed IUQ with the resulting data sets.
	\end{itemize}
\end{frame}

\begin{frame}{Swelling at low $F_d$}
	\label{frame:scatter_lo}
	\begin{columns}
	\column{0.5\textwidth}
		\begin{itemize}
			\item At $F_d = \num{2.31e21}$ fsn/cm$^3$,
				the swelling values range from $9 \%$ to $22 \%$.
			\item Only the parameter \texttt{rResolBulk}
				seems to have a strong correlation with fuel swelling.
			\item \texttt{vResol} has a weak negative correlation.
		\end{itemize}
	\column{0.5\textwidth}
		\begin{figure}
			\centering
			\includegraphics[width=\textwidth]{iuq2/images/scatter_lo.pdf}
			\caption{
				Scatter plots of fuel swelling
				against the fission-gas-behavior parameters
				at $F_d = \num{2.31e21}$ fsn/cm$^3$.
			}
		\end{figure}
	\end{columns}
\end{frame}

\begin{frame}{Swelling at high $F_d$}
	\begin{columns}
	\column{0.5\textwidth}
		\begin{itemize}
			\item At $F_d = \num{5.34e21}$ fsn/cm$^3$,
				the swelling values range from $20 \%$ to $140 \%$.
			\item \texttt{dGrainHBS} shows a positive correlation with swelling.
			\item \texttt{SwellLink} and \texttt{rResolBulk}
				show a weak correlations.
		\end{itemize}
	\column{0.5\textwidth}
		\begin{figure}
			\centering
			\includegraphics[width=\textwidth]{iuq2/images/scatter_hi.pdf}
			\caption{
				Scatter plots of fuel swelling
				against the fission-gas-behavior parameters
				at $F_d = \num{5.34e21}$ fsn/cm$^3$.
			}
		\end{figure}
	\end{columns}
\end{frame}

\begin{frame}{Surrogate modeling}
	\begin{columns}
	\column{0.5\textwidth}
		\begin{itemize}
			\item It's not feasible to run calculations
				for all possible parameter combinations
				using DART.
			\item A surrogate model
				(also called a response surface or an emulator)
				is thus needed.
			\item Many machine learning models,
				such as Gaussian processes and neural networks,
				can act as a surrogate model.
		\end{itemize}
	\column{0.5\textwidth}
		\begin{figure}[ht]
			\centering
			\includegraphics[width=\textwidth]{sfigs/surrogate_modeling.png}
			\caption{
				The principle of surrogate modeling
				(Kocijan et al.
				{\color{blue}\url{https://doi.org/10.5516/NET.07.2014.706}}).
			}
		\end{figure}
	\end{columns}
\end{frame}

\begin{frame}{DART surrogates}
	\begin{columns}
	\column{0.5\textwidth}
		\begin{itemize}
			\item A GP with a Mat\'ern kernel was found to perform the best
				with $R^2$ scores of $0.99$ and $0.97$,
				and mean absolute errors of $0.19 \%$ and $1.94 \%$,
				for emulating low and high $F_d$, respectively.
			\item The NN model has $4$ hidden layers with $250$ neurons each.
				It has $R^2$ scores of $0.98$ and $0.98$,
				and mean absolute errors of $0.27 \%$ and $1.54 \%$
				for low $F_d$ and high $F_d$.
			\item We continue with the GP surrogates since they provide
				uncertainty estimates along with predictions.
		\end{itemize}
	\column{0.5\textwidth}
		\begin{figure}[ht]
			\centering
			\begin{subfigure}{0.49\textwidth}
				\centering
				\includegraphics[width=\textwidth]{iuq2/images/gp_lo.pdf}
			\end{subfigure}
			\begin{subfigure}{0.49\textwidth}
				\centering
				\includegraphics[width=\textwidth]{iuq2/images/gp_hi.pdf}
			\end{subfigure}
			\begin{subfigure}{0.49\textwidth}
				\centering
				\includegraphics[width=\textwidth]{iuq2/images/nn_lo.pdf}
			\end{subfigure}
			\begin{subfigure}{0.49\textwidth}
				\centering
				\includegraphics[width=\textwidth]{iuq2/images/nn_hi.pdf}
			\end{subfigure}
			\caption{
				Surrogate predictions vs actual test data for
				Gaussian process (GP) and neural network (NN) surrogate models.
				\textit{RERTR5-V6018G} and \textit{RERTR12-L1P755} represent
				low and high fission density data sets, respectively.
			}
		\end{figure}
	\end{columns}
\end{frame}

\begin{frame}{Global sensitivity analysis}
	\begin{columns}
	\column{0.5\textwidth}
		\begin{itemize}
			\item At low $F_d$, \texttt{rResolBulk} and \texttt{vResol}
				are the parameters accounting for most of the variability.
			\item At high $F_d$, \texttt{dGrainHBS}
				is the most influential parameter,
				with some minor influences from
				\texttt{rResolBulk}, \texttt{SwellLink} and \texttt{FaceCovMax}.
			\item The effects of the parameters are additive since
				the first-order and total-effect indices are almost identical.
		\end{itemize}
	\column{0.5\textwidth}
		\begin{figure}[ht]
			\centering
			\begin{subfigure}{\textwidth}
				\centering
				\includegraphics[width=0.8\textwidth]{iuq2/images/sobol_lo.pdf}
			\end{subfigure}
			\begin{subfigure}{\textwidth}
				\centering
				\includegraphics[width=0.8\textwidth]{iuq2/images/sobol_hi.pdf}
			\end{subfigure}
			\caption{
				First-order ($S_i$) and total-effect ($S_{T_i}$) Sobol' indices
				calculated using the neural network surrogate models.
			}
		\end{figure}
	\end{columns}
\end{frame}

\begin{frame}{Markov chain Monte Carlo sampling}
	\begin{columns}
	\column{0.5\textwidth}
		\begin{itemize}
			\item Markov chain Monte Carlo (MCMC) sampling was performed
				with the GP surrogate models.
			\item Two chains of $200,000$ samples are overlaid on the trace plots.
				The cumulative averages of the blue and orange traces
				are shown in black and red, respectively.
			\item Except for \texttt{dGrainHBS},
				the MCMC samples span the entire range
				defined by the parameter priors.
		\end{itemize}
	\column{0.5\textwidth}
		\begin{figure}[ht]
			\centering
			\includegraphics[width=\textwidth]{iuq2/images/mcmc_trace.png}
			\caption{
				Trace plots of the posterior samples
				of the fission-gas-behavior parameters from MCMC sampling.
			}
		\end{figure}
	\end{columns}
\end{frame}

\begin{frame}{Parameter posteriors}
	\begin{columns}
	\column{0.5\textwidth}
		\begin{itemize}
			\item \texttt{dGrainHBS} has a defined posterior
				resembling a normal distribution.
			\item \texttt{FaceCovMax}, \texttt{SwellLink}, \texttt{rResolBulk},
				and \texttt{vResol} have left skewed marginal posteriors.
			\item All the other parameters have
				almost uniform posterior distirbutions.
			\item The posterior of \texttt{dGrainHBS}
				shows positive correlations with that of
				\texttt{FaceCovMax}, \texttt{SwellLink}, and \texttt{rResolBulk}.
		\end{itemize}
	\column{0.5\textwidth}
		\begin{figure}[ht]
			\centering
			\includegraphics[width=\textwidth]{iuq2/images/iuq.png}
			\caption{
				Posterior distributions of $9$ fission-gas-behavior parameters.
			}
		\end{figure}
	\end{columns}
\end{frame}

\begin{frame}{Forward uncertainty quantification (FUQ)}
	\begin{columns}
	\column{0.5\textwidth}
		\begin{itemize}
			\item Unlike the observed swelling distribution,
				this distribution is asymmetric.
			\item The swelling density at low $F_d$
				decreases steeply around $8 \%$ (Slide \ref{frame:scatter_lo}).
			\item The slight mismatch at high $F_d$ is due to
				the consideration of $\Sigma_{code}$ from the GP surrogate models.
		\end{itemize}
	\column{0.5\textwidth}
		\begin{figure}
			\begin{subfigure}{\textwidth}
				\centering
				\includegraphics[width=0.7\textwidth]{iuq2/images/fuq_mvn.pdf}
			\end{subfigure}
			\begin{subfigure}{0.49\textwidth}
				\centering
				\includegraphics[width=\textwidth]{iuq2/images/fuq_lo.pdf}
			\end{subfigure}
			\begin{subfigure}{0.49\textwidth}
				\centering
				\includegraphics[width=\textwidth]{iuq2/images/fuq_hi.pdf}
			\end{subfigure}
			\caption{
				(Top) Joint density plot of swelling predictions calculated by
				the forward propagation of the thinned MCMC samples.
				(b) $F_d = \num{2.31e21}$ fsn/cm$^3$,
				and (c) $F_d = \num{5.34e21}$ fsn/cm$^3$.
			}
		\end{figure}
	\end{columns}
\end{frame}

\begin{frame}{Predictions from FUQ}
	\begin{columns}
	\column{0.5\textwidth}
		\begin{itemize}
			\item At low $F_d$,
				the mean of the values from the forward propagation
				matches the observation,
				but the standard deviation is narrower than that of the observation.
			\item At high $F_d$,
				the range of swelling values is significantly reduced
				from the DART predictions
				to the predictions from the forward propagation,
				matching the observation almost exactly.
		\end{itemize}
	\column{0.5\textwidth}
		\begin{figure}[ht]
			\centering
			\includegraphics[width=\textwidth]{iuq2/images/aftermath.pdf}
			\caption{
				Comparison of experimental observations with
				swelling predictions from the DART data set
				and swelling predictions from the forward propagation
				of the surrogate models using the MCMC samples.
			}
		\end{figure}
	\end{columns}
\end{frame}

\begin{frame}{Conclusions}
	\begin{itemize}
		\item IUQ was performed based on two data sets
			generated from $6,400$ DART simulations,
			one for simulating the low $F_d$ conditions of \textit{RERTR5-V6018G}
			and another for simulating the high $F_d$ conditions
			of \textit{RERTR12-L1P755}.
		\item GP and NN surrogates were built to emulate DART predictions.
		\item Global sensitivity analysis was performed with the GP surrogates.
			The parameter \texttt{rResolBulk} is
			the most influential parameter at low fission density,
			whereas \texttt{dGrainHBS} is the major driver of variance
			in swelling at high fission density.
		\item MCMC sampling was performed with the GP surrogates,
			leading to converged posteriors.
			\texttt{dGrainHBS} has a well-defined posterior,
			with a mean value of $\num{3.33e-5}$ cm.
			The other parameters span the whole range,
			indicating the prior distributions may have been too narrow.
		\item Forward propagation showed
			the swelling predictions with the parameter posteriors
			match the experimental observation better.
	\end{itemize}
\end{frame}

\section{Conclusions}

\begin{frame}{Contributions}
	Overall, this dissertation achieves the following:
	\begin{itemize}
		\item Fully quantifies the GB diffusion coefficients
			of U, Mo, and Xe in $\gamma$U-Mo
			for various compositions, temperatures, and GB orientations.
			The findings also provide insight into the anisotropy of GB diffusion.
		\item Provides a mathematical model for the re-solution rate of Xe gas bubbles,
			taking into account both bubble size and pressure.
			The model also considers
			both the heterogeneous and the homogeneous mechanisms of re-solution.
			This is the most robust and thorough study
			of re-solution of any nuclear fuel up to date.
		\item Performs IUQ on still unknown fission-gas-behavior parameters of U-Mo
			as implemented in DART.
			A few significant parameters are identified and subsequently optimized
			for better meso-scale simulation of the U-Mo fuel.
	\end{itemize}
\end{frame}

\begin{frame}{Future work}
	\begin{itemize}
		\item \textit{Assessment of blistering mechanism in U-Mo:}
			The blisters are observed as a raised region on the fuel-plate surface,
			resulting from the plastic deformation of the aluminium cladding
			due to large fission gas pore pressure in the fuel meat.
			Verifying the possible causes of fuel blistering
			can provide the fuel developers with a better understanding
			of the fuel behavior under extreme conditions.
		\item \textit{Incorporation of cladding elements in the U-Mo potential:}
			Fuel-cladding interactions need to be understood well for fuel evaluation.
			An accurate MLIP of U, Mo, and Al/Zr would be extremely helpful
			for atomistic simulations of the U-Mo fuel with liners and/or claddings.
		\item \textit{IUQ of fission-gas-behavior parameters:}
			The IUQ work should be extended
			to incorporate quantities of interest other than fuel swelling.
			This may help quantify some parameters that seemed noninfluential in our study.
			Also, more DART simulations of swelling at different operational conditions
			can be performed to enable the evaluation of model bias.
	\end{itemize}
\end{frame}

\section*{Publications}

\begin{frame}{Publications}
	The presented content is being/has been published
	in the following journal articles and technical reports:
	{\footnotesize
		\begin{itemize}
			\item \textbf{ATM Jahid Hasan}, Zhi-Gang Mei, Gyuchul Park,
				Bei Ye and Benjamin Beeler.
				Inverse uncertainty quantification of U-10Mo
				fission-gas-behavior parameters.
				In preparation.
			\item \textbf{ATM Jahid Hasan}, Linu Malakkal,
				Mathew Swisher and Benjamin Beeler.
				Xe gas bubble re-solution in U-10Mo nuclear fuel.
				In preparation.
			\item \textbf{ATM Jahid Hasan} and Benjamin Beeler.
				Calculation of grain boundary diffusion coefficients
				in $\gamma$U-Mo using atomistic simulations.
				Journal of Nuclear Materials, page 155190, 2024.
			\item Benjamin Beeler, Larry Aagesen, James Cole,
				\textbf{ATM Jahid Hasan}, et al.
				Microstructural-level fuel performance modeling
				of U-Mo monolithic fuel.
				Technical Report INL/RPT-23-74528, Idaho National Lab. (INL),
				Idaho Falls, ID (United States), 2023.
			\item Benjamin Beeler, Larry Aagesen, James Cole,
				\textbf{ATM Jahid Hasan}, et al.
				Microstructural-level fuel performance modeling
				of U-Mo monolithic fuel.
				Technical report, Idaho National Lab. (INL),
				Idaho Falls, ID (United States), 2022.
			\item Benjamin W Beeler, James I Cole, Sourabh Bhagwan Kadambi,
				Linu Malakkal, Larry K Aagesen Jr, Gerard Hofman,
				\textbf{ATM Jahid Hasan}, et al.
				Microstructural-level fuel performance modeling
				of U-Mo monolithic fuel.
				Technical report, Idaho National Lab. (INL),
				Idaho Falls, ID (United States), 2021.
		\end{itemize}
	}
\end{frame}

\section*{Thanks}

\begin{frame}
	\centering \Large
	\emph{Thank you!}
\end{frame}

\end{document}
