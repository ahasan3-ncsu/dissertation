\documentclass[10pt]{beamer}
\usepackage{booktabs, subcaption, multirow}
\usepackage{amsmath, amssymb, bm, siunitx}

\usetheme{CambridgeUS}
\usecolortheme{beaver}
\setbeamertemplate{caption}[numbered]
\setbeamerfont{caption}{size=\scriptsize}

\AtBeginSection[]
{
  \begin{frame}
    \frametitle{Table of Contents}
    \tableofcontents[currentsection]
  \end{frame}
}

\title[Evaluation of U-Mo Microstructural Properties]
{Evaluation of Microstructural Properties
of U-Mo Monolithic Fuel using Computational Methods}

\author{ATM Jahid Hasan}
\institute[NCSU]{Department of Nuclear Engineering\\
North Carolina State University}
\date{January 28, 2026}

\begin{document}
\frame{\titlepage}

\begin{frame}{Table of Contents}
	\tableofcontents
\end{frame}

\section{Introduction}

\begin{frame}{Introduction}
	\begin{columns}
	\column{0.5\textwidth}
		\begin{itemize}
			\item Research reactors operate at low temperature and pressure
				but at high fission density and specific power.
				Thus, RRs historically needed highly enriched uranium (HEU).
			\item With the inception of RERTR in 1978,
				there has been an ongoing effort
				to change HEU into low enriched uranium (LEU).
			\item U$_3$Si$_2$ is the only qualified LEU for RRs.
			\item High-performance RRs need even higher uranium density.
		\end{itemize}
	\column{0.5\textwidth}
		\begin{figure}[ht]
			\centering
			\includegraphics[width=6cm]{sfigs/historic_fuels.png}
			\caption{
				U-density of research reactor fuel vs time of first use.
				(Jamison et al.
				{\color{blue}\url{https://doi.org/10.2172/1890226}}).
			}
		\end{figure}
	\end{columns}
\end{frame}

\begin{frame}{Introduction (cont.)}
	\begin{columns}
	\column{0.5\textwidth}
		\begin{itemize}
			\item US High-Performance Research Reactor (USHPRR) program
				selected $\gamma$U-10Mo fuel for qualification.
			\item $\gamma$U-10Mo monolithic fuel design involves
				a solid $\gamma$U-10Mo fuel meat monded to Al-6061 cladding
				with a zirconium diffusion barrier.
			\item The U-Mo dispersion fuel design has also been tested.
				The dispersion fuel shows breakaway swelling,
				associated with the formation of Mo-stabilized high-Al
				intermetallic phase.
		\end{itemize}
	\column{0.5\textwidth}
		\begin{figure}[ht]
			\centering
			\includegraphics[width=5cm]{sfigs/umo_design.jpg}
			\caption{
				Depiction of monolithic fuel cross-section
				(Meyer et al.
				{\color{blue}\url{https://doi.org/10.5516/NET.07.2014.706}}).
			}
		\end{figure}
	\end{columns}
\end{frame}

\begin{frame}{Swelling in $\gamma$U-Mo}
	\begin{columns}
	\column{0.5\textwidth}
		\begin{itemize}
			\item Excessive swelling during operation is a major concern
				for $\gamma$U-Mo fuels.
			\item With increasing bunrup,
				the bubble population increases in the grain boundaries
				and newer grain boundaries form.
			\item This grain refinement is accompanied by an increase
				in bubble size and number,
				which increases ffuel swelling rate.
			\item To account for it in the fuel design,
				fuel swelling needs to be predicted
				for different operational conditions.
		\end{itemize}
	\column{0.5\textwidth}
		\begin{figure}[ht]
			\centering
			\includegraphics[width=4cm]{sfigs/umo_burnup.png}
			\caption{
				$\gamma$U-Mo microstructure as a function of burnup
				(Kim et al. {\color{blue}
				\url{https://doi.org/10.1016/j.jnucmat.2011.08.018}}).
			}
		\end{figure}
	\end{columns}
\end{frame}

\begin{frame}{Dispersion Analysis Research Tool (DART)}
	\begin{columns}
	\column{0.5\textwidth}
		\begin{itemize}
			\item DART is a mechanistic rate-theory-based meso-scale model
				for the calculation of fuel swelling.
			\item The code uses material properties,
				such as gas atom diffusivity, recrystallization kinetics,
				and gas re-solution rate.
			\item Many of material properties used by DART are still unknown.
			\item These properties can be gleaned
				from lower-length-scale studies
				or through parameter optimization using experimental data.
		\end{itemize}
	\column{0.5\textwidth}
		\begin{figure}[ht]
			\centering
			\includegraphics[width=6cm]{sfigs/dart_schematic.png}
			\caption{
				DART schematic.
				(Ye et al. {\color{blue}
				\url{https://doi.org/10.1016/j.jnucmat.2023.154542}}).
			}
		\end{figure}
	\end{columns}
\end{frame}

\section{Grain boundary diffusion in \texorpdfstring{$\gamma$}{gamma}U-Mo}

\begin{frame}{Grain boundary (GB) diffusion in $\gamma$U-Mo}
	\begin{itemize}
		\item Accurate calculation of fuel swelling requires
			diffusion coefficients of the related species in the fuel.
		\item  Furthermore, creep modeling also requires diffusion coefficients
			to determine creep rates and evaluate the evolving microstructure.
		\item Therefore, it is essential to understand
			the diffusion behavior of the $\gamma$U-Mo fuel.
		\item The diffusion coefficients of the relevant species
			in $\gamma$U-Mo grain boundaries (GBs) are yet unknown.
	\end{itemize}
\end{frame}

\begin{frame}{Simulation setup}
	\begin{columns}
	\column{0.5\textwidth}
		\begin{itemize}
			\item Bicrystals for simulations are created
				by rotating two halves of the system around a tilt axis.
			\item Two GBs: one in the middle and one at the edge.
			\item GB region is defined using atomic trajectories.
			\item We tracked lattice point jumps.
				GB region can be differentiated from the bulk
				by determining lattice points associated with jumps.
		\end{itemize}
	\column{0.5\textwidth}
		\begin{figure}[ht]
			\centering
			\begin{subfigure}{1.0\textwidth}
				\centering
				\includegraphics[width=\textwidth]
					{diffusion/images/configuration.png}
			\end{subfigure}

			\begin{subfigure}{0.7\textwidth}
				\centering
				\includegraphics[width=0.7\textwidth]
					{diffusion/images/gb_def.png}
			\end{subfigure}
			\caption{
				(Top) $\Sigma 5$ Symmetric tilt GB initial configuration.
				(Bottom) Definition of GB based on atomic trajectories.
			}
		\end{figure}
	\end{columns}
\end{frame}

\begin{frame}{Explored GBs}
	\begin{itemize}
		\item Three compositions: $\gamma$U-7Mo, $\gamma$U-10Mo, and $\gamma$U-12Mo
			are examined.
		\item The temperature of the simulated GBs ranged from 600 K to 1200 K
			with an interval of 100 K.
			The dimensions of the supercells are at least
			$50 \times 200 \times 50$ \AA$^3$.
			This corresponds to at least 35,000 atoms.
		\item The following symmetric tilt GBs are studied:
			\begin{itemize}
				\item (120)
				\item (130)
				\item (150)
				\item (190)
				\item (340)
				\item (350)
			\end{itemize}
		\item Other simulated systems are:
			\begin{itemize}
				\item asymmetric (110)
				\item asymmetric (130)
				\item asymmetric (190)
				\item asymmetric (350)
				\item twist (110)
				\item twist (230)
			\end{itemize}
	\end{itemize}
\end{frame}

\begin{frame}{GB validation}
	\begin{columns}
	\column{0.5\textwidth}
		\begin{itemize}
			\item For validation, we computed GB energies
				with the configurations first.
			\item The energies are within 2$\sigma$ of the values
				reported by Beeler et al.
			\item Beeler et al. employed roughly 10 times fewer atoms.
			\item The average GB energy for the examined systems
				is about 0.68 Jm$^{-2}$.
		\end{itemize}
	\column{0.5\textwidth}
		\begin{figure}[ht]
			\centering
			\includegraphics[width=6cm]{diffusion/images/gbe.pdf}
			\caption{
				GB energies as a function of misorientation angle.
			}
		\end{figure}
	\end{columns}
\end{frame}

\begin{frame}{GB width}
	\begin{columns}
	\column{0.5\textwidth}
		\begin{itemize}
			\item GB width in this work means the structural width of a GB.
			\item This is different from the diffusional width,
				where width is calculated using the number of mobile GB atoms.
			\item GB width increases linearly with temperature.
			\item The width is about 6 \r{A} at 600 K
				and increases to about 12 \r{A} at 1200 K.
			\item In literature, GB width assumptions fall between
				5 \r{A} to 15 \r{A}.
		\end{itemize}
	\column{0.5\textwidth}
		\begin{figure}[ht]
			\centering
			\includegraphics[width=6cm]{diffusion/images/d_gb_sym.pdf}
			\caption{
				Widths of symmetric tilt GBs as a function of temperature.
			}
		\end{figure}
	\end{columns}
\end{frame}

\begin{frame}{GB diffusivities}
	\begin{columns}
	\column{0.5\textwidth}
		\begin{itemize}
			\item Diffusion behavior is generally Arrhenius.
			\item The spread in diffusion coefficients due to tilt angles
				is about one order or magnitude.
			\item Diffusivity ranges from $10^{-14}$
				to $10^{-11}$ m$^2$s$^{-1}$.
			\item The activation energy is about 0.4--0.7 eV for U.
			\item Diffusion perpendicular to the tilt axis
				has similar characteristics.
		\end{itemize}
	\column{0.5\textwidth}
		\begin{figure}[ht]
			\centering
			\includegraphics[width=6cm]{diffusion/images/u10mo_U_Dz.pdf}
			\caption{
				Diffusion coefficients of U parallel to the GB tilt axis.
			}
		\end{figure}
	\end{columns}
\end{frame}

\begin{frame}{Symmetric tilt GB diffusivities}
	\begin{figure}[ht]
		\centering
		\begin{subfigure}{0.325\textwidth}
			\centering
			\includegraphics[width=\textwidth]
				{diffusion/images/u10mo_U_Dz.pdf}
		\end{subfigure}
		\begin{subfigure}{0.325\textwidth}
			\centering
			\includegraphics[width=\textwidth]
				{diffusion/images/u10mo_Mo_Dz.pdf}
		\end{subfigure}
		\begin{subfigure}{0.325\textwidth}
			\centering
			\includegraphics[width=\textwidth]
				{diffusion/images/u10mo_Xe_Dz.pdf}
		\end{subfigure}
		\caption{
			Diffusion coefficients parallel to the GB tilt axis
			for symmetric tilts GBs.
		}
	\end{figure}

	\begin{itemize}
		\item Mo diffusivity is lower than that of U.
		\item Xe diffusivity is the higher than that of other species
			at high temperatures.
		\item At low tempeartures, all three species have similar coefficients.
	\end{itemize}
\end{frame}

\begin{frame}{Asymmetric tilt \& Twist GB diffusivities}
	\begin{figure}[ht]
		\centering
		\begin{subfigure}{0.45\textwidth}
			\centering
			\includegraphics[width=\textwidth]
				{diffusion/images/asym_twist_U_Dz.pdf}
		\end{subfigure}
		\begin{subfigure}{0.45\textwidth}
			\centering
			\includegraphics[width=\textwidth]
				{diffusion/images/asym_twist_Mo_Dz.pdf}
		\end{subfigure}
		\caption{
			Diffusion coefficients parallel to the GB tilt axis
			for asymmetric tilt and twist GBs.
		}
	\end{figure}

	\begin{itemize}
		\item The diffusion behavior is similar
			for asymmetric tilt and twist GBs.
		\item U diffusivity is higher than that of Mo.
		\item Overall impact of orientation of GBs on diffusivity
			appears to be minimal.
	\end{itemize}
\end{frame}

\begin{frame}{Orientation-averaged GB diffusivities}
	\begin{figure}[ht]
		\centering
		\begin{subfigure}{0.45\textwidth}
			\centering
			\includegraphics[width=\textwidth]
				{diffusion/images/comp_U_Dz.pdf}
		\end{subfigure}
		\begin{subfigure}{0.45\textwidth}
			\centering
			\includegraphics[width=\textwidth]
				{diffusion/images/comp_Mo_Dz.pdf}
		\end{subfigure}
		\caption{
			Orientation-averaged GB diffusivities parallel to the tilt axis
			for $\gamma$U-7Mo, $\gamma$U-10Mo, and $\gamma$U-12Mo.
		}
	\end{figure}

	\begin{itemize}
		\item To compare diffusion of different compositions,
			diffusivities parallel to the tilt axis of all symmetric tilt GBs
			are averaged.
		\item GB diffusivity is negatively correlated with the Mo content.
	\end{itemize}
\end{frame}

\begin{frame}{Comparison with literature}
	\begin{columns}
	\column{0.5\textwidth}
		\begin{itemize}
			\item The difference between GB diffusivity and self-diffusivity
				grows larger with decreasing temperature.
			\item With GB diffusion enhancement factors and GB widths,
				we can calculate effective diffusion coefficient $D_{eff}$:
				$ D_{eff} = f D_{gb} + (1-f) D_l $,
				where $D_l$ is the bulk diffusion coefficient
				and $f$ is the volume fraction of GBs.
			\item Assuming 10 $\mu$m grain size,
				we find that $D_{eff} \approx 1.5 \times 10^3 D_l$ at 600 K
				and $D_{eff} \approx 1.1 D_l$ at 1200 K.
			\item Overall diffusion is dominated by GB diffusion
				at lower temperatures.
		\end{itemize}
	\column{0.5\textwidth}
		\begin{figure}[ht]
			\centering
			\includegraphics[width=0.9\textwidth]
				{diffusion/images/newLitComp.pdf}
			\caption{
				GB vs bulk diffusivities.
			}
		\end{figure}
	\end{columns}
\end{frame}

\begin{frame}{Effect of the misorientation angle}
	\begin{figure}[ht]
		\centering
		\begin{subfigure}{0.45\textwidth}
			\centering
			\includegraphics[width=\textwidth]
				{diffusion/images/DvsTilt_600K.pdf}
		\end{subfigure}
		\begin{subfigure}{0.45\textwidth}
			\centering
			\includegraphics[width=\textwidth]
				{diffusion/images/DvsTilt_1200K.pdf}
		\end{subfigure}
		\caption{
			Diffusivites against misorientation angle at 600 K and 1200 K.
		}
	\end{figure}

	\begin{itemize}
		\item U GB diffusivity is higher than that of Mo and Xe at 600 K.
			At 1200 K, U GB diffusivity is still higher than that of Mo.
			However, Xe diffusivity becomes larger than that of U.
		\item In general, the greater the misorientation angle,
			the greater the diffusivity.
	\end{itemize}
\end{frame}

\begin{frame}{GB diffusivity vs energy}
	\begin{figure}[ht]
		\centering
		\begin{subfigure}{0.45\textwidth}
			\centering
			\includegraphics[width=\textwidth]
				{diffusion/images/DvsGBE_600K.pdf}
		\end{subfigure}
		\begin{subfigure}{0.45\textwidth}
			\centering
			\includegraphics[width=\textwidth]
				{diffusion/images/DvsGBE_1200K.pdf}
		\end{subfigure}
		\caption{
			Diffusivity vs GB energy at 600 K and 1200 K.
		}
	\end{figure}

	\begin{itemize}
		\item There is not a statiscally significant correlation
			between diffusion coefficients and GB energies.
		\item The R$^2$ of linear fit is 0.02 at 600 K and 0.36 at 1200 K.
	\end{itemize}
\end{frame}

\begin{frame}{Anisotropic nature of diffusion}
	\begin{figure}[ht]
		\centering
		\begin{subfigure}{0.325\textwidth}
			\centering
			\includegraphics[width=\textwidth]
				{diffusion/images/ratio_U.pdf}
		\end{subfigure}
		\begin{subfigure}{0.325\textwidth}
			\centering
			\includegraphics[width=\textwidth]
				{diffusion/images/ratio_Mo.pdf}
		\end{subfigure}
		\begin{subfigure}{0.325\textwidth}
			\centering
			\includegraphics[width=\textwidth]
				{diffusion/images/ratio_Xe.pdf}
		\end{subfigure}
		\caption{
			Ratios of diffusion coefficients parallel to the tilt axis
			to the diffusion coefficients perpendicular to the tilt axis
			of U, Mo, and Xe.
		}
	\end{figure}

	\begin{itemize}
		\item The solid black line represents isotropic GB diffusion.
		\item All high-angle GBs, diffusion behavior is almost isotropic.
		\item (190) and (340) low-angle GBs show highly anisotropic behavior.
	\end{itemize}
\end{frame}

\begin{frame}{Anisotropic nature (cont.)}
	\begin{columns}
	\column{0.5\textwidth}
		\begin{itemize}
			\item For some GBs like (190), diffusion is similar to
				dislocation pipe diffusion.
			\item A symmetric tilt low-angle GB is essnetially
				an array of parallel edge dislocations.
			\item Even at high temperatures,
				the anisotropic nature of diffusion remains.
			\item Isotropic behavior of (130) is shown for comparison.
		\end{itemize}
	\column{0.5\textwidth}
		\begin{figure}[ht]
			\centering
			\begin{subfigure}{0.48\textwidth}
				\centering
				\caption{}
				\includegraphics[width=\textwidth]
					{diffusion/images/130at700cs.png}
			\end{subfigure}
			\begin{subfigure}{0.48\textwidth}
				\centering
				\caption{}
				\includegraphics[width=\textwidth]
					{diffusion/images/130at1100cs.png}
			\end{subfigure}
			\begin{subfigure}{0.48\textwidth}
				\centering
				\caption{}
				\includegraphics[width=\textwidth]
					{diffusion/images/190at700cs.png}
			\end{subfigure}
			\begin{subfigure}{0.48\textwidth}
				\centering
				\caption{}
				\includegraphics[width=\textwidth]
					{diffusion/images/190at1100cs.png}
			\end{subfigure}
			\caption{
				Cross-sectional view of atomic trajectories at
				(a) 700 K and (b) 1100 K of (130),
				and (c) 700 K and (d) 1100 K of (190).
			}
		\end{figure}
	\end{columns}
\end{frame}

\begin{frame}{Anisotropic nature (cont.)}
	\begin{columns}
	\column{0.5\textwidth}
		\begin{itemize}
			\item To add to the visual example,
				mean squared displacements (MSDs) are shown here.
			\item In the isotropic case,
				MSDs parallel and perpendicular to the tilt axis
				are quite similar.
			\item In the anisotropic case, MSD perpendicular to the tilt axis
				is similar to MSD perpendicular to the GB plane,
				indicating an almost one dimensional diffusion.
		\end{itemize}
	\column{0.5\textwidth}
		\begin{figure}[ht]
			\centering
			\begin{subfigure}{0.95\textwidth}
				\centering
				\includegraphics[height=3cm]
					{diffusion/images/130at1100xyz.pdf}
			\end{subfigure}
			\begin{subfigure}{0.95\textwidth}
				\centering
				\includegraphics[height=3cm]
					{diffusion/images/190at1100xyz.pdf}
			\end{subfigure}
			\caption{
				Mean squared displacements of
				(Top) (130) and (Bottom) (190) at 1100 K.
			}
		\end{figure}
	\end{columns}
\end{frame}

\begin{frame}{Two diffusion regimes}
	\begin{figure}[ht]
		\centering
		\begin{subfigure}{0.45\textwidth}
			\centering
			\includegraphics[width=\textwidth]
				{diffusion/images/2reg_U_Dz.pdf}
		\end{subfigure}
		\begin{subfigure}{0.45\textwidth}
			\centering
			\includegraphics[width=\textwidth]
				{diffusion/images/2reg_Mo_Dz.pdf}
		\end{subfigure}
		\caption{
			Arrhenius fits to the two diffusion regimes.
		}
	\end{figure}

	\begin{itemize}
		\item GB diffusivities don't follow Arrhenius equation exactly.
		\item There's concavity, which indicates sub-Arrhenius behavior.
		\item Arrhenius fits are thus separated
			into two different diffusion regimes.
	\end{itemize}
\end{frame}

\begin{frame}{Two diffusion regimes}
	\begin{table}[!ht]
	\centering
	\caption{
		Prefactors and activation energies for the Arrhenius equation fits
		to $\overline{D}^{i}_{\parallel}$ ($i=$ U or Mo)
		in $\gamma$U-7Mo, $\gamma$U-10Mo, and $\gamma$U-12Mo
		for the two different temperature regimes.
		$D_0$ is in m$^2$s$^{-1}$ and $E_a$ is in eV.
	}
	\begin{tabular}{ccllll}
	\toprule
	Composition & Temp. (K)
		& $D_{0}^U$      & $E_{a}^U$
		& $D_{0}^{Mo}$   & $E_{a}^{Mo}$ \\
	\midrule
	\multirow{2}{*}{ $\gamma$U-7Mo }
		& 600 - 800
		& 1.03 $\times 10^{-09}$ & 0.433
		& 8.87 $\times 10^{-11}$ & 0.366 \\
		& 900 - 1200
		& 4.10 $\times 10^{-08}$ & 0.715
		& 2.77 $\times 10^{-08}$ & 0.806 \vspace{0.2cm } \\
	\multirow{2}{*}{ $\gamma$U-10Mo }
		& 600 - 800
		& 1.96 $\times 10^{-10}$ & 0.362
		& 5.26 $\times 10^{-11}$ & 0.353 \\
		& 900 - 1200
		& 2.10 $\times 10^{-08}$ & 0.712
		& 1.31 $\times 10^{-08}$ & 0.779 \vspace{0.2cm } \\
	\multirow{2}{*}{ $\gamma$U-12Mo }
		& 600 - 800
		& 1.81 $\times 10^{-10}$ & 0.380
		& 6.17 $\times 10^{-11}$ & 0.375 \\
		& 900 - 1200
		& 1.33 $\times 10^{-08}$ & 0.700
		& 6.62 $\times 10^{-09}$ & 0.738 \\
	\bottomrule
	\end{tabular}
	\end{table}

	\begin{itemize}
		\item This table is the culmination of the GB diffusion study.
		\item These values can be readily employed
			in higher-length-scale modeling of $\gamma$U-Mo.
	\end{itemize}
\end{frame}

\section{Re-solution of Xe gas bubbles
in \texorpdfstring{$\gamma$}{gamma}U-10Mo}

\begin{frame}{Xe gas bubble re-solution}
\begin{itemize}
	\item Xe gas bubbles act as a sink for individual Xe atoms,
		trapping them and causing the bubbles to grow after absorption.
	\item Under irradiation, the Xe atoms in the gas bubble are reintroduced
		into the fuel matrix through fission-product-induced cascades
		and thermal spikes---a process known as re-solution.
	\item The relative rates of re-solution
		affect the overall size and number density of the bubbles,
		in turn impacting bubble evolution and subsequent fuel swelling.
	\item There is no model for Xe re-solution rate in $\gamma$U-Mo yet.
\end{itemize}
\end{frame}

\begin{frame}{Bubble re-solution}
	\begin{columns}
	\column{0.5\textwidth}
		\begin{itemize}
			\item Fission gases can diffuse to grain boundaries
				to form intergranular bubbles
				where they can grow by addition of gas and vacancies.
			\item Fission also strongly limits bubble size
				(or fission-gas atom population)
				in the fuel lattice by re-solution of gas bubbles.
			\item Individual gas atoms can be ejected from the gas bubble
				through collisions with energetic fission fragments
				or recoil atoms (PKAs).
			\item Bubbles can also be destroyed
				by the passage of fission tracks
				where local temperature can be higher than
				the fuel melting temperature.
		\end{itemize}
	\column{0.5\textwidth}
		\begin{figure}[ht]
			\centering
			\begin{subfigure}{0.6\textwidth}
				\centering
				\caption{}
				\includegraphics[width=\textwidth]{sfigs/homogeneous.png}
			\end{subfigure}

			\begin{subfigure}{0.6\textwidth}
				\centering
				\caption{}
				\includegraphics[width=\textwidth]{sfigs/heterogeneous.png}
			\end{subfigure}
			\caption{
				Illustration of
				(a) homogeneous and (b) heterogeneous re-solution.
			}
		\end{figure}
	\end{columns}
\end{frame}

\begin{frame}{Re-solution setup}
	\begin{itemize}
		\item Two types of simulations are performed:
			primary knock-on atom (PKA)
			and thermal spike (fission track) simulations
			for low- and high-energy interactions.
			These simulations account for re-solution
			through ballistic and thermal processes.
		\item For the PKA model supercell is $160 \times 160 \times 160$
			\r{A}$^3$, $\sim$ 8 million atoms in the system.
			For the thermal spike model,
			the supercell is $160 \times 160 \times 80$ \r{A}$^3$,
			$\sim$ 4 million atoms.
			The simulations are performed in NVE ensembles,
			with an NVT region around the edges.
		\item For the PKA model, PKA energies up to 500 keV,
			and for the thermal spike model, we impart up to 30 keV/nm
			to a cylindrical region containing the gas bubble.
		\item A void is first created in the middle of the supercell,
			then Xe atoms are inserted up to certain Xe/vac ratios.
			Xe atoms that traverse more than 1 nm
			from the surface of the spherical void region
			are counted as re-solved Xe atoms.
	\end{itemize}
\end{frame}

\begin{frame}{PKA simulation progression}
	\begin{columns}
	\column{0.5\textwidth}
		\begin{itemize}
			\item In the PKA model, the imparted energy propagates quickly
				as a shock wave and creates many point defects,
				the majority of which eventually annihilate.
			\item The gas bubble is seen to deform a little
				in the beginning stage of the shock wave,
				however, minimal re-solution is observed.
			\item The re-solution rates, considering uncertainty,
				are effectively zero,
				and are neglected from further examination.
				This is similar to observations in UO$_2$.
		\end{itemize}
	\column{0.5\textwidth}
		\begin{figure}[ht]
			\centering
			\begin{subfigure}{0.49\textwidth}
				\centering
				\caption{}
				\includegraphics[width=\textwidth]{resol/images/pka1.png}
			\end{subfigure}
			\begin{subfigure}{0.49\textwidth}
				\centering
				\caption{}
				\includegraphics[width=\textwidth]{resol/images/pka2.png}
			\end{subfigure}

			\begin{subfigure}{0.49\textwidth}
				\centering
				\caption{}
				\includegraphics[width=\textwidth]{resol/images/pka3.png}
			\end{subfigure}
			\begin{subfigure}{0.49\textwidth}
				\centering
				\caption{}
				\includegraphics[width=\textwidth]{resol/images/pka4.png}
			\end{subfigure}
			\caption{
				Snapshots of a PKA (500 keV) simulation
				at (a) 0, (b) 1.5, (c) 44.5, and (d) 114.5 ps.
				U, Mo, and Xe atoms are shown in red, blue, and black.
			}
		\end{figure}
	\end{columns}
\end{frame}

\begin{frame}{Thermal spike progression}
	\begin{columns}
	\column{0.5\textwidth}
		\begin{itemize}
			\item The thermal spike imparts a large amount of kinetic energy
				to the atoms, which propagates out quickly.
			\item Thermal spikes induce significant re-solution,
				seeming to break apart the entire bubble
				into a loose collection of Xe atoms.
			\item The Xe atoms coalesce to reform a gas bubbles,
				but a significant portion remain outside
				the initial spherical region.
				These re-solved atoms may be isolated
				or form small Xe clusters.
			\item Initial bubble radii are varied from 5 \r{A} to 40 \r{A}.
		\end{itemize}
	\column{0.5\textwidth}
		\begin{figure}
			\begin{subfigure}{0.49\textwidth}
				\centering
				\caption{}
				\includegraphics[width=\textwidth]{resol/images/spike1.png}
			\end{subfigure}
			\begin{subfigure}{0.49\textwidth}
				\centering
				\caption{}
				\includegraphics[width=\textwidth]{resol/images/spike2.png}
			\end{subfigure}

			\begin{subfigure}{0.49\textwidth}
				\centering
				\caption{}
				\includegraphics[width=\textwidth]{resol/images/spike3.png}
			\end{subfigure}
			\begin{subfigure}{0.49\textwidth}
				\centering
				\caption{}
				\includegraphics[width=\textwidth]{resol/images/spike4.png}
			\end{subfigure}
			\caption{
				Snapshots of a thermal spike (30 keV/nm) simulation
				at (a) 0, (b) 1.5, (c) 44.5, and (d) 114.5 ps.
				U, Mo, and Xe atoms are shown in red, blue, and black.
			}
		\end{figure}
	\end{columns}
\end{frame}

\begin{frame}{Bubble size dependence}
	\begin{columns}
	\column{0.5\textwidth}
		\begin{itemize}
			\item The fraction of re-solved atoms is a convenient way
				to plot different bubble sizes for comparisons.
			\item Error bars denote the standard deviations
				calculated from 5 bubbles of each radius.
			\item Data can be fit to an exponentially saturating function:
				$ \chi_0 = 1 - \exp[-\alpha S_{e,eff}] $
			\item $S_{e,eff}$ is the effective energy
				transferred to the lattice
				and $\alpha$ is the saturation factor.
			\item The saturation factor itself is a function of radius:
				$ \alpha = \frac{5.1}{R_{bubble}^{2.2}} $
		\end{itemize}
	\column{0.5\textwidth}
		\begin{figure}[ht]
			\centering
			\begin{subfigure}{\textwidth}
				\centering
				\includegraphics[height=3.0cm]
					{resol/images/resolutionVsRadius.pdf}
			\end{subfigure}
			\begin{subfigure}{\textwidth}
				\centering
				\includegraphics[height=2.5cm]
					{resol/images/saturationFactor.pdf}
			\end{subfigure}
			\caption{
				(Top) Fraction of re-solved Xe atoms
				as a function of the energy deposited to the lattice.
				(Bottom) Saturation factor as a function of bubble radius.
			}
		\end{figure}
	\end{columns}
\end{frame}

\begin{frame}{Off-centered thermal spike}
	\begin{columns}
	\column{0.5\textwidth}
		\begin{itemize}
			\item Off-centered thermal spikes are simulated as well,
				with variable degree of off-centeredness.
				By varying the amount of overlap,
				a spatial distribution can be generated.
			\item At an off-center distance of $r_c := R_{bubble} + R_{spike}$,
				the re-solution stops.
			\item Normalized fraction of re-solved Xe atoms
				as a function of normalized off-centered distance
				can be modeled with a logistic equation:
				$
					\frac{\chi}{\chi_0}
						= \frac{1.058}{1 + \exp \big[8.168
						\big(\frac{r}{r_c}\big) - 3.331 \big]}
				$
		\end{itemize}
	\column{0.5\textwidth}
		\begin{figure}[ht]
			\centering
			\includegraphics[height=4.5cm]{resol/images/offcentered.pdf}
			\caption{
				Xe re-solution caused by
				off-centered thermal spikes (15 keV/nm).
			}
		\end{figure}
	\end{columns}
\end{frame}

\begin{frame}{Effect of bubble pressure}
	\begin{columns}
	\column{0.5\textwidth}
		\begin{itemize}
			\item Xe/vac ratio determines bubble pressure.
				Bubbles with different Xe/vac ratios were simulated
				to evaluate the effect of pressure.
			\item The number of re-solved Xe atoms is apparently invariant
				with respect to the Xe/vac ratio.
			\item With increasing thermal spike energy,
				more atoms are re-solved as expected.
			\item Similar trends were observed in bubbles
				with radii of 15 \r{A} and 35 \r{A}.
		\end{itemize}
	\column{0.5\textwidth}
		\begin{figure}[ht]
			\centering
			\includegraphics[width=\textwidth]{resol/images/xevac.pdf}
			\caption{
				Number of re-solved Xe atoms from 25 \r{A} radius bubbles
				as a function of Xe/vacancy ratio.
			}
		\end{figure}
	\end{columns}
\end{frame}

\begin{frame}{Pressure and size invariance}
	\begin{columns}
	\column{0.5\textwidth}
		\begin{itemize}
			\item The number of re-solved atoms from identical-radius bubbles
				with different pressures were averaged.
			\item The number of re-solved atoms also appears consistent
				across bubbles of various sizes.
			\item Thermal spikes create low-density regions
				around the Xe gas bubble,
				and these regions facilitate the separation of Xe atoms
				from the bubble.
			\item Therefore, thermal spike energy dictates
				the volume of these low-energy regions,
				and thus affects the total number of re-solved atoms.
		\end{itemize}
	\column{0.5\textwidth}
		\begin{figure}[ht]
			\centering
			\includegraphics[width=\textwidth]{resol/images/r2dep.pdf}
			\caption{
				Number of re-solved Xe atoms against bubble radius.
				Data from identical-radius bubbles
				with different Xe/vacancy ratios were averaged.
			}
		\end{figure}
	\end{columns}
\end{frame}

\begin{frame}{Calculation of re-solution rate}
	\begin{columns}
	\column{0.5\textwidth}
		\begin{itemize}
			\item We need to sum up all the contributions
				from all the fission products originating
				at different distances from a bubble and different offsets
				by means of a volume integral.
			\item The integral can be expressed explicitly
				in terms of fission product yields,
				and $\zeta = S_{e,eff} / S_{e}$,
				the fraction of FP energy that is imparted to the lattice.
			\item
				$
					f(R_{bubble}, \dot{F}) $
				$
					= \sum_{i=1}^2 \int_{x} \int_{r}
						\chi_i \dot{F} 2 \pi r dr dx $
				$
					= \dot{F} \sum_{i=1}^2 \int_{r}
						\bigg( \frac{\chi_i}{\chi_{0,i}} \bigg)
						2 \pi r dr \int_{x} \chi_{0,i} dx $
		\end{itemize}
	\column{0.5\textwidth}
		\begin{figure}[ht]
			\centering
			\includegraphics[height=5cm]{resol/images/coordSystem.pdf}
			\caption{
				Cylindrical coordinate system for calculating
				the heterogeneous re-solution rate.
			}
		\end{figure}
	\end{columns}
\end{frame}

\begin{frame}{Electronic stopping power}
	\begin{columns}
	\column{0.5\textwidth}
		\begin{itemize}
			\item The electronic stopping power of the fission products (FPs)
				Xe and Sr in U-10Mo are calculated using SRIM.
			\item Ion irradiation is simulated 2000 times
				for each fission product.
			\item Equations are fit to the data, which can be integrated,
				allowing for the calculation of the re-solution rate.
			\item
				$
					S_{e,Xe} = 21.3 \exp(-0.239 x^{1.78})
						+ 5.23 \exp(-4.67 \times 10^{-8} x^{11}) $
			\item
				$
					S_{e,Sr} = 19.7 \exp(-0.00273 x^{3.71})
						+ 6.8 \exp(-0.424 x^{1.45}) $
		\end{itemize}
	\column{0.5\textwidth}
		\begin{figure}[ht]
			\centering
			\includegraphics[height=4.5cm]{resol/images/elec_stopping.pdf}
			\caption{
				Total electronic stopping power ($S_e$)
				of Xe-140 and Sr-94 in $\gamma$U-10Mo,
				as calculated by the SRIM software as a function of
				distance traversed by the fission product
				from the location of the fission reaction.
			}
			\label{fig:elec}
		\end{figure}
	\end{columns}
\end{frame}

\begin{frame}{Re-solution rate at nominal pressure}
	\begin{columns}
	\column{0.5\textwidth}
		\begin{itemize}
			\item The re-solution rate for the nominal pressure
				is then computed using numerical integration.
			\item An approximate analytical form is:
				$ f(R_{bubble}, \dot{F}) $
				$ = \frac{a}{1+(R_{bubble}/c)^d} 10^{-14} \dot{F} $
			\item The DART model is defined as:
				$ b_{dart} = b_0 \cdot \dot{F} \cdot G $
				where
				$ b_0 = R_{spike}^2 \cdot \mu_{ff} $
				and
				$ G =
					\begin{cases}
						1 \qquad \qquad \qquad \;\;\; ,
							R_{bubble} \leq \lambda \\
						1 - (\frac{R_{bubble}-R_{resol}}{R_{bubble}})^3,
							R_{bubble} > \lambda
					\end{cases} $
			\item We have a smooth function for re-solution rate now
				instead of a piecewise one.
		\end{itemize}
	\column{0.5\textwidth}
		\begin{figure}[ht]
			\centering
			\includegraphics[width=6cm]{resol/images/resRate_withDart.pdf}
			\caption{
				Xe gas bubble re-solution rate in $\gamma$U-10Mo
				as a function of bubble radius for Xe/vac ratio of 0.2
				at a fission rate of $10^{14}$ fiss/cm$^3$/s for $\zeta=0.01$.
			}
		\end{figure}
	\end{columns}
\end{frame}

\begin{frame}{Final re-solution rate}
	\begin{columns}
	\column{0.5\textwidth}
		\begin{itemize}
			\item With the re-solution rate
				for the nominal pressure already provided,
				accounting for bubble pressure is straightforward
				due to the pressure invariance we have observed.
			\item
				$ b_{het}(R_{bubble}, \dot{F}, \phi) $
				$ = f(R_{bubble}, \dot{F}) \cdot g(\phi) $
				$ = \bigg[ \frac{a}{1+(R_{bubble}/c)^d} 10^{-14} \dot{F} \bigg]
					\bigg( \frac{0.2}{\phi} \bigg) $
			\item Here, $a$, $c$, and $d$  depend on the value of $\zeta$.
			\item Re-solution rate decreases with
				both bubble size and pressure.
		\end{itemize}
	\column{0.5\textwidth}
		\begin{figure}[ht]
			\centering
			\includegraphics[trim={4cm 2cm 1cm 2cm}, clip, width=\textwidth]
				{resol/images/3d.pdf}
			\caption{
				Xe gas bubble re-solution rate in $\gamma$U-10Mo
				as a function of bubble radius and Xe/vacancy ratio
				at a fission rate of $10^{14}$ fiss/cm$^3$/s for $\zeta=0.01$.
			}
		\end{figure}
	\end{columns}
\end{frame}

\section{IUQ of \texorpdfstring{$\gamma$}{gamma}U-10Mo
fission-gas-behavior parameters}

\begin{frame}{Quantifying unknown DART parameters}
	\begin{itemize}
		\item The key challenge in simulating fission gas swelling
			with a mechanistic model is
			obtaining key material properties related to gas bubble behavior,
			as many of them cannot be measured experimentally.
		\item For the parameters that do not have measurement data
			or atomic-scale simulation results,
			they are usually estimated by either fitting to measured bubble morphology
			or by borrowing from other similar fuel systems where the data is available.
		\item The fission-gas-behavior parameters used in the GRASS module of DART
			were calibrated in a previous study,
			using the bubble size distributions
			measured from irradiated $\gamma$U-10Mo dispersion fuel particles.
		\item However, this set of parameters needs recalibration
			because new atomic-scale data have become available
			since the previous calibration.
	\end{itemize}
\end{frame}

\begin{frame}{Fission-gas-behavior parameters}
	\begin{table}[ht]
	\centering
	\caption{
		Fission gas behavior parameters and their ranges.
	}
	\begin{tabular}{lccc}
	\toprule
	Parameter    & Minimum value & Maximum value & Reference value \\
	\midrule
	dGrainHBS    & \num{2.4e-5 } & \num{5.6e-5 } & \num{4e-5   }   \\
	FaceCovMax   & \num{0.6    } & \num{0.907  } & \num{0.907  }   \\
	SwellLink    & \num{0.016  } & \num{0.034  } & \num{0.025  }   \\
	vResol       & \num{1.3e-18} & \num{2.7e-18} & \num{2e-18  }   \\
	DatomFissGBx & \num{19641  } & \num{40530  } & \num{3e4    }   \\
	fNucleate    & \num{3.5e-10} & \num{8.3e-10} & \num{6e-10  }   \\
	aAtomDifFiss & \num{3.2e-31} & \num{7.2e-31} & \num{5.1e-31}   \\
	\bottomrule
	\end{tabular}
	\end{table}

	\begin{itemize}
		\item DART is run with 7 fission-gas-behavior parameters perturbed
			to compile a dataset necessary for IUQ.
		\item 3200 samples were generated at different fission density values.
		\item The list above contains the names of the parameters
			we are interested in, and their probed ranges.
	\end{itemize}
\end{frame}

\begin{frame}{Inverse uncertainty quantification (IUQ)}
	\begin{columns}
	\column{0.5\textwidth}
		\begin{itemize}
			\item IUQ is a process to quantify uncertainties
				in the input parameters of a computer model
				given experimental data.
			\item It's the opposite of
				forward uncertainty quantification (FUQ),
				which quantifies the uncertainty in the output given the input.
			\item IUQ quantifies the uncertainty in the model itself,
				and thus can help to improve
				the accuracy of the model predictions.
			\item It's helpful in situations where the underlying parameters
				of the model are unknown.
		\end{itemize}
	\column{0.5\textwidth}
		\begin{figure}[ht]
			\centering
			\includegraphics[width=6cm]{sfigs/wu_iuq.png}
			\caption{
				Some essential parts of modeling and simulation
				(Wu et al. {\color{blue}
				\url{https://doi.org/10.1016/j.nucengdes.2018.06.004}}).
			}
		\end{figure}
	\end{columns}
\end{frame}

\begin{frame}{Bayesian framework for IUQ}
	\begin{columns}
	\column{0.4\textwidth}
		\begin{itemize}
			\item $y^E$ is the observation
				and $y^R$ is the reality.
				$y^M$ is the model with a bias $\delta(x)$.
			\item $\epsilon$ is the measurement error
				and assumed to be Gaussian.
			\item To get posterior distributions of parameters $\theta^*$,
				Bayes' theorem can be used.
			\item Since the error is related to $y^E$ and $y^M$,
				the likelihood can be described in terms of the error.
		\end{itemize}
	\column{0.6\textwidth}
		\begin{align}
			\bm{y}^R(\bm{x})
			&= \bm{y}^M(\bm{x}, \bm{\theta}^*) + \delta(\bm{x}) \\
			\bm{y}^E(\bm{x})
			&= \bm{y}^R(\bm{x}) + \bm{\epsilon} \\
			\bm{y}^E(\bm{x})
			&= \bm{y}^M(\bm{x}, \bm{\theta}^*)
				+ \delta(\bm{x}) + \bm{\epsilon} \\
			\bm{\epsilon}
			&= \bm{y}^E(\bm{x}) - \bm{y}^M(\bm{x}, \bm{\theta}^*)
				- \delta(\bm{x}) \\
			\bm{\epsilon}
			&\sim \mathcal{N} (\bm{\mu}, \bm{\Sigma}_{exp}) \\
			p(\bm{\theta}^* | \bm{y}^E, \bm{y}^M)
			&\propto p(\bm{\theta}^*)
				\cdot p(\bm{y}^E, \bm{y}^M | \bm{\theta}^*) \\
			\propto p(\bm{\theta}^*)
				&\cdot \frac{1}{\sqrt{|\bm{\Sigma}|}}
				\exp \bigg[-\frac{1}{2} \bm{\epsilon}^T
				\bm{\Sigma}^{-1} \bm{\epsilon} \bigg] \\
			\bm{\Sigma}
			&= \bm{\Sigma}_{exp}
				+ \bm{\Sigma}_{bias} + \bm{\Sigma}_{code}
		\end{align}
	\end{columns}
\end{frame}

\begin{frame}{Experimental observation}
	\begin{columns}
	\column{0.5\textwidth}
		\begin{itemize}
			\item For our problem, the observation comes from
				the existing fuel swelling correlations.
			\item The goal is thus to find
				the fission-gas-behavior parameter distributions
				that correspond to the experimental correlations.
			\item The Robinson correlation is created
				using data from different irradiation experiments
				and different operational conditions.
			\item DART predictions are very narrow at low fission densities
				and thus not helpful for IUQ purposes.
		\end{itemize}
	\column{0.5\textwidth}
		\begin{figure}[ht]
			\centering
			\begin{subfigure}{\textwidth}
				\centering
				\includegraphics[width=0.7\textwidth]
					{sfigs/robinson_corr.png}
			\end{subfigure}
			\begin{subfigure}{\textwidth}
				\centering
				\includegraphics[width=0.7\textwidth]
					{invuq/images/robinson_vs_dart.png}
			\end{subfigure}
			\caption{
				(Top) Robinson fuel swelling correlation.
				(Bottom) Robinson correlation and DART prediction.
			}
		\end{figure}
	\end{columns}
\end{frame}

\begin{frame}{Target for IUQ}
	\begin{columns}
	\column{0.5\textwidth}
		\begin{itemize}
			\item We chose the highest fission density
				($7 \times 10^{21}$ fiss/cm$^3$) data for IUQ.
			\item At this fission density,
				DART prediction covers the entire prediction bound
				of the Robinson correlation.
			\item The goal is to find a set of parameter distributions
				that leads to a model prediction
				matching the experimental observation.
		\end{itemize}
	\column{0.5\textwidth}
		\begin{figure}[ht]
			\centering
			\includegraphics[width=5cm]{invuq/images/target.png}
			\caption{
				Robinson correlation and DART prediction
				at $7 \times 10^{21}$ fiss/cm$^3$.
			}
		\end{figure}
	\end{columns}
\end{frame}

\begin{frame}{Data exploration: scatter plots}
	\begin{figure}[ht]
		\centering
		\includegraphics[width=8cm]{invuq/images/scatter.png}
		\caption{
			Scatter plots of fuel swelling
			against the fission-gas-behavior parameters.
		}
	\end{figure}

	\begin{itemize}
		\item Except for the parameters "dGrainHBS" and "FaceCovMax",
			all other parameters have weak correlations with fuel swelling.
	\end{itemize}
\end{frame}

\begin{frame}{Data exploration: correlation heatmap}
	\begin{columns}
	\column{0.5\textwidth}
		\begin{itemize}
			\item To also understand the correlation among the parameters,
				correlation heatmap is shown.
			\item The parameters themselves are also
				uncorrelated to each other.
			\item The heatmap corroborates the observation
				from the scatter plots.
			\item The parameter "SwellLink" seems to have
				a non-negligible correlation,
				along with "dGrainHBS" and "FaceCovMax".
		\end{itemize}
	\column{0.5\textwidth}
		\begin{figure}[ht]
			\centering
			\includegraphics[width=5cm]{invuq/images/corrheat.png}
			\caption{
				Correlation heatmap of fission-gas-behavior parameters
				and fuel swelling.
			}
		\end{figure}
	\end{columns}
\end{frame}

\begin{frame}{Surrogate modeling}
	\begin{columns}
	\column{0.5\textwidth}
		\begin{itemize}
			\item We need to perturb the parameter space
				to match the observation.
			\item But it's not feasible to run calculations
				for all possible parameter combinations
				using DART.
			\item A surrogate model
				(also called a response surface or an emulator)
				is thus needed.
			\item Many machine learning models,
				such as Gaussian processes and neural networks,
				can act as a surrogate model.
		\end{itemize}
	\column{0.5\textwidth}
		\begin{figure}[ht]
			\centering
			\includegraphics[width=\textwidth]{sfigs/surrogate_modeling.png}
			\caption{
				The principle of surrogate modeling
				(Kocijan et al.
				{\color{blue}\url{https://doi.org/10.5516/NET.07.2014.706}}).
			}
		\end{figure}
	\end{columns}
\end{frame}

\begin{frame}{Surrogate models}
	\begin{columns}
	\column{0.5\textwidth}
		\begin{itemize}
			\item A few linear (OLS, Lasso, and Ridge)
				and non-linear (GP, NN, and SVR)
				are used as surrogate models.
			\item The training and testing of most surrogate models
				are done on a 70/30 split.
				Only the GP model is trained with 14\% of the total data.
			\item All surrogate models have R$^2$ scores around 0.997.
			\item Lasso made the coefficients of 4 parameters zero,
				thus selecting the parameters
				"dGrainHBS", "FaceCovMax", and "SwellLink"
				as the influential ones.
		\end{itemize}
	\column{0.5\textwidth}
		\begin{figure}[ht]
			\centering
			\begin{subfigure}{0.49\textwidth}
				\centering
				\includegraphics[width=\textwidth]
					{invuq/images/ols.png}
			\end{subfigure}
			\begin{subfigure}{0.49\textwidth}
				\centering
				\includegraphics[width=\textwidth]
					{invuq/images/lasso.png}
			\end{subfigure}
			\begin{subfigure}{0.49\textwidth}
				\centering
				\includegraphics[width=\textwidth]
					{invuq/images/gp.png}
			\end{subfigure}
			\begin{subfigure}{0.49\textwidth}
				\centering
				\includegraphics[width=\textwidth]
					{invuq/images/nn.png}
			\end{subfigure}
			\caption{
				Surroage prediction vs actual test data
				for surrogate models.
			}
		\end{figure}
	\end{columns}
\end{frame}

\begin{frame}{MCMC samples}
	\begin{figure}[ht]
		\centering
		\includegraphics[width=11cm]{invuq/images/trace.png}
		\caption{
			Trace plots of the posterior samples
			of the fission-gas-behavior parameters from MCMC sampling.
		}
	\end{figure}

	\begin{itemize}
		\item Two chains of 100,000 samples are overlaid
			on the trace plots.
			All diagnostics point to convergence.
		\item The cumulative averages of the blue and orange traces
			are shown in black and red, respectively.
		\item "FaceCovMax" and "SwellLink" are essentially bounded
			by the priors.
	\end{itemize}
\end{frame}

\begin{frame}{Posterior distributions}
	\begin{figure}[ht]
		\centering
		\includegraphics[width=11cm]{invuq/images/hist.png}
		\caption{
			Density plots of the posterior samples
			of the fission-gas-behavior parameters from MCMC sampling.
		}
	\end{figure}

	\begin{itemize}
		\item The density plots (also posteriors) from both MCMC chains
			are similar.
		\item The posterior of "dGrainHBS" is reminiscent
			of a Gaussian distribution.
		\item The other two parameter distributions are more akin
			to a uniform distribution.
		\item The ranges of the parameters "FaceCovMax" and "SwellLink"
			might have been chosen in a constricted manner.
	\end{itemize}
\end{frame}

\begin{frame}{Joint probability distributions}
	\begin{columns}
	\column{0.5\textwidth}
		\begin{itemize}
			\item MCMC samples are thinned
				so that the number of samples is approximately equal
				to the effective sample size (ESS).
				Thinning didn't change
				the characteristics of the distributions.
			\item The correlations among the posteriors are displayed
				as contour plots.
			\item Even though the prior distributions
				are assumed to be independent,
				IUQ still identified the correlations among the posteriors.
			\item "dGrainHBS" and "FaceCovMax"
				have a strong positive correlation,
				and "dGrainHBS" and "SwellLink" have a weak one.
		\end{itemize}
	\column{0.5\textwidth}
		\begin{figure}[ht]
			\centering
			\includegraphics[width=5cm]{invuq/images/iuq.png}
			\caption{
				Posterior distributions of 3 fission-gas-behavior parameters.
				Marginal densities are in the diagonal
				and the pair-wise joint densities are in the off-diagonal.
			}
		\end{figure}
	\end{columns}
\end{frame}

\begin{frame}{FUQ and validation}
	\begin{columns}
	\column{0.5\textwidth}
		\begin{itemize}
			\item To validate IUQ results,
				posterior samples are propagated using the GP surroate.
			\item A multivariate Gaussian copula is fitted to the MCMC samples.
				Synthetic samples are then generated using the copula,
				and then propagated using the GP surrogate.
			\item The fitted copula does a decent job,
				although it's not as accurate.
				The lower accuracy is possibly due to
				the usage of a Gaussian copula with uniform-like posteriors.
		\end{itemize}
	\column{0.5\textwidth}
		\begin{figure}[ht]
			\centering
			\begin{subfigure}{0.60\textwidth}
				\centering
				\includegraphics[width=\textwidth]
					{invuq/images/fuq_gp.png}
			\end{subfigure}
			\begin{subfigure}{0.60\textwidth}
				\centering
				\includegraphics[width=\textwidth]
					{invuq/images/fuq_gp_synth.png}
			\end{subfigure}
			\caption{
				Forward propagation using GP surrogate with
				(Top) posterior samples from MCMC
				and (Bottom) synthetic samples generated using copulas.
			}
		\end{figure}
	\end{columns}
\end{frame}

\section{Conclusions}

\begin{frame}{Conclusions}
	With the application of various computational methods,
	we quantified crucial material properties of $\gamma$U-Mo fuel.
	This will assist the higher-length-scale models
	in making more robust predictions.
	Overall, this dissertation achieves the following:
	\begin{itemize}
		\item Fully quantifies GB diffusion coefficients of U, Mo, and Xe
			in $\gamma$U-Mo for various composition, temperatures,
			and GB orientations.
			The findings also provide insight into GB anisotropy.
		\item Provides a mathematical model for Xe re-solution rate,
			taking into account both bubble size and pressure.
			This work considered
			both homogeneous and heterogeneous re-solution rate.
		\item Performs IUQ on still unknown fission-gas-behavior parameters.
			Three influential parameters are identified
			and subsequently optimized for better meso-scale simulation
			of $\gamma$U-Mo fuel.
	\end{itemize}
\end{frame}

\begin{frame}{Future work}
	\begin{itemize}
		\item Further investigation is needed to evaluate
			the relative dominance of homogeneous and heterogeneous mechanisms.
			The two-temperature model will be employed to quantify
			the amount of energy transferred
			from the electronic subsystem to the ionic subsystem.
		\item The IUQ process will be expanded to incorporate new data,
			which takes into account variations in fission rate and grain size.
			Surrogate models will be developed for new data,
			enabling a refined IUQ to better quantify parameter distributions
			pertient to a wide range of reactor conditions.
		\item Machine-learned interatomic potentials describing U, Mo
			and other elements, such as Al and Zr,
			will be developed and evaluated.
			Successful implementation of this will enable MD simulations
			of $\gamma$U-Mo fuel with liners and/or claddings.
	\end{itemize}
\end{frame}

\section*{Thanks}

\begin{frame}
	\centering \Large
	\emph{Thank you!}
\end{frame}

\end{document}
