\documentclass[10pt]{beamer}
\usepackage{booktabs, subcaption, multirow, xspace}
\usepackage{amsmath, amssymb, bm, siunitx}

% Ion shorthand
\usepackage[version=4]{mhchem}
\newcommand{\Y}{\ce{_{39}^{97}Y}\xspace}
\newcommand{\I}{\ce{_{53}^{136}I}\xspace}

\usetheme{CambridgeUS}
\usecolortheme{beaver}
\setbeamertemplate{caption}[numbered]
\setbeamerfont{caption}{size=\scriptsize}

\AtBeginSection[]
{
  \begin{frame}
    \frametitle{Table of Contents}
    \tableofcontents[currentsection]
  \end{frame}
}

\title[Evaluation of U-Mo Microstructural Properties]
{Evaluation of Microstructural Properties
of U-Mo Monolithic Fuel using Computational Methods}

\author{ATM Jahid Hasan}
\institute[NCSU]{Department of Nuclear Engineering\\
North Carolina State University}
\date{January 28, 2026}

\begin{document}
\frame{\titlepage}

\begin{frame}{Table of Contents}
	\tableofcontents
\end{frame}

\section{Introduction}

\begin{frame}{Introduction}
	\begin{columns}
	\column{0.5\textwidth}
		\begin{itemize}
			\item Research reactors operate at low temperature and pressure
				but at high fission density and specific power.
				Thus, RRs historically needed highly enriched uranium (HEU).
			\item With the inception of RERTR in 1978,
				there has been an ongoing effort
				to change HEU into low enriched uranium (LEU).
			\item U$_3$Si$_2$ is the only qualified LEU for RRs.
			\item High-performance RRs need even higher uranium density.
		\end{itemize}
	\column{0.5\textwidth}
		\begin{figure}[ht]
			\centering
			\includegraphics[width=6cm]{sfigs/historic_fuels.png}
			\caption{
				U-density of research reactor fuel vs time of first use.
				(Jamison et al.
				{\color{blue}\url{https://doi.org/10.2172/1890226}}).
			}
		\end{figure}
	\end{columns}
\end{frame}

\begin{frame}{Introduction (cont.)}
	\begin{columns}
	\column{0.5\textwidth}
		\begin{itemize}
			\item US High-Performance Research Reactor (USHPRR) program
				selected U-10Mo fuel for qualification.
			\item U-10Mo monolithic fuel design involves
				a solid U-10Mo fuel meat monded to Al-6061 cladding
				with a zirconium diffusion barrier.
			\item The U-Mo dispersion fuel design has also been tested.
				The dispersion fuel shows breakaway swelling,
				associated with the formation of Mo-stabilized high-Al
				intermetallic phase.
		\end{itemize}
	\column{0.5\textwidth}
		\begin{figure}[ht]
			\centering
			\includegraphics[width=5cm]{sfigs/umo_design.jpg}
			\caption{
				Depiction of monolithic fuel cross-section
				(Meyer et al.
				{\color{blue}\url{https://doi.org/10.5516/NET.07.2014.706}}).
			}
		\end{figure}
	\end{columns}
\end{frame}

\begin{frame}{Swelling in U-Mo}
	\begin{columns}
	\column{0.5\textwidth}
		\begin{itemize}
			\item Excessive swelling during operation is a major concern
				for U-Mo fuels.
			\item With increasing bunrup,
				the bubble population increases in the grain boundaries
				and newer grain boundaries form.
			\item This grain refinement is accompanied by an increase
				in bubble size and number,
				which increases fuel swelling rate.
			\item To account for it in the fuel design,
				fuel swelling needs to be predicted
				for different operational conditions.
		\end{itemize}
	\column{0.5\textwidth}
		\begin{figure}[ht]
			\centering
			\includegraphics[width=4cm]{sfigs/umo_burnup.png}
			\caption{
				U-Mo microstructure as a function of burnup
				(Kim et al. {\color{blue}
				\url{https://doi.org/10.1016/j.jnucmat.2011.08.018}}).
			}
		\end{figure}
	\end{columns}
\end{frame}

\begin{frame}{Dispersion Analysis Research Tool (DART)}
	\begin{columns}
	\column{0.5\textwidth}
		\begin{itemize}
			\item DART is a mechanistic rate-theory-based meso-scale model
				for the calculation of fuel swelling.
			\item The code uses material properties,
				such as gas atom diffusivity, recrystallization kinetics,
				and gas re-solution rate.
			\item Many of material properties used by DART are still unknown.
			\item These properties can be gleaned
				from lower-length-scale studies
				or through parameter optimization using experimental data.
		\end{itemize}
	\column{0.5\textwidth}
		\begin{figure}[ht]
			\centering
			\includegraphics[width=6cm]{sfigs/dart_schematic.png}
			\caption{
				DART schematic.
				(Ye et al. {\color{blue}
				\url{https://doi.org/10.1016/j.jnucmat.2023.154542}}).
			}
		\end{figure}
	\end{columns}
\end{frame}

% TODO: might be better in the intro
%
% \begin{frame}{Xe gas bubble re-solution}
% \begin{itemize}
% 	\item Xe gas bubbles act as a sink for individual Xe atoms,
% 		trapping them and causing the bubbles to grow after absorption.
% 	\item Under irradiation, the Xe atoms in the gas bubble are reintroduced
% 		into the fuel matrix through fission-product-induced cascades
% 		and thermal spikes---a process known as re-solution.
% 	\item The relative rates of re-solution
% 		affect the overall size and number density of the bubbles,
% 		in turn impacting bubble evolution and subsequent fuel swelling.
% 	\item There is no model for Xe re-solution rate in U-Mo yet.
% \end{itemize}
% \end{frame}

\section{Grain boundary diffusion in \texorpdfstring{$\gamma$}{gamma}U-Mo}

\begin{frame}{Grain boundary (GB) diffusion in $\gamma$U-Mo}
	\begin{itemize}
		\item Accurate calculation of fuel swelling requires
			diffusion coefficients of the related species in the fuel.
		\item  Furthermore, creep modeling also requires diffusion coefficients
			to determine creep rates and evaluate the evolving microstructure.
		\item Therefore, it is essential to understand
			the diffusion behavior of the $\gamma$U-Mo fuel.
		\item The diffusion coefficients of the relevant species
			in $\gamma$U-Mo grain boundaries (GBs) are yet unknown.
	\end{itemize}
\end{frame}

\begin{frame}{Simulation setup}
	\begin{columns}
	\column{0.5\textwidth}
		\begin{itemize}
			\item Bicrystals for simulations are created
				by rotating two halves of the system around a tilt axis.
			\item Two GBs: one in the middle and one at the edge.
			\item GB region is defined using atomic trajectories.
			\item We tracked lattice point jumps.
				GB region can be differentiated from the bulk
				by determining lattice points associated with jumps.
		\end{itemize}
	\column{0.5\textwidth}
		\begin{figure}[ht]
			\centering
			\begin{subfigure}{1.0\textwidth}
				\centering
				\includegraphics[width=\textwidth]
					{diffusion/images/configuration.png}
			\end{subfigure}

			\begin{subfigure}{0.7\textwidth}
				\centering
				\includegraphics[width=0.7\textwidth]
					{diffusion/images/gb_def.png}
			\end{subfigure}
			\caption{
				(Top) $\Sigma 5$ Symmetric tilt GB initial configuration.
				(Bottom) Definition of GB based on atomic trajectories.
			}
		\end{figure}
	\end{columns}
\end{frame}

\begin{frame}{Explored GBs}
	\begin{itemize}
		\item Three compositions: $\gamma$U-7Mo, $\gamma$U-10Mo, and $\gamma$U-12Mo
			are examined.
		\item The temperature of the simulated GBs ranged from 600 K to 1200 K
			with an interval of 100 K.
			The dimensions of the supercells are at least
			$50 \times 200 \times 50$ \AA$^3$.
			This corresponds to at least 35,000 atoms.
		\item The following symmetric tilt GBs are studied:
			\begin{itemize}
				\item (120)
				\item (130)
				\item (150)
				\item (190)
				\item (340)
				\item (350)
			\end{itemize}
		\item Other simulated systems are:
			\begin{itemize}
				\item asymmetric (110)
				\item asymmetric (130)
				\item asymmetric (190)
				\item asymmetric (350)
				\item twist (110)
				\item twist (230)
			\end{itemize}
	\end{itemize}
\end{frame}

\begin{frame}{GB validation}
	\begin{columns}
	\column{0.5\textwidth}
		\begin{itemize}
			\item For validation, we computed GB energies
				with the configurations first.
			\item The energies are within 2$\sigma$ of the values
				reported by Beeler et al.
			\item Beeler et al. employed roughly 10 times fewer atoms.
			\item The average GB energy for the examined systems
				is about 0.68 Jm$^{-2}$.
		\end{itemize}
	\column{0.5\textwidth}
		\begin{figure}[ht]
			\centering
			\includegraphics[width=6cm]{diffusion/images/gbe.pdf}
			\caption{
				GB energies as a function of misorientation angle.
			}
		\end{figure}
	\end{columns}
\end{frame}

\begin{frame}{GB width}
	\begin{columns}
	\column{0.5\textwidth}
		\begin{itemize}
			\item GB width in this work means the structural width of a GB.
			\item This is different from the diffusional width,
				where width is calculated using the number of mobile GB atoms.
			\item GB width increases linearly with temperature.
			\item The width is about 6 \r{A} at 600 K
				and increases to about 12 \r{A} at 1200 K.
			\item In literature, GB width assumptions fall between
				5 \r{A} to 15 \r{A}.
		\end{itemize}
	\column{0.5\textwidth}
		\begin{figure}[ht]
			\centering
			\includegraphics[width=6cm]{diffusion/images/d_gb_sym.pdf}
			\caption{
				Widths of symmetric tilt GBs as a function of temperature.
			}
		\end{figure}
	\end{columns}
\end{frame}

\begin{frame}{GB diffusivities}
	\begin{columns}
	\column{0.5\textwidth}
		\begin{itemize}
			\item Diffusion behavior is generally Arrhenius.
			\item The spread in diffusion coefficients due to tilt angles
				is about one order or magnitude.
			\item Diffusivity ranges from $10^{-14}$
				to $10^{-11}$ m$^2$s$^{-1}$.
			\item The activation energy is about 0.4--0.7 eV for U.
			\item Diffusion perpendicular to the tilt axis
				has similar characteristics.
		\end{itemize}
	\column{0.5\textwidth}
		\begin{figure}[ht]
			\centering
			\includegraphics[width=6cm]{diffusion/images/u10mo_U_Dz.pdf}
			\caption{
				Diffusion coefficients of U parallel to the GB tilt axis.
			}
		\end{figure}
	\end{columns}
\end{frame}

\begin{frame}{Symmetric tilt GB diffusivities}
	\begin{figure}[ht]
		\centering
		\begin{subfigure}{0.325\textwidth}
			\centering
			\includegraphics[width=\textwidth]
				{diffusion/images/u10mo_U_Dz.pdf}
		\end{subfigure}
		\begin{subfigure}{0.325\textwidth}
			\centering
			\includegraphics[width=\textwidth]
				{diffusion/images/u10mo_Mo_Dz.pdf}
		\end{subfigure}
		\begin{subfigure}{0.325\textwidth}
			\centering
			\includegraphics[width=\textwidth]
				{diffusion/images/u10mo_Xe_Dz.pdf}
		\end{subfigure}
		\caption{
			Diffusion coefficients parallel to the GB tilt axis
			for symmetric tilts GBs.
		}
	\end{figure}

	\begin{itemize}
		\item Mo diffusivity is lower than that of U.
		\item Xe diffusivity is the higher than that of other species
			at high temperatures.
		\item At low temperatures, all three species have similar coefficients.
	\end{itemize}
\end{frame}

\begin{frame}{Asymmetric tilt \& Twist GB diffusivities}
	\begin{figure}[ht]
		\centering
		\begin{subfigure}{0.45\textwidth}
			\centering
			\includegraphics[width=\textwidth]
				{diffusion/images/asym_twist_U_Dz.pdf}
		\end{subfigure}
		\begin{subfigure}{0.45\textwidth}
			\centering
			\includegraphics[width=\textwidth]
				{diffusion/images/asym_twist_Mo_Dz.pdf}
		\end{subfigure}
		\caption{
			Diffusion coefficients parallel to the GB tilt axis
			for asymmetric tilt and twist GBs.
		}
	\end{figure}

	\begin{itemize}
		\item The diffusion behavior is similar
			for asymmetric tilt and twist GBs.
		\item U diffusivity is higher than that of Mo.
		\item Overall impact of orientation of GBs on diffusivity
			appears to be minimal.
	\end{itemize}
\end{frame}

\begin{frame}{Orientation-averaged GB diffusivities}
	\begin{figure}[ht]
		\centering
		\begin{subfigure}{0.45\textwidth}
			\centering
			\includegraphics[width=\textwidth]
				{diffusion/images/comp_U_Dz.pdf}
		\end{subfigure}
		\begin{subfigure}{0.45\textwidth}
			\centering
			\includegraphics[width=\textwidth]
				{diffusion/images/comp_Mo_Dz.pdf}
		\end{subfigure}
		\caption{
			Orientation-averaged GB diffusivities parallel to the tilt axis
			for $\gamma$U-7Mo, $\gamma$U-10Mo, and $\gamma$U-12Mo.
		}
	\end{figure}

	\begin{itemize}
		\item To compare diffusion of different compositions,
			diffusivities parallel to the tilt axis of all symmetric tilt GBs
			are averaged.
		\item GB diffusivity is negatively correlated with the Mo content.
	\end{itemize}
\end{frame}

\begin{frame}{Comparison with literature}
	% NOTE: From Beeler: you are plotting self diffusion,
	% but you are making comparisons to interstitial diffusion in the text?
	% Be careful about the comparisons here
	\begin{columns}
	\column{0.5\textwidth}
		\begin{itemize}
			\item The difference between GB diffusivity and self-diffusivity
				grows larger with decreasing temperature.
			\item With GB diffusion enhancement factors and GB widths,
				we can calculate effective diffusion coefficient $D_{eff}$:
				$ D_{eff} = f D_{gb} + (1-f) D_{\ell} $,
				where $D_{\ell}$ is the bulk diffusion coefficient
				and $f$ is the volume fraction of GBs.
			\item Assuming 10 $\mu$m grain size,
				we find that $D_{eff} \approx 1.5 \times 10^3 D_{\ell}$ at 600 K
				and $D_{eff} \approx 1.1 D_{\ell}$ at 1200 K.
			\item Overall diffusion is dominated by GB diffusion
				at lower temperatures.
		\end{itemize}
	\column{0.5\textwidth}
		\begin{figure}[ht]
			\centering
			\includegraphics[width=0.9\textwidth]
				{diffusion/images/newLitComp.pdf}
			\caption{
				GB vs bulk diffusivities.
			}
		\end{figure}
	\end{columns}
\end{frame}

\begin{frame}{Effect of the misorientation angle}
	\begin{figure}[ht]
		\centering
		\begin{subfigure}{0.45\textwidth}
			\centering
			\includegraphics[width=\textwidth]
				{diffusion/images/DvsTilt_600K.pdf}
		\end{subfigure}
		\begin{subfigure}{0.45\textwidth}
			\centering
			\includegraphics[width=\textwidth]
				{diffusion/images/DvsTilt_1200K.pdf}
		\end{subfigure}
		\caption{
			Diffusivites against misorientation angle at 600 K and 1200 K.
		}
	\end{figure}

	\begin{itemize}
		\item U GB diffusivity is higher than that of Mo and Xe at 600 K.
			At 1200 K, U GB diffusivity is still higher than that of Mo.
			However, Xe diffusivity becomes larger than that of U.
		\item In general, the greater the misorientation angle,
			the greater the diffusivity.
	\end{itemize}
\end{frame}

\begin{frame}{Anisotropic nature of diffusion}
	\begin{columns}
	\column{0.5\textwidth}
		\begin{itemize}
			\item For some GBs like (190), diffusion is similar to
				dislocation pipe diffusion.
			\item A symmetric tilt low-angle GB is essnetially
				an array of parallel edge dislocations.
			\item Even at high temperatures,
				the anisotropic nature of diffusion remains.
			\item Isotropic behavior of (130) is shown for comparison.
		\end{itemize}
	\column{0.5\textwidth}
		\begin{figure}[ht]
			\centering
			\begin{subfigure}{0.48\textwidth}
				\centering
				\caption{}
				\includegraphics[width=\textwidth]
					{diffusion/images/130at700cs.png}
			\end{subfigure}
			\begin{subfigure}{0.48\textwidth}
				\centering
				\caption{}
				\includegraphics[width=\textwidth]
					{diffusion/images/130at1100cs.png}
			\end{subfigure}
			\begin{subfigure}{0.48\textwidth}
				\centering
				\caption{}
				\includegraphics[width=\textwidth]
					{diffusion/images/190at700cs.png}
			\end{subfigure}
			\begin{subfigure}{0.48\textwidth}
				\centering
				\caption{}
				\includegraphics[width=\textwidth]
					{diffusion/images/190at1100cs.png}
			\end{subfigure}
			\caption{
				Cross-sectional view of atomic trajectories at
				(a) 700 K and (b) 1100 K of (130),
				and (c) 700 K and (d) 1100 K of (190).
			}
		\end{figure}
	\end{columns}
\end{frame}

\begin{frame}{Anisotropic nature (cont.)}
	\begin{columns}
	\column{0.5\textwidth}
		\begin{itemize}
			\item To add to the visual example,
				mean squared displacements (MSDs) are shown here.
			\item In the isotropic case,
				MSDs parallel and perpendicular to the tilt axis
				are quite similar.
			\item In the anisotropic case, MSD perpendicular to the tilt axis
				is similar to MSD perpendicular to the GB plane,
				indicating an almost one dimensional diffusion.
		\end{itemize}
	\column{0.5\textwidth}
		\begin{figure}[ht]
			\centering
			\begin{subfigure}{0.95\textwidth}
				\centering
				\includegraphics[height=3cm]
					{diffusion/images/130at1100xyz.pdf}
			\end{subfigure}
			\begin{subfigure}{0.95\textwidth}
				\centering
				\includegraphics[height=3cm]
					{diffusion/images/190at1100xyz.pdf}
			\end{subfigure}
			\caption{
				Mean squared displacements of
				(Top) (130) and (Bottom) (190) at 1100 K.
			}
		\end{figure}
	\end{columns}
\end{frame}

% NOTE: From Beeler: this slide is not needed
\begin{frame}{Two diffusion regimes}
	\begin{figure}[ht]
		\centering
		\begin{subfigure}{0.45\textwidth}
			\centering
			\includegraphics[width=\textwidth]
				{diffusion/images/2reg_U_Dz.pdf}
		\end{subfigure}
		\begin{subfigure}{0.45\textwidth}
			\centering
			\includegraphics[width=\textwidth]
				{diffusion/images/2reg_Mo_Dz.pdf}
		\end{subfigure}
		\caption{
			Arrhenius fits to the two diffusion regimes.
		}
	\end{figure}

	\begin{itemize}
		\item GB diffusivities don't follow Arrhenius equation exactly.
		\item There's concavity, which indicates sub-Arrhenius behavior.
		\item Arrhenius fits are thus separated
			into two different diffusion regimes.
	\end{itemize}
\end{frame}

\section{Re-solution of Xe gas bubbles in U-10Mo}

\begin{frame}{Fission gas bubble re-solution}
	\begin{columns}
	\column{0.5\textwidth}
		\begin{itemize}
			\item Fission strongly limits bubble size in the fuel lattice
				by reintroducing the atoms in the bubble
				into the fuel matrix---a process known as re-solution.
			\item In homogeneous re-solution,
				individual gas atoms are ejected from the gas bubble
				through collisions with energetic fission fragments
				or recoil atoms (PKAs).
			\item In heterogeneous re-solution,
				bubbles are destroyed by the passage of fission tracks
				where local temperature can be higher than
				the fuel melting temperature.
		\end{itemize}
	\column{0.5\textwidth}
		\begin{figure}[ht]
			\centering
			\begin{subfigure}{0.6\textwidth}
				\centering
				\includegraphics[width=\textwidth]{sfigs/homogeneous.png}
			\end{subfigure}

			\begin{subfigure}{0.6\textwidth}
				\centering
				\includegraphics[width=\textwidth]{sfigs/heterogeneous.png}
			\end{subfigure}
			\caption{
				Illustration of
				(Top) homogeneous and (Bottom) heterogeneous re-solution.
			}
		\end{figure}
	\end{columns}
\end{frame}

\begin{frame}{Energy loss of fission fragments}
	\begin{columns}
	\column{0.5\textwidth}
		\begin{itemize}
			\item Binary collision approximation (BCA) simulations
				of two representative fission fragments (FFs) \Y and \I
				were performed using RustBCA.
		\end{itemize}
		\begin{align}
			\ce{
				_0^1n + _{92}^{235}U -> _{39}^{97}Y + _{53}^{136}I + Q
			}
		\end{align}
		\begin{itemize}
			\item Peak electronic stopping power is just shy of $20$ keV/nm
				for both FFs.
			\item Nuclear stopping power has a characteristic Bragg peak
				at the end of FF trajectories.
		\end{itemize}
	\column{0.5\textwidth}
		\begin{figure}[ht]
			\centering
			\begin{subfigure}{0.6\textwidth}
				\centering
				\includegraphics[width=\textwidth]{resol2/images/Y_stopping.pdf}
			\end{subfigure}

			\begin{subfigure}{0.6\textwidth}
				\centering
				\includegraphics[width=\textwidth]{resol2/images/I_stopping.pdf}
			\end{subfigure}
			\caption{
				Nuclear and electronic stopping powers of (Top) \Y and (Bottom) \I.
			}
		\end{figure}
	\end{columns}
\end{frame}

\begin{frame}{Heterogeneous re-solution}
	\begin{columns}
	\column{0.5\textwidth}
		\begin{itemize}
			\item MD simulations with the two-temperature model (TTM)
				were performed to simulate thermal spikes
				due to electronic stopping.
			\item No re-solution has been observed in any of the simulations.
			\item Since U-10Mo is a metallic system
				with high thermal conductivity,
				the energy in the electronic subsystem dissipates quickly.
		\end{itemize}
	\column{0.5\textwidth}
		\begin{figure}[ht]
			\centering
			\begin{subfigure}{0.49\textwidth}
				\centering
				\caption{}
				\includegraphics[width=\textwidth]{resol2/images/ttm1.png}
			\end{subfigure}
			\begin{subfigure}{0.49\textwidth}
				\centering
				\caption{}
				\includegraphics[width=\textwidth]{resol2/images/ttm2.png}
			\end{subfigure}
			\begin{subfigure}{0.49\textwidth}
				\centering
				\caption{}
				\includegraphics[width=\textwidth]{resol2/images/ttm3.png}
			\end{subfigure}
			\begin{subfigure}{0.49\textwidth}
				\centering
				\caption{}
				\includegraphics[width=\textwidth]{resol2/images/ttm4.png}
			\end{subfigure}
			\caption{
				Snapshots of a thermal spike ($30$ keV/nm) simulation at
				(a) $0$, (b) $2.4$, (c) $10.1$, and (d) $29.1$ ps.
				U, Mo and Xe atoms are shown in red, blue and black.
			}
		\end{figure}
	\end{columns}
\end{frame}

\begin{frame}{Model for homogeneous re-solution}
	\begin{columns}
	\column{0.5\textwidth}
		\begin{itemize}
			\item For homogeneous re-solution,
				FFs originating at different distances from a bubble
				need to be accounted for.
			\item A particular rotation of fission events around the bubble
				makes the calculation tenable.
		\end{itemize}
		\begin{align}
			b &= \sum_{k = Y, I} \int_V \xi_k(x, w) \dot{F} dV \\
			&= \dot{F} \sum_{k = Y, I} \int_{x=0}^{\infty} \int_{w=0}^{\infty}
				\xi_k(x, w) 2 \pi w dw dx
		\end{align}
	\column{0.5\textwidth}
		\begin{figure}[ht]
			\centering
			\begin{subfigure}{\textwidth}
				\centering
				\includegraphics[width=0.8\textwidth]{resol2/images/rotation.pdf}
			\end{subfigure}

			\begin{subfigure}{\textwidth}
				\centering
				\includegraphics[width=0.5\textwidth]{resol2/images/coord.pdf}
			\end{subfigure}
			\caption{
				(Top) Rotation of fission events around the origin.
				(Bottom) A coordinate system with a bubble at the origin
				and all FFs pointing to the $-x$ direction.
			}
		\end{figure}
	\end{columns}
\end{frame}

\begin{frame}{Reference fission fragment simulations}
	\begin{columns}
	\column{0.5\textwidth}
		\begin{itemize}
			\item The brute-force approach is computationally prohibitive.
			\item If the probability of a FF going through a certain point
				with a certain energy is known,
				only local re-solution behavior needs to be simulated.
			\item RustBCA was used to perform BCA simulations of ions in U-10Mo.
			\item A discretization scheme is used to gather FF information.
		\end{itemize}
	\column{0.5\textwidth}
		\begin{figure}[ht]
			\centering
			\begin{subfigure}{\textwidth}
				\centering
				\includegraphics[width=0.6\textwidth]{resol2/images/ff_track.png}
			\end{subfigure}

			\begin{subfigure}{\textwidth}
				\centering
				\includegraphics[width=0.5\textwidth]{resol2/images/surf_grid.pdf}
			\end{subfigure}
			\caption{
				(Top) Trajectories of $100$ \Y ions in U-10Mo.
				(Bottom) Surface discretization
				of fission fragment information across volume.
			}
		\end{figure}
	\end{columns}
\end{frame}

\begin{frame}{Ion profiles}
	\begin{figure}[ht]
		\centering
		\begin{subfigure}{0.32\textwidth}
			\centering
			\includegraphics
				[width=\textwidth, trim={0.8cm 0 1.5cm 0.4cm}, clip]
				{resol2/images/Y_p.pdf}
		\end{subfigure}
		\begin{subfigure}{0.32\textwidth}
			\centering
			\includegraphics
				[width=\textwidth, trim={0.8cm 0 1.5cm 0.4cm}, clip]
				{resol2/images/Y_e.pdf}
		\end{subfigure}
		\begin{subfigure}{0.32\textwidth}
			\centering
			\includegraphics
				[width=\textwidth, trim={0.8cm 0 1.5cm 0.4cm}, clip]
				{resol2/images/Y_a.pdf}
		\end{subfigure}
		\begin{subfigure}{0.32\textwidth}
			\centering
			\includegraphics
				[width=\textwidth, trim={0.8cm 0 1.5cm 0.4cm}, clip]
				{resol2/images/I_p.pdf}
		\end{subfigure}
		\begin{subfigure}{0.32\textwidth}
			\centering
			\includegraphics
				[width=\textwidth, trim={0.8cm 0 1.5cm 0.4cm}, clip]
				{resol2/images/I_e.pdf}
		\end{subfigure}
		\begin{subfigure}{0.32\textwidth}
			\centering
			\includegraphics
				[width=\textwidth, trim={0.8cm 0 1.5cm 0.4cm}, clip]
				{resol2/images/I_a.pdf}
		\end{subfigure}
		\caption{
			(Top, left to right) \Y ion
			incidence probability per unit surface area, energy, and angle.
			(Bottom, left to right) \I ion
			incidence probability per unit surface area, energy, and angle.
		}
	\end{figure}
\end{frame}

\begin{frame}{Fission fragment and bubble interactions}
	\begin{columns}
	\column{0.5\textwidth}
		\begin{itemize}
			\item A specific 3D geometry has been implemented in RustBCA
				to allow the simulation of gas bubbles in solid materials.
			\item A constant mean-free-path was used for solid U-10Mo,
				and an exponentially distributed mean-free-path
				was used for the Xe gas bubble.
			\item Xe recoils were analyzed
				to count the number of re-solved Xe atoms.
		\end{itemize}
	\column{0.5\textwidth}
		\begin{figure}[ht]
			\centering
			\begin{subfigure}{\textwidth}
				\centering
				\includegraphics[width=0.6\textwidth]{resol2/images/ff_bubble.png}
			\end{subfigure}

			\begin{subfigure}{0.49\textwidth}
				\centering
				\includegraphics[width=\textwidth]{resol2/images/xe_dr.pdf}
			\end{subfigure}
			\begin{subfigure}{0.49\textwidth}
				\centering
				\includegraphics[width=\textwidth]{resol2/images/xe_hist.pdf}
			\end{subfigure}
			\caption{
				(Top) BCA simulation of a $64$ nm radius bubble
				with a $5$ MeV \Y ion.
				(Bottom) Xe recoil distribution.
			}
		\end{figure}
	\end{columns}
\end{frame}

\begin{frame}{Local re-solution behavior}
	\begin{columns}
	\column{0.5\textwidth}
		\begin{itemize}
			\item In BCA simulations, ion energy ($E$)
				and off-centered distance ($\ell$) were varied.
			\item $5,000$ simulations were performed for each configuration
				with the \texttt{SPHEREINCUBOID} geometry in RustBCA.
			\item The results are denoted by $\chi(E, \ell)$.
			\item Interpolation can be used to obtain local FF-bubble behavior
				for any energy and off-centered distance.
		\end{itemize}
	\column{0.5\textwidth}
		\begin{figure}[ht]
			\centering
			\begin{subfigure}{0.49\textwidth}
				\centering
				\includegraphics[width=\textwidth]{resol2/images/chi_2nm_Y.pdf}
			\end{subfigure}
			\begin{subfigure}{0.49\textwidth}
				\centering
				\includegraphics[width=\textwidth]{resol2/images/chi_2nm_I.pdf}
			\end{subfigure}
			\begin{subfigure}{0.49\textwidth}
				\centering
				\includegraphics[width=\textwidth]{resol2/images/chi_64nm_Y.pdf}
			\end{subfigure}
			\begin{subfigure}{0.49\textwidth}
				\centering
				\includegraphics[width=\textwidth]{resol2/images/chi_64nm_I.pdf}
			\end{subfigure}
			\caption{
				Re-solved bubble fraction
				as a function of energy and off-centered distance.
				Error bars show $2\sigma$ deviations.
			}
		\end{figure}
	\end{columns}
\end{frame}

\begin{frame}{Re-solution due to any fission fragment}
	\begin{columns}
	\column{0.5\textwidth}
		\begin{itemize}
			\item With the ion profiles and local re-solution behavior,
				it is possible to calculate the re-solved bubble fraction
				due to a FF originating at an arbitrary $(x, w)$.
			\item Probability, energy and angle are combined with
				a probabilistic interpretation of a FF-bubble interaction.
		\end{itemize}
		{\scriptsize
			\begin{align}
				\xi(x, w) &= \sum_{m \in S}
					p(r_m) \frac{A_m}{\cos \alpha(x, w)}
					\chi(E(r_m), ||r_m - r_c||)
			\end{align}
		}
	\column{0.5\textwidth}
		\begin{figure}[ht]
			\centering
			\includegraphics[width=\textwidth]{resol2/images/surf_mesh.pdf}
			\caption{
				Illustration of an arbitrary
				fission fragment and bubble interaction.
			}
		\end{figure}
	\end{columns}
\end{frame}

\begin{frame}{Re-solution profiles}
	\begin{columns}
	\column{0.5\textwidth}
		\begin{itemize}
			\item Re-solution of a bubble at $(x, w)$ due to a FF at the origin
				can be calculated for all $(x, w)$.
			\item If the positions of the bubble and the FF are swapped,
				we would get the same $\xi$ profiles.
			\item The re-solution rate can then be calculated
				using a discretized version of the model we described before.
		\end{itemize}
		{\scriptsize
			\begin{align}
				b / \dot{F}
					&= \sum_{k=Y,I} \sum \xi_k \Delta V
			\end{align}
		}
	\column{0.5\textwidth}
		\begin{figure}
			\centering
			\begin{subfigure}{0.9\textwidth}
				\centering
				\includegraphics
					[width=\textwidth, trim={0.8cm 0 1.4cm 0.7cm}, clip]
					{resol2/images/64nm_Y_xi.pdf}
			\end{subfigure}
			\begin{subfigure}{0.9\textwidth}
				\centering
				\includegraphics
					[width=\textwidth, trim={0.8cm 0 1.4cm 0.7cm}, clip]
					{resol2/images/64nm_Y_db.pdf}
			\end{subfigure}
			\caption{
				(Top) $\xi$ and (Bottom) $\xi \Delta V$
				for \Y incident on a bubble of radius $64$ nm.
			}
		\end{figure}
	\end{columns}
\end{frame}

\begin{frame}{Homogeneous re-solution rate}
	\begin{columns}
	\column{0.5\textwidth}
		\begin{itemize}
			\item Summing up $\xi \Delta V$ for both \Y and \I
				provide the overall re-solution rate.
			\item The probability of an interaction between a FF
				and a larger bubble is higher.
		\end{itemize}
		\begin{align}
			b / \dot{F} &= a R_b^k + c \\
			a &= \num{8.43e-25} \\
			k &= -0.926 \\
			c &= \num{3.46e-26}
		\end{align}
	\column{0.5\textwidth}
		\begin{figure}
			\centering
			\includegraphics[width=\textwidth]{resol2/images/bhom.pdf}
			\caption{
				Homogeneous re-solution rate at equilibrium Xe number density.
			}
		\end{figure}
	\end{columns}
\end{frame}

\begin{frame}{Effect of bubble pressure}
	\begin{columns}
	\column{0.5\textwidth}
		\begin{itemize}
			\item The number of re-solved Xe atoms is almost an invariant
				with respect to the Xe number density in the bubble.
			\item $\chi$ and $n$ are thus inversely proportional:
				$\chi/\chi_{eq} = n_{eq}/n$.
			\item Pressure effects can be incorporated as follows.
		\end{itemize}
		\begin{align}
			b &\propto \xi \propto \chi \\
			b / b_{eq} &= n_{eq} / n \\
			b &= \left( a R_b^k + c \right)
				\left( \frac{n_{eq}}{n} \right) \dot{F}
		\end{align}
	\column{0.5\textwidth}
		\begin{figure}
			\centering
			\includegraphics[width=\textwidth]{resol2/images/pressure.pdf}
			\caption{
				Effect of Xe number density on homogeneous re-solution rate.
			}
		\end{figure}
	\end{columns}
\end{frame}

\begin{frame}{Comparison with other fuels}
	\begin{columns}
	\column{0.5\textwidth}
		\begin{itemize}
			\item The re-solution in UO$_2$ is dominated
				by the heterogeneous mechanism
				and the re-solution in UC happens entirely through
				the homogeneous mechanism.
			\item The re-solution rates in UO$_2$, UC and U-10Mo
				are within one order of magnitude.
			\item The re-solution rate used in DART is higher for smaller bubbles
				and much lower for larger bubbles than the rate we calculated.
		\end{itemize}
	\column{0.5\textwidth}
		\begin{figure}
			\centering
			\includegraphics[width=\textwidth]{resol2/images/comp.pdf}
			\caption{
				Comparison of re-solution rates of different nuclear fuels.
			}
		\end{figure}
	\end{columns}
\end{frame}

\section{IUQ of U-10Mo fission-gas-behavior parameters}

\begin{frame}{Quantifying unknown DART parameters}
	\begin{itemize}
		\item The key challenge in simulating fission gas swelling
			with a mechanistic model is
			obtaining key material properties related to gas bubble behavior,
			as many of them cannot be measured experimentally.
		\item For the parameters that do not have measurement data
			or atomic-scale simulation results,
			they are usually estimated by either fitting to measured bubble morphology
			or by borrowing from other similar fuel systems where the data is available.
		\item The fission-gas-behavior parameters used in the GRASS module of DART
			were calibrated in a previous study,
			using the bubble size distributions
			measured from irradiated U-10Mo dispersion fuel particles.
		\item However, this set of parameters needs recalibration
			because new atomic-scale data have become available
			since the previous calibration.
	\end{itemize}
\end{frame}

\begin{frame}{Inverse uncertainty quantification (IUQ)}
	\begin{columns}
	\column{0.5\textwidth}
		\begin{itemize}
			\item IUQ is a process to quantify uncertainties
				in the input parameters of a computer model
				given experimental data.
			\item It's the opposite of
				forward uncertainty quantification (FUQ),
				which quantifies the uncertainty in the output given the input.
			\item IUQ quantifies the uncertainty in the model itself,
				and thus can help to improve
				the accuracy of the model predictions.
			\item It's helpful in situations where the underlying parameters
				of the model are unknown.
		\end{itemize}
	\column{0.5\textwidth}
		\begin{figure}[ht]
			\centering
			\includegraphics[width=6cm]{sfigs/wu_iuq.png}
			\caption{
				Some essential parts of modeling and simulation
				(Wu et al. {\color{blue}
				\url{https://doi.org/10.1016/j.nucengdes.2018.06.004}}).
			}
		\end{figure}
	\end{columns}
\end{frame}

\begin{frame}{asdf}
	\begin{columns}
	\column{0.5\textwidth}
		\begin{itemize}
			\item asdf
		\end{itemize}
	\column{0.5\textwidth}
		\begin{figure}
			\centering
			\includegraphics[width=0.7\textwidth]{iuq2/images/mvn_expt.pdf}
			\caption{
				asdf
			}
		\end{figure}
	\end{columns}
\end{frame}

% FIX: update table
\begin{frame}{Fission-gas-behavior parameters}
	\begin{table}[ht]
	\centering
	\caption{
		Fission gas behavior parameters and their ranges.
	}
	\begin{tabular}{lccc}
	\toprule
	Parameter    & Minimum value & Maximum value & Reference value \\
	\midrule
	dGrainHBS    & \num{2.4e-5 } & \num{5.6e-5 } & \num{4e-5   }   \\
	FaceCovMax   & \num{0.6    } & \num{0.907  } & \num{0.907  }   \\
	SwellLink    & \num{0.016  } & \num{0.034  } & \num{0.025  }   \\
	vResol       & \num{1.3e-18} & \num{2.7e-18} & \num{2e-18  }   \\
	DatomFissGBx & \num{19641  } & \num{40530  } & \num{3e4    }   \\
	fNucleate    & \num{3.5e-10} & \num{8.3e-10} & \num{6e-10  }   \\
	aAtomDifFiss & \num{3.2e-31} & \num{7.2e-31} & \num{5.1e-31}   \\
	\bottomrule
	\end{tabular}
	\end{table}

	\begin{itemize}
		\item DART is run with 7 fission-gas-behavior parameters perturbed
			to compile a dataset necessary for IUQ.
		\item 3200 samples were generated at different fission density values.
		\item The list above contains the names of the parameters
			we are interested in, and their probed ranges.
	\end{itemize}
\end{frame}

\begin{frame}{asdf}
	\label{frame:scatter_lo}
	\begin{columns}
	\column{0.5\textwidth}
		\begin{itemize}
			\item asdf
		\end{itemize}
	\column{0.5\textwidth}
		\begin{figure}
			\centering
			\includegraphics[width=\textwidth]{iuq2/images/scatter_lo.pdf}
			\caption{
				asdf
			}
		\end{figure}
	\end{columns}
\end{frame}

\begin{frame}{asdf}
	\begin{columns}
	\column{0.5\textwidth}
		\begin{itemize}
			\item asdf
		\end{itemize}
	\column{0.5\textwidth}
		\begin{figure}
			\centering
			\includegraphics[width=\textwidth]{iuq2/images/scatter_hi.pdf}
			\caption{
				asdf
			}
		\end{figure}
	\end{columns}
\end{frame}

\begin{frame}{Surrogate modeling}
	\begin{columns}
	\column{0.5\textwidth}
		\begin{itemize}
			\item We need to perturb the parameter space
				to match the observation.
			\item But it's not feasible to run calculations
				for all possible parameter combinations
				using DART.
			\item A surrogate model
				(also called a response surface or an emulator)
				is thus needed.
			\item Many machine learning models,
				such as Gaussian processes and neural networks,
				can act as a surrogate model.
		\end{itemize}
	\column{0.5\textwidth}
		\begin{figure}[ht]
			\centering
			\includegraphics[width=\textwidth]{sfigs/surrogate_modeling.png}
			\caption{
				The principle of surrogate modeling
				(Kocijan et al.
				{\color{blue}\url{https://doi.org/10.5516/NET.07.2014.706}}).
			}
		\end{figure}
	\end{columns}
\end{frame}

\begin{frame}{asdf}
	\begin{columns}
	\column{0.5\textwidth}
		\begin{itemize}
			\item asdf
		\end{itemize}
	\column{0.5\textwidth}
		\begin{figure}[ht]
			\centering
			\begin{subfigure}{0.49\textwidth}
				\centering
				\includegraphics[width=\textwidth]{iuq2/images/gp_lo.pdf}
			\end{subfigure}
			\begin{subfigure}{0.49\textwidth}
				\centering
				\includegraphics[width=\textwidth]{iuq2/images/gp_hi.pdf}
			\end{subfigure}
			\begin{subfigure}{0.49\textwidth}
				\centering
				\includegraphics[width=\textwidth]{iuq2/images/nn_lo.pdf}
			\end{subfigure}
			\begin{subfigure}{0.49\textwidth}
				\centering
				\includegraphics[width=\textwidth]{iuq2/images/nn_hi.pdf}
			\end{subfigure}
			\caption{
				asdf
			}
		\end{figure}
	\end{columns}
\end{frame}

\begin{frame}{asdf}
	\begin{columns}
	\column{0.5\textwidth}
		\begin{itemize}
			\item asdf
		\end{itemize}
	\column{0.5\textwidth}
		\begin{figure}[ht]
			\centering
			\begin{subfigure}{\textwidth}
				\centering
				\includegraphics[width=0.8\textwidth]{iuq2/images/sobol_lo.pdf}
			\end{subfigure}
			\begin{subfigure}{\textwidth}
				\centering
				\includegraphics[width=0.8\textwidth]{iuq2/images/sobol_hi.pdf}
			\end{subfigure}
			\caption{
				asdf
			}
		\end{figure}
	\end{columns}
\end{frame}

% NOTE: From Beeler: did you define MCMC?
\begin{frame}{asdf}
	\begin{columns}
	\column{0.5\textwidth}
		\begin{itemize}
			\item asdf
		\end{itemize}
	\column{0.5\textwidth}
		\begin{figure}[ht]
			\centering
			\includegraphics[width=\textwidth]{iuq2/images/mcmc_trace.png}
			\caption{
				asdf
			}
		\end{figure}
	\end{columns}
\end{frame}

\begin{frame}{asdf}
	\begin{columns}
	\column{0.5\textwidth}
		\begin{itemize}
			\item asdf
		\end{itemize}
	\column{0.5\textwidth}
		\begin{figure}[ht]
			\centering
			\includegraphics[width=\textwidth]{iuq2/images/iuq.png}
			\caption{
				asdf
			}
		\end{figure}
	\end{columns}
\end{frame}

\begin{frame}{asdf}
	\begin{columns}
	\column{0.5\textwidth}
		\begin{itemize}
			\item asdf
			\item Slide \ref{frame:scatter_lo}
		\end{itemize}
	\column{0.5\textwidth}
		\begin{figure}
			\begin{subfigure}{\textwidth}
				\centering
				\includegraphics[width=0.7\textwidth]{iuq2/images/fuq_mvn.pdf}
			\end{subfigure}
			\begin{subfigure}{0.49\textwidth}
				\centering
				\includegraphics[width=\textwidth]{iuq2/images/fuq_lo.pdf}
			\end{subfigure}
			\begin{subfigure}{0.49\textwidth}
				\centering
				\includegraphics[width=\textwidth]{iuq2/images/fuq_hi.pdf}
			\end{subfigure}
			\caption{
				asdf
			}
		\end{figure}
	\end{columns}
\end{frame}

\begin{frame}{asdf}
	\begin{columns}
	\column{0.5\textwidth}
		\begin{itemize}
			\item asdf
		\end{itemize}
	\column{0.5\textwidth}
		\begin{figure}[ht]
			\centering
			\includegraphics[width=\textwidth]{iuq2/images/aftermath.pdf}
			\caption{
				asdf
			}
		\end{figure}
	\end{columns}
\end{frame}

\section{Conclusions}

\begin{frame}{Conclusions}
	With the application of various computational methods,
	we quantified crucial material properties of U-Mo fuel.
	This will assist the higher-length-scale models
	in making more robust predictions.
	Overall, this dissertation achieves the following:
	\begin{itemize}
		\item Fully quantifies GB diffusion coefficients of U, Mo, and Xe
			in $\gamma$U-Mo for various composition, temperatures,
			and GB orientations.
			The findings also provide insight into GB anisotropy.
		\item Provides a mathematical model for Xe re-solution rate,
			taking into account both bubble size and pressure.
			This work considered
			both homogeneous and heterogeneous re-solution rate.
		\item Performs IUQ on still unknown fission-gas-behavior parameters.
			Three influential parameters are identified
			and subsequently optimized for better meso-scale simulation
			of U-Mo fuel.
	\end{itemize}
\end{frame}

\begin{frame}{Future work}
	\begin{itemize}
		\item Further investigation is needed to evaluate
			the relative dominance of homogeneous and heterogeneous mechanisms.
			The two-temperature model will be employed to quantify
			the amount of energy transferred
			from the electronic subsystem to the ionic subsystem.
		\item The IUQ process will be expanded to incorporate new data,
			which takes into account variations in fission rate and grain size.
			Surrogate models will be developed for new data,
			enabling a refined IUQ to better quantify parameter distributions
			pertient to a wide range of reactor conditions.
		\item Machine-learned interatomic potentials describing U, Mo
			and other elements, such as Al and Zr,
			will be developed and evaluated.
			Successful implementation of this will enable MD simulations
			of U-Mo fuel with liners and/or claddings.
	\end{itemize}
\end{frame}

\section*{Thanks}

\begin{frame}
	\centering \Large
	\emph{Thank you!}
\end{frame}

\end{document}

% NOTE: From Beeler: is this slide really necessary?

% \begin{frame}{Bayesian framework for IUQ}
% 	\begin{columns}
% 	\column{0.4\textwidth}
% 		\begin{itemize}
% 			\item $y^E$ is the observation
% 				and $y^R$ is the reality.
% 				$y^M$ is the model with a bias $\delta(x)$.
% 			\item $\epsilon$ is the measurement error
% 				and assumed to be Gaussian.
% 			\item To get posterior distributions of parameters $\theta^*$,
% 				Bayes' theorem can be used.
% 			\item Since the error is related to $y^E$ and $y^M$,
% 				the likelihood can be described in terms of the error.
% 		\end{itemize}
% 	\column{0.6\textwidth}
% 		\begin{align}
% 			\bm{y}^R(\bm{x})
% 			&= \bm{y}^M(\bm{x}, \bm{\theta}^*) + \delta(\bm{x}) \\
% 			\bm{y}^E(\bm{x})
% 			&= \bm{y}^R(\bm{x}) + \bm{\epsilon} \\
% 			\bm{y}^E(\bm{x})
% 			&= \bm{y}^M(\bm{x}, \bm{\theta}^*)
% 				+ \delta(\bm{x}) + \bm{\epsilon} \\
% 			\bm{\epsilon}
% 			&= \bm{y}^E(\bm{x}) - \bm{y}^M(\bm{x}, \bm{\theta}^*)
% 				- \delta(\bm{x}) \\
% 			\bm{\epsilon}
% 			&\sim \mathcal{N} (\bm{\mu}, \bm{\Sigma}_{exp}) \\
% 			p(\bm{\theta}^* | \bm{y}^E, \bm{y}^M)
% 			&\propto p(\bm{\theta}^*)
% 				\cdot p(\bm{y}^E, \bm{y}^M | \bm{\theta}^*) \\
% 			\propto p(\bm{\theta}^*)
% 				&\cdot \frac{1}{\sqrt{|\bm{\Sigma}|}}
% 				\exp \bigg[-\frac{1}{2} \bm{\epsilon}^T
% 				\bm{\Sigma}^{-1} \bm{\epsilon} \bigg] \\
% 			\bm{\Sigma}
% 			&= \bm{\Sigma}_{exp}
% 				+ \bm{\Sigma}_{bias} + \bm{\Sigma}_{code}
% 		\end{align}
% 	\end{columns}
% \end{frame}
