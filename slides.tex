\documentclass[10pt]{beamer}
\usepackage{booktabs, subcaption, multirow}
\usepackage{amsmath, amssymb, bm, siunitx}

\usetheme{Boadilla}
\usecolortheme{seahorse}
%\usefonttheme{serif}
\setbeamertemplate{caption}[numbered]
\setbeamerfont{caption}{size=\scriptsize}

\title[Computational methods for $\gamma$U-Mo]
{Application of computational methods for the evaluation of
\texorpdfstring{$\gamma$}{gamma}U-Mo monolithic fuel}
\author{ATM Jahid Hasan}
\institute[NCSU]{Department of Nuclear Engineering\\
North Carolina State University}
\date{April 4, 2025}

\begin{document}
\frame{\titlepage}

\begin{frame}{Table of Contents}
	\tableofcontents
\end{frame}

\section{Introduction}

\begin{frame}{Introduction}
	\begin{columns}
	\column{0.5\textwidth}
		\begin{itemize}
			\item Put empty slides with a certain template.
			\item Populate with all the figures after that.
		\end{itemize}
	\column{0.5\textwidth}
		\begin{figure}[ht]
			\centering
			\includegraphics[width=5cm]{sfigs/historic_fuels.png}
			\caption{
				asdf
			}
			% \label{fig:design}
		\end{figure}
	\end{columns}
\end{frame}

\begin{frame}{Introduction}
	\begin{columns}
	\column{0.5\textwidth}
		\begin{itemize}
			\item Put empty slides with a certain template.
			\item Populate with all the figures after that.
		\end{itemize}
	\column{0.5\textwidth}
		\begin{figure}[ht]
			\centering
			\includegraphics[width=5cm]{sfigs/umo_design.jpg}
			\caption{
				Depiction of monolithic fuel cross-section
				(Meyer et al.
				{\color{blue}\url{https://doi.org/10.5516/NET.07.2014.706}}).
			}
			\label{fig:design}
		\end{figure}
		\begin{itemize}
			\item \textbf{Goals:}
				\begin{itemize}
					\item This is a good style!
				\end{itemize}
		\end{itemize}
	\end{columns}
\end{frame}

\begin{frame}{Swelling}
	\begin{columns}
	\column{0.5\textwidth}
		\begin{itemize}
			\item asdf
		\end{itemize}
	\column{0.5\textwidth}
		\begin{figure}[ht]
			\centering
			\includegraphics[width=4cm]{sfigs/umo_burnup.png}
			\caption{
				asdf
			}
			% \label{fig:design}
		\end{figure}
	\end{columns}
\end{frame}

\begin{frame}{Dispersion Analysis Research Tool}
	\begin{columns}
	\column{0.5\textwidth}
		\begin{itemize}
			\item asdf
		\end{itemize}
	\column{0.5\textwidth}
		\begin{figure}[ht]
			\centering
			\includegraphics[width=6cm]{sfigs/dart_schematic.png}
			\caption{
				asdf
			}
			% \label{fig:design}
		\end{figure}
	\end{columns}
\end{frame}

\section{Grain boundary diffusion in \texorpdfstring{$\gamma$}{gamma}U-Mo}

\begin{frame}{GB diffusion}
	\begin{columns}
	\column{0.5\textwidth}
		\begin{itemize}
			\item Item 1
			\item Item 2
			\item Item 3
		\end{itemize}
	\column{0.5\textwidth}
		\begin{figure}[ht]
			\centering
			\includegraphics[width=5cm]{example-image-a}
			\caption{
				An example caption!
			}
			\label{fig:diffusion}
		\end{figure}
	\end{columns}
\end{frame}

\begin{frame}{Simulation setup}
	\begin{columns}
	\column{0.5\textwidth}
		\begin{itemize}
			\item Item 1
			\item Item 2
			\item Item 3
		\end{itemize}
	\column{0.5\textwidth}
		\begin{figure}[ht]
			\centering
			\begin{subfigure}{1.0\textwidth}
				\centering
				% \caption{}
				\includegraphics[width=\textwidth]
					{diffusion/images/configuration.png}
			\end{subfigure}

			\begin{subfigure}{0.7\textwidth}
				\centering
				% \caption{}
				\includegraphics[width=0.7\textwidth]
					{diffusion/images/gb_def.png}
			\end{subfigure}
			\caption{
				asf
			}
			% \label{fig:schematic}
		\end{figure}
	\end{columns}
\end{frame}

\begin{frame}{GB validation}
	\begin{columns}
	\column{0.5\textwidth}
		\begin{itemize}
			\item Item 1
			\item Item 2
			\item Item 3
		\end{itemize}
	\column{0.5\textwidth}
		\begin{figure}[ht]
			\centering
			\includegraphics[width=6cm]{diffusion/images/gbe.pdf}
			\caption{
				An example caption!
			}
			% \label{fig:template}
		\end{figure}
	\end{columns}
\end{frame}

\begin{frame}{GB width}
	\begin{columns}
	\column{0.5\textwidth}
		\begin{itemize}
			\item Item 1
			\item Item 2
			\item Item 3
		\end{itemize}
	\column{0.5\textwidth}
		\begin{figure}[ht]
			\centering
			\includegraphics[width=6cm]{diffusion/images/d_gb_sym.pdf}
			\caption{
				An example caption!
			}
			% \label{fig:template}
		\end{figure}
	\end{columns}
\end{frame}

\begin{frame}{GB diffusivities}
	\begin{columns}
	\column{0.5\textwidth}
		\begin{itemize}
			\item Item 1
			\item Item 2
			\item Item 3
		\end{itemize}
	\column{0.5\textwidth}
		\begin{figure}[ht]
			\centering
			\includegraphics[width=6cm]{diffusion/images/u10mo_U_Dz.pdf}
			\caption{
				An example caption!
			}
			% \label{fig:template}
		\end{figure}
	\end{columns}
\end{frame}

\begin{frame}{Symmetric tilt GB diffusivities}
	\begin{figure}[ht]
		\centering
		\begin{subfigure}{0.325\textwidth}
			\centering
			\caption{}
			\includegraphics[width=\textwidth]
				{diffusion/images/u10mo_U_Dz.pdf}
		\end{subfigure}
		\begin{subfigure}{0.325\textwidth}
			\centering
			\caption{}
			\includegraphics[width=\textwidth]
				{diffusion/images/u10mo_Mo_Dz.pdf}
		\end{subfigure}
		\begin{subfigure}{0.325\textwidth}
			\centering
			\caption{}
			\includegraphics[width=\textwidth]
				{diffusion/images/u10mo_Xe_Dz.pdf}
		\end{subfigure}
		\caption{
			asf
		}
		% \label{fig:schematic}
	\end{figure}
\end{frame}

\begin{frame}{Asymmetric tilt \& Twist GB diffusivities}
	\begin{figure}[ht]
		\centering
		\begin{subfigure}{0.45\textwidth}
			\centering
			\caption{}
			\includegraphics[width=\textwidth]
				{diffusion/images/asym_twist_U_Dz.pdf}
		\end{subfigure}
		\begin{subfigure}{0.45\textwidth}
			\centering
			\caption{}
			\includegraphics[width=\textwidth]
				{diffusion/images/asym_twist_Mo_Dz.pdf}
		\end{subfigure}
		\caption{
			asf
		}
		% \label{fig:schematic}
	\end{figure}
\end{frame}

\begin{frame}{Orientation-averaged GB diffusivities}
	\begin{figure}[ht]
		\centering
		\begin{subfigure}{0.45\textwidth}
			\centering
			\caption{}
			\includegraphics[width=\textwidth]
				{diffusion/images/comp_U_Dz.pdf}
		\end{subfigure}
		\begin{subfigure}{0.45\textwidth}
			\centering
			\caption{}
			\includegraphics[width=\textwidth]
				{diffusion/images/comp_Mo_Dz.pdf}
		\end{subfigure}
		\caption{
			asf
		}
		% \label{fig:schematic}
	\end{figure}
\end{frame}

\begin{frame}{Comparison with literature}
	\begin{columns}
	\column{0.5\textwidth}
		\begin{itemize}
			\item asdf
		\end{itemize}
	\column{0.5\textwidth}
		\begin{figure}[ht]
			\centering
			\includegraphics[width=0.9\textwidth]
				{diffusion/images/newLitComp.pdf}
			\caption{
				ssdfw
			}
			% \label{fig:NvsRad}
		\end{figure}
	\end{columns}
\end{frame}

\begin{frame}{Effect of the misorientation angle}
	\begin{figure}[ht]
		\centering
		\begin{subfigure}{0.45\textwidth}
			\centering
			\caption{}
			\includegraphics[width=\textwidth]
				{diffusion/images/DvsTilt_600K.pdf}
		\end{subfigure}
		\begin{subfigure}{0.45\textwidth}
			\centering
			\caption{}
			\includegraphics[width=\textwidth]
				{diffusion/images/DvsTilt_1200K.pdf}
		\end{subfigure}
		\caption{
			asf
		}
		% \label{fig:schematic}
	\end{figure}
\end{frame}

\begin{frame}{GB diffusivity vs energy}
	\begin{figure}[ht]
		\centering
		\begin{subfigure}{0.45\textwidth}
			\centering
			\caption{}
			\includegraphics[width=\textwidth]
				{diffusion/images/DvsGBE_600K.pdf}
		\end{subfigure}
		\begin{subfigure}{0.45\textwidth}
			\centering
			\caption{}
			\includegraphics[width=\textwidth]
				{diffusion/images/DvsGBE_1200K.pdf}
		\end{subfigure}
		\caption{
			asf
		}
		% \label{fig:schematic}
	\end{figure}
\end{frame}

\begin{frame}{Anisotropic nature of diffusion}
	\begin{figure}[ht]
		\centering
		\begin{subfigure}{0.325\textwidth}
			\centering
			\caption{}
			\includegraphics[width=\textwidth]
				{diffusion/images/ratio_U.pdf}
		\end{subfigure}
		\begin{subfigure}{0.325\textwidth}
			\centering
			\caption{}
			\includegraphics[width=\textwidth]
				{diffusion/images/ratio_Mo.pdf}
		\end{subfigure}
		\begin{subfigure}{0.325\textwidth}
			\centering
			\caption{}
			\includegraphics[width=\textwidth]
				{diffusion/images/ratio_Xe.pdf}
		\end{subfigure}
		\caption{
			asf
		}
		% \label{fig:schematic}
	\end{figure}
\end{frame}

\begin{frame}{Anisotropy}
	\begin{columns}
	\column{0.5\textwidth}
		\begin{itemize}
			\item asdf
		\end{itemize}
	\column{0.5\textwidth}
		\begin{figure}[ht]
			\centering
			\begin{subfigure}{0.48\textwidth}
				\centering
				\caption{}
				\includegraphics[width=\textwidth]
					{diffusion/images/130at700cs.png}
			\end{subfigure}
			\begin{subfigure}{0.48\textwidth}
				\centering
				\caption{}
				\includegraphics[width=\textwidth]
					{diffusion/images/130at1100cs.png}
			\end{subfigure}
			\begin{subfigure}{0.48\textwidth}
				\centering
				\caption{}
				\includegraphics[width=\textwidth]
					{diffusion/images/190at700cs.png}
			\end{subfigure}
			\begin{subfigure}{0.48\textwidth}
				\centering
				\caption{}
				\includegraphics[width=\textwidth]
					{diffusion/images/190at1100cs.png}
			\end{subfigure}
			\caption{
				asf
			}
			% \label{fig:schematic}
		\end{figure}
	\end{columns}
\end{frame}

\begin{frame}{Anisotropy}
	\begin{columns}
	\column{0.5\textwidth}
		\begin{itemize}
			\item asdf
		\end{itemize}
	\column{0.5\textwidth}
		\begin{figure}[ht]
			\centering
			\begin{subfigure}{0.95\textwidth}
				\centering
				\includegraphics[height=3cm]
					{diffusion/images/130at1100xyz.pdf}
			\end{subfigure}
			\begin{subfigure}{0.95\textwidth}
				\centering
				\includegraphics[height=3cm]
					{diffusion/images/190at1100xyz.pdf}
			\end{subfigure}
			\caption{
				asf
			}
			% \label{fig:schematic}
		\end{figure}
	\end{columns}
\end{frame}

\begin{frame}{Two diffusion regimes}
	\begin{figure}[ht]
		\centering
		\begin{subfigure}{0.45\textwidth}
			\centering
			\caption{}
			\includegraphics[width=\textwidth]
				{diffusion/images/2reg_U_Dz.pdf}
		\end{subfigure}
		\begin{subfigure}{0.45\textwidth}
			\centering
			\caption{}
			\includegraphics[width=\textwidth]
				{diffusion/images/2reg_Mo_Dz.pdf}
		\end{subfigure}
		\caption{
			asf
		}
		% \label{fig:schematic}
	\end{figure}
\end{frame}

\begin{frame}{Two diffusion regimes}
	\begin{table}[!ht]
	\centering
	\caption{
		Prefactors and activation energies for the Arrhenius equation fits
		to $\overline{D}^{i}_{\parallel}$ ($i=$ U or Mo)
		in $\gamma$U-7Mo, $\gamma$U-10Mo, and $\gamma$U-12Mo
		for the two different temperature regimes.
		$D_0$ is in m$^2$s$^{-1}$ and $E_a$ is in eV.
	}
	\label{tab:2reg}
	\begin{tabular}{ccllll}
	\toprule
	Composition & Temp. (K)
		& $D_{0}^U$      & $E_{a}^U$
		& $D_{0}^{Mo}$   & $E_{a}^{Mo}$ \\
	\midrule
	\multirow{2}{*}{ $\gamma$U-7Mo }
		& 600 - 800
		& 1.03 $\times 10^{-09}$ & 0.433
		& 8.87 $\times 10^{-11}$ & 0.366 \\
		& 900 - 1200
		& 4.10 $\times 10^{-08}$ & 0.715
		& 2.77 $\times 10^{-08}$ & 0.806 \vspace{0.2cm } \\
	\multirow{2}{*}{ $\gamma$U-10Mo }
		& 600 - 800
		& 1.96 $\times 10^{-10}$ & 0.362
		& 5.26 $\times 10^{-11}$ & 0.353 \\
		& 900 - 1200
		& 2.10 $\times 10^{-08}$ & 0.712
		& 1.31 $\times 10^{-08}$ & 0.779 \vspace{0.2cm } \\
	\multirow{2}{*}{ $\gamma$U-12Mo }
		& 600 - 800
		& 1.81 $\times 10^{-10}$ & 0.380
		& 6.17 $\times 10^{-11}$ & 0.375 \\
		& 900 - 1200
		& 1.33 $\times 10^{-08}$ & 0.700
		& 6.62 $\times 10^{-09}$ & 0.738 \\
	\bottomrule
	\end{tabular}
	\end{table}
\end{frame}

\section{Re-solution of Xe gas bubbles
in \texorpdfstring{$\gamma$}{gamma}U-10Mo}

\begin{frame}{Bubble re-solution}
	\begin{columns}
	\column{0.5\textwidth}
		\begin{itemize}
			\item Fission gases can diffuse to grain boundaries
				to form intergranular bubbles
				where they can grow by addition of gas and vacancies.
			\item Fission also strongly limits bubble size
				(or fission-gas atom population)
				in the fuel lattice by re-solution of gas bubbles.
			\item Individual gas atoms can be ejected from the gas bubble
				through collisions with energetic fission fragments
				or recoil atoms (PKAs).
			\item Bubbles can also be destroyed
				by the passage of fission tracks
				where local temperature can be higher than
				the fuel melting temperature.
		\end{itemize}
	\column{0.5\textwidth}
		\begin{figure}[ht]
			\centering
			\begin{subfigure}{0.6\textwidth}
				\centering
				\caption{}
				\includegraphics[width=\textwidth]{sfigs/homogeneous.png}
			\end{subfigure}

			\begin{subfigure}{0.6\textwidth}
				\centering
				\caption{}
				\includegraphics[width=\textwidth]{sfigs/heterogeneous.png}
			\end{subfigure}
			\caption{
				Illustration of
				(a) homogeneous and (b) heterogeneous re-solution.
			}
		\end{figure}
	\end{columns}
\end{frame}

\begin{frame}{Re-solution setup}
	\begin{itemize}
		\item Two types of simulations are performed:
			primary knock-on atom (PKA)
			and thermal spike (fission track) simulations
			for low- and high-energy interactions.
			These simulations account for re-solution
			through ballistic and thermal processes.
		\item For the PKA model supercell is $160 \times 160 \times 160$
			\r{A}$^3$, $\sim$ 8 million atoms in the system.
			For the thermal spike model,
			the supercell is $160 \times 160 \times 80$ \r{A}$^3$,
			$\sim$ 4 million atoms.
			The simulations are performed in NVE ensembles,
			with an NVT region around the edges.
		\item For the PKA model, PKA energies up to 500 keV,
			and for the thermal spike model, we impart up to 30 keV/nm
			to a cylindrical region containing the gas bubble.
		\item A void is first created in the middle of the supercell,
			then Xe atoms are inserted up to certain Xe/vac ratios.
			Xe atoms that traverse more than 1 nm
			from the surface of the spherical void region
			are counted as re-solved Xe atoms.
	\end{itemize}
\end{frame}

\begin{frame}{PKA simulation progression}
	\begin{columns}
	\column{0.5\textwidth}
		\begin{itemize}
			\item In the PKA model, the imparted energy propagates quickly
				as a shock wave and creates many point defects,
				the majority of which eventually annihilate.
			\item The gas bubble is seen to deform a little
				in the beginning stage of the shock wave,
				however, minimal re-solution is observed.
			\item The re-solution rates, considering uncertainty,
				are effectively zero,
				and are neglected from further examination.
				This is similar to observations in UO$_2$.
		\end{itemize}
	\column{0.5\textwidth}
		\begin{figure}[ht]
			\centering
			\begin{subfigure}{0.49\textwidth}
				\centering
				\caption{}
				\includegraphics[width=\textwidth]{resol/images/pka1.png}
			\end{subfigure}
			\begin{subfigure}{0.49\textwidth}
				\centering
				\caption{}
				\includegraphics[width=\textwidth]{resol/images/pka2.png}
			\end{subfigure}

			\begin{subfigure}{0.49\textwidth}
				\centering
				\caption{}
				\includegraphics[width=\textwidth]{resol/images/pka3.png}
			\end{subfigure}
			\begin{subfigure}{0.49\textwidth}
				\centering
				\caption{}
				\includegraphics[width=\textwidth]{resol/images/pka4.png}
			\end{subfigure}
			\caption{
				Snapshots of a PKA (500 keV) simulation
				at (a) 0, (b) 1.5, (c) 44.5, and (d) 114.5 ps.
				U, Mo, and Xe atoms are shown in red, blue, and black.
			}
		\end{figure}
	\end{columns}
\end{frame}

\begin{frame}{Thermal spike progression}
	\begin{columns}
	\column{0.5\textwidth}
		\begin{itemize}
			\item The thermal spike imparts a large amount of kinetic energy
				to the atoms, which propagates out quickly.
			\item Thermal spikes induce significant re-solution,
				seeming to break apart the entire bubble
				into a loose collection of Xe atoms.
			\item The Xe atoms coalesce to reform a gas bubbles,
				but a significant portion remain outside
				the initial spherical region.
				These re-solved atoms may be isolated
				or form small Xe clusters.
			\item Initial bubble radii are varied from 5 \r{A} to 40 \r{A}.
		\end{itemize}
	\column{0.5\textwidth}
		\begin{figure}
			\begin{subfigure}{0.49\textwidth}
				\centering
				\caption{}
				\includegraphics[width=\textwidth]{resol/images/spike1.png}
			\end{subfigure}
			\begin{subfigure}{0.49\textwidth}
				\centering
				\caption{}
				\includegraphics[width=\textwidth]{resol/images/spike2.png}
			\end{subfigure}

			\begin{subfigure}{0.49\textwidth}
				\centering
				\caption{}
				\includegraphics[width=\textwidth]{resol/images/spike3.png}
			\end{subfigure}
			\begin{subfigure}{0.49\textwidth}
				\centering
				\caption{}
				\includegraphics[width=\textwidth]{resol/images/spike4.png}
			\end{subfigure}
			\caption{
				Snapshots of a thermal spike (30 keV/nm) simulation
				at (a) 0, (b) 1.5, (c) 44.5, and (d) 114.5 ps.
				U, Mo, and Xe atoms are shown in red, blue, and black.
			}
		\end{figure}
	\end{columns}
\end{frame}

\begin{frame}{Bubble size dependence}
	\begin{columns}
	\column{0.5\textwidth}
		\begin{itemize}
			\item The fraction of re-solved atoms is a convenient way
				to plot different bubble sizes for comparisons.
			\item Error bars denote the standard deviations
				calculated from 5 bubbles of each radius.
			\item Data can be fit to an exponentially saturating function:
				$ \chi_0 = 1 - \exp[-\alpha S_{e,eff}] $
			\item $S_{e,eff}$ is the effective energy
				transferred to the lattice
				and $\alpha$ is the saturation factor.
			\item The saturation factor itself is a function of radius:
				$ \alpha = \frac{5.1}{R_{bubble}^{2.2}} $
		\end{itemize}
	\column{0.5\textwidth}
		\begin{figure}[ht]
			\centering
			\begin{subfigure}{\textwidth}
				\centering
				\includegraphics[height=3.0cm]
					{resol/images/resolutionVsRadius.pdf}
			\end{subfigure}
			\begin{subfigure}{\textwidth}
				\centering
				\includegraphics[height=2.5cm]
					{resol/images/saturationFactor.pdf}
			\end{subfigure}
			\caption{
				(Top) Fraction of re-solved Xe atoms
				as a function of the energy deposited to the lattice.
				(Bottom) Saturation factor as a function of bubble radius.
			}
		\end{figure}
	\end{columns}
\end{frame}

\begin{frame}{Off-centered thermal spike}
	\begin{columns}
	\column{0.5\textwidth}
		\begin{itemize}
			\item Off-centered thermal spikes are simulated as well,
				with variable degree of off-centeredness.
				By varying the amount of overlap,
				a spatial distribution can be generated.
			\item At an off-center distance of $r_c := R_{bubble} + R_{spike}$,
				the re-solution stops.
			\item Normalized fraction of re-solved Xe atoms
				as a function of normalized off-centered distance
				can be modeled with a logistic equation:
				$
					\frac{\chi}{\chi_0}
						= \frac{1.058}{1 + \exp \big[8.168
						\big(\frac{r}{r_c}\big) - 3.331 \big]}
				$
		\end{itemize}
	\column{0.5\textwidth}
		\begin{figure}[ht]
			\centering
			\includegraphics[height=4.5cm]{resol/images/offcentered.pdf}
			\caption{
				Xe re-solution caused by
				off-centered thermal spikes (15 keV/nm).
			}
		\end{figure}
	\end{columns}
\end{frame}

\begin{frame}{Effect of bubble pressure}
	\begin{columns}
	\column{0.5\textwidth}
		\begin{itemize}
			\item Xe/vac ratio determines bubble pressure.
				Bubbles with different Xe/vac ratios were simulated
				to evaluate the effect of pressure.
			\item The number of re-solved Xe atoms is apparently invariant
				with respect to the Xe/vac ratio.
			\item With increasing thermal spike energy,
				more atoms are re-solved as expected.
			\item Similar trends were observed in bubbles
				with radii of 15 \r{A} and 35 \r{A}.
		\end{itemize}
	\column{0.5\textwidth}
		\begin{figure}[ht]
			\centering
			\includegraphics[width=\textwidth]{resol/images/xevac.pdf}
			\caption{
				Number of re-solved Xe atoms from 25 \r{A} radius bubbles
				as a function of Xe/vacancy ratio.
			}
		\end{figure}
	\end{columns}
\end{frame}

\begin{frame}{Pressure and size invariance}
	\begin{columns}
	\column{0.5\textwidth}
		\begin{itemize}
			\item The number of re-solved atoms from identical-radius bubbles
				with different pressures were averaged.
			\item The number of re-solved atoms also appears consistent
				across bubbles of various sizes.
			\item Thermal spikes create low-density regions
				around the Xe gas bubble,
				and these regions facilitate the separation of Xe atoms
				from the bubble.
			\item Therefore, thermal spike energy dictates
				the volume of these low-energy regions,
				and thus affects the total number of re-solved atoms.
		\end{itemize}
	\column{0.5\textwidth}
		\begin{figure}[ht]
			\centering
			\includegraphics[width=\textwidth]{resol/images/r2dep.pdf}
			\caption{
				Number of re-solved Xe atoms against bubble radius.
				Data from identical-radius bubbles
				with different Xe/vacancy ratios were averaged.
			}
		\end{figure}
	\end{columns}
\end{frame}

\begin{frame}{Calculation of re-solution rate}
	\begin{columns}
	\column{0.5\textwidth}
		\begin{itemize}
			\item We need to sum up all the contributions
				from all the fission products originating
				at different distances from a bubble and different offsets
				by means of a volume integral.
			\item The integral can be expressed explicitly
				in terms of fission product yields,
				and $\zeta = S_{e,eff} / S_{e}$,
				the fraction of FP energy that is imparted to the lattice.
			\item
				$
					f(R_{bubble}, \dot{F}) $
				$
					= \sum_{i=1}^2 \int_{x} \int_{r}
						\chi_i \dot{F} 2 \pi r dr dx $
				$
					= \dot{F} \sum_{i=1}^2 \int_{r}
						\bigg( \frac{\chi_i}{\chi_{0,i}} \bigg)
						2 \pi r dr \int_{x} \chi_{0,i} dx $
		\end{itemize}
	\column{0.5\textwidth}
		\begin{figure}[ht]
			\centering
			\includegraphics[height=5cm]{resol/images/coordSystem.pdf}
			\caption{
				Cylindrical coordinate system for calculating
				the heterogeneous re-solution rate.
			}
		\end{figure}
	\end{columns}
\end{frame}

\begin{frame}{Electronic stopping power}
	\begin{columns}
	\column{0.5\textwidth}
		\begin{itemize}
			\item The electronic stopping power of the fission products (FPs)
				Xe and Sr in U-10Mo are calculated using SRIM.
			\item Ion irradiation is simulated 2000 times
				for each fission product.
			\item Equations are fit to the data, which can be integrated,
				allowing for the calculation of the re-solution rate.
			\item
				$
					S_{e,Xe} = 21.3 \exp(-0.239 x^{1.78})
						+ 5.23 \exp(-4.67 \times 10^{-8} x^{11}) $
			\item
				$
					S_{e,Sr} = 19.7 \exp(-0.00273 x^{3.71})
						+ 6.8 \exp(-0.424 x^{1.45}) $
		\end{itemize}
	\column{0.5\textwidth}
		\begin{figure}[ht]
			\centering
			\includegraphics[height=4.5cm]{resol/images/elec_stopping.pdf}
			\caption{
				Total electronic stopping power ($S_e$)
				of Xe-140 and Sr-94 in $\gamma$U-10Mo,
				as calculated by the SRIM software as a function of
				distance traversed by the fission product
				from the location of the fission reaction.
			}
			\label{fig:elec}
		\end{figure}
	\end{columns}
\end{frame}

\begin{frame}{Re-solution rate at nominal pressure}
	\begin{columns}
	\column{0.5\textwidth}
		\begin{itemize}
			\item The re-solution rate for the nominal pressure
				is then computed using numerical integration.
			\item An approximate analytical form is:
				$ f(R_{bubble}, \dot{F}) $
				$ = \frac{a}{1+(R_{bubble}/c)^d} 10^{-14} \dot{F} $
			\item The DART model is defined as:
				$ b_{dart} = b_0 \cdot \dot{F} \cdot G $
				where
				$ b_0 = R_{spike}^2 \cdot \mu_{ff} $
				and
				$ G =
					\begin{cases}
						1 \qquad \qquad \qquad \;\;\; ,
							R_{bubble} \leq \lambda \\
						1 - (\frac{R_{bubble}-R_{resol}}{R_{bubble}})^3,
							R_{bubble} > \lambda
					\end{cases} $
			\item We have a smooth function for re-solution rate now
				instead of a piecewise one.
		\end{itemize}
	\column{0.5\textwidth}
		\begin{figure}[ht]
			\centering
			\includegraphics[width=6cm]{resol/images/resRate_withDart.pdf}
			\caption{
				Xe gas bubble re-solution rate in $\gamma$U-10Mo
				as a function of bubble radius for Xe/vac ratio of 0.2
				at a fission rate of $10^{14}$ fiss/cm$^3$/s for $\zeta=0.01$.
			}
		\end{figure}
	\end{columns}
\end{frame}

\begin{frame}{Final re-solution rate}
	\begin{columns}
	\column{0.5\textwidth}
		\begin{itemize}
			\item With the re-solution rate
				for the nominal pressure already provided,
				accounting for bubble pressure is straightforward
				due to the pressure invariance we have observed.
			\item
				$ b_{het}(R_{bubble}, \dot{F}, \phi) $
				$ = f(R_{bubble}, \dot{F}) \cdot g(\phi) $
				$ = \bigg[ \frac{a}{1+(R_{bubble}/c)^d} 10^{-14} \dot{F} \bigg]
					\bigg( \frac{0.2}{\phi} \bigg) $
			\item Here, $a$, $c$, and $d$  depend on the value of $\zeta$.
			\item Re-solution rate decreases with
				both bubble size and pressure.
		\end{itemize}
	\column{0.5\textwidth}
		\begin{figure}[ht]
			\centering
			\includegraphics[trim={4cm 2cm 1cm 2cm}, clip, width=\textwidth]
				{resol/images/3d.pdf}
			\caption{
				Xe gas bubble re-solution rate in $\gamma$U-10Mo
				as a function of bubble radius and Xe/vacancy ratio
				at a fission rate of $10^{14}$ fiss/cm$^3$/s for $\zeta=0.01$.
			}
		\end{figure}
	\end{columns}
\end{frame}

\section{IUQ of \texorpdfstring{$\gamma$}{gamma}U-10Mo
fission-gas-behavior parameters}

\begin{frame}{Inverse uncertainty quantification (IUQ)}
	\begin{columns}
	\column{0.5\textwidth}
		\begin{itemize}
			\item IUQ is a process to quantify uncertainties
				in the input parameters of a computer model
				given experimental data.
			\item It's the opposite of
				forward uncertainty quantification (FUQ),
				which quantifies the uncertainty in the output given the input.
			\item IUQ quantifies the uncertainty in the model itself,
				and thus can help to improve
				the accuracy of the model predictions.
			\item It's helpful in situations where the underlying parameters
				of the model are unknown.
		\end{itemize}
	\column{0.5\textwidth}
		\begin{figure}[ht]
			\centering
			\includegraphics[width=6cm]{sfigs/wu_iuq.png}
			\caption{
				Some essential parts of modeling and simulation
				(Wu et al. {\color{blue}
				\url{https://doi.org/10.1016/j.nucengdes.2018.06.004}}).
			}
		\end{figure}
	\end{columns}
\end{frame}

\begin{frame}{Bayesian framework for IUQ}
	\begin{columns}
	\column{0.4\textwidth}
		\begin{itemize}
			\item $y^E$ is the observation
				and $y^R$ is the reality.
				$y^M$ is the model with a bias $\delta(x)$.
			\item $\epsilon$ is the measurement error
				and assumed to be Gaussian.
			\item To get posterior distributions of parameters $\theta^*$,
				Bayes' theorem can be used.
			\item Since the error is related to $y^E$ and $y^M$,
				the likelihood can be described in terms of the error.
		\end{itemize}
	\column{0.6\textwidth}
		\begin{align}
			\bm{y}^R(\bm{x})
			&= \bm{y}^M(\bm{x}, \bm{\theta}^*) + \delta(\bm{x}) \\
			\bm{y}^E(\bm{x})
			&= \bm{y}^R(\bm{x}) + \bm{\epsilon} \\
			\bm{y}^E(\bm{x})
			&= \bm{y}^M(\bm{x}, \bm{\theta}^*)
				+ \delta(\bm{x}) + \bm{\epsilon} \\
			\bm{\epsilon}
			&= \bm{y}^E(\bm{x}) - \bm{y}^M(\bm{x}, \bm{\theta}^*)
				- \delta(\bm{x}) \\
			\bm{\epsilon}
			&\sim \mathcal{N} (\bm{\mu}, \bm{\Sigma}_{exp}) \\
			p(\bm{\theta}^* | \bm{y}^E, \bm{y}^M)
			&\propto p(\bm{\theta}^*)
				\cdot p(\bm{y}^E, \bm{y}^M | \bm{\theta}^*) \\
			\propto p(\bm{\theta}^*)
				&\cdot \frac{1}{\sqrt{|\bm{\Sigma}|}}
				\exp \bigg[-\frac{1}{2} \bm{\epsilon}^T
				\bm{\Sigma}^{-1} \bm{\epsilon} \bigg] \\
			\bm{\Sigma}
			&= \bm{\Sigma}_{exp}
				+ \bm{\Sigma}_{bias} + \bm{\Sigma}_{code}
		\end{align}
	\end{columns}
\end{frame}

\begin{frame}{Fission-gas-behavior parameters}
	\begin{table}[ht]
	\centering
	\caption{
		Fission gas behavior parameters and their ranges.
	}
	\begin{tabular}{lccc}
	\toprule
	Parameter    & Minimum value & Maximum value & Reference value \\
	\midrule
	dGrainHBS    & \num{2.4e-5 } & \num{5.6e-5 } & \num{4e-5   }   \\
	FaceCovMax   & \num{0.6    } & \num{0.907  } & \num{0.907  }   \\
	SwellLink    & \num{0.016  } & \num{0.034  } & \num{0.025  }   \\
	vResol       & \num{1.3e-18} & \num{2.7e-18} & \num{2e-18  }   \\
	DatomFissGBx & \num{19641  } & \num{40530  } & \num{3e4    }   \\
	fNucleate    & \num{3.5e-10} & \num{8.3e-10} & \num{6e-10  }   \\
	aAtomDifFiss & \num{3.2e-31} & \num{7.2e-31} & \num{5.1e-31}   \\
	\bottomrule
	\end{tabular}
	\end{table}

	\begin{itemize}
		\item DART is run with 7 fission-gas-behavior parameters perturbed
			to compile a dataset necessary for IUQ.
		\item 3200 samples were generated at different fission density values.
		\item The list above contains the names of the parameters
			we are interested in, and their probed ranges.
	\end{itemize}
\end{frame}

\begin{frame}{Experimental observation}
	\begin{columns}
	\column{0.5\textwidth}
		\begin{itemize}
			\item For our problem, the observation comes from
				the existing fuel swelling correlations.
			\item The goal is thus to find
				the fission-gas-behavior parameter distributions
				that correspond to the experimental correlations.
			\item The Robinson correlation is created
				using data from different irradiation experiments
				and different operational conditions.
			\item DART predictions are very narrow at low fission densities
				and thus not helpful for IUQ purposes.
		\end{itemize}
	\column{0.5\textwidth}
		\begin{figure}[ht]
			\centering
			\begin{subfigure}{\textwidth}
				\centering
				\includegraphics[width=0.7\textwidth]
					{sfigs/robinson_corr.png}
			\end{subfigure}
			\begin{subfigure}{\textwidth}
				\centering
				\includegraphics[width=0.7\textwidth]
					{invuq/images/robinson_vs_dart.png}
			\end{subfigure}
			\caption{
				(Top) Robinson fuel swelling correlation.
				(Bottom) Robinson correlation and DART prediction.
			}
		\end{figure}
	\end{columns}
\end{frame}

\begin{frame}{Target for IUQ}
	\begin{columns}
	\column{0.5\textwidth}
		\begin{itemize}
			\item We chose the highest fission density
				($7 \times 10^{21}$ fiss/cm$^3$) data for IUQ.
			\item At this fission density,
				DART prediction covers the entire prediction bound
				of the Robinson correlation.
			\item The goal is to find a set of parameter distributions
				that leads to a model prediction
				matching the experimental observation.
		\end{itemize}
	\column{0.5\textwidth}
		\begin{figure}[ht]
			\centering
			\includegraphics[width=5cm]{invuq/images/target.png}
			\caption{
				Robinson correlation and DART prediction
				at $7 \times 10^{21}$ fiss/cm$^3$.
			}
		\end{figure}
	\end{columns}
\end{frame}

\begin{frame}{Data exploration: scatter plots}
	\begin{figure}[ht]
		\centering
		\includegraphics[width=8cm]{invuq/images/scatter.png}
		\caption{
			Scatter plots of fuel swelling
			against the fission-gas-behavior parameters.
		}
	\end{figure}

	\begin{itemize}
		\item Except for the parameters "dGrainHBS" and "FaceCovMax",
			all other parameters have weak correlations with fuel swelling.
	\end{itemize}
\end{frame}

\begin{frame}{Data exploration: correlation heatmap}
	\begin{columns}
	\column{0.5\textwidth}
		\begin{itemize}
			\item To also understand the correlation among the parameters,
				correlation heatmap is shown.
			\item The parameters themselves are also
				uncorrelated to each other.
			\item The heatmap corroborates the observation
				from the scatter plots.
			\item The parameter "SwellLink" seems to have
				a non-negligible correlation,
				along with "dGrainHBS" and "FaceCovMax".
		\end{itemize}
	\column{0.5\textwidth}
		\begin{figure}[ht]
			\centering
			\includegraphics[width=5cm]{invuq/images/corrheat.png}
			\caption{
				Correlation heatmap of fission-gas-behavior parameters
				and fuel swelling.
			}
		\end{figure}
	\end{columns}
\end{frame}

\begin{frame}{Surrogate modeling}
	\begin{columns}
	\column{0.5\textwidth}
		\begin{itemize}
			\item We need to perturb the parameter space
				to match the observation.
			\item But it's not feasible to run calculations
				for all possible parameter combinations
				using DART.
			\item A surrogate model
				(also called a response surface or an emulator)
				is thus needed.
			\item Many machine learning models,
				such as Gaussian processes and neural networks,
				can act as a surrogate model.
		\end{itemize}
	\column{0.5\textwidth}
		\begin{figure}[ht]
			\centering
			\includegraphics[width=\textwidth]{sfigs/surrogate_modeling.png}
			\caption{
				The principle of surrogate modeling
				(Kocijan et al.
				{\color{blue}\url{https://doi.org/10.5516/NET.07.2014.706}}).
			}
		\end{figure}
	\end{columns}
\end{frame}

\begin{frame}{Surrogate models}
	\begin{columns}
	\column{0.5\textwidth}
		\begin{itemize}
			\item A few linear (OLS, Lasso, and Ridge)
				and non-linear (GP, NN, and SVR)
				are used as surrogate models.
			\item The training and testing of most surrogate models
				are done on a 70/30 split.
				Only the GP model is trained with 14\% of the total data.
			\item All surrogate models have R$^2$ scores around 0.997.
			\item Lasso made the coefficients of 4 parameters zero,
				thus selecting the parameters
				"dGrainHBS", "FaceCovMax", and "SwellLink"
				as the influential ones.
		\end{itemize}
	\column{0.5\textwidth}
		\begin{figure}[ht]
			\centering
			\begin{subfigure}{0.49\textwidth}
				\centering
				\includegraphics[width=\textwidth]
					{invuq/images/ols.png}
			\end{subfigure}
			\begin{subfigure}{0.49\textwidth}
				\centering
				\includegraphics[width=\textwidth]
					{invuq/images/lasso.png}
			\end{subfigure}
			\begin{subfigure}{0.49\textwidth}
				\centering
				\includegraphics[width=\textwidth]
					{invuq/images/gp.png}
			\end{subfigure}
			\begin{subfigure}{0.49\textwidth}
				\centering
				\includegraphics[width=\textwidth]
					{invuq/images/nn.png}
			\end{subfigure}
			\caption{
				Surroage prediction vs actual test data
				for surrogate models.
			}
		\end{figure}
	\end{columns}
\end{frame}

\begin{frame}{MCMC samples}
	\begin{figure}[ht]
		\centering
		\includegraphics[width=11cm]{invuq/images/trace.png}
		\caption{
			Trace plots of the posterior samples
			of the fission-gas-behavior parameters from MCMC sampling.
		}
	\end{figure}

	\begin{itemize}
		\item Two chains of 100,000 samples are overlaid
			on the trace plots.
			All diagnostics point to convergence.
		\item The cumulative averages of the blue and orange traces
			are shown in black and red, respectively.
		\item "FaceCovMax" and "SwellLink" are essentially bounded
			by the priors.
	\end{itemize}
\end{frame}

\begin{frame}{Posterior distributions}
	\begin{figure}[ht]
		\centering
		\includegraphics[width=11cm]{invuq/images/hist.png}
		\caption{
			Density plots of the posterior samples
			of the fission-gas-behavior parameters from MCMC sampling.
		}
	\end{figure}

	\begin{itemize}
		\item The density plots (also posteriors) from both MCMC chains
			are similar.
		\item The posterior of "dGrainHBS" is reminiscent
			of a Gaussian distribution.
		\item The other two parameter distributions are more akin
			to a uniform distribution.
		\item The ranges of the parameters "FaceCovMax" and "SwellLink"
			might have been chosen in a constricted manner.
	\end{itemize}
\end{frame}

\begin{frame}{Joint probability distributions}
	\begin{columns}
	\column{0.5\textwidth}
		\begin{itemize}
			\item MCMC samples are thinned
				so that the number of samples is approximately equal
				to the effective sample size (ESS).
				Thinning didn't change
				the characteristics of the distributions.
			\item The correlations among the posteriors are displayed
				as contour plots.
			\item Even though the prior distributions
				are assumed to be independent,
				IUQ still identified the correlations among the posteriors.
			\item "dGrainHBS" and "FaceCovMax"
				have a strong positive correlation,
				and "dGrainHBS" and "SwellLink" have a weak one.
		\end{itemize}
	\column{0.5\textwidth}
		\begin{figure}[ht]
			\centering
			\includegraphics[width=5cm]{invuq/images/iuq.png}
			\caption{
				Posterior distributions of 3 fission-gas-behavior parameters.
				Marginal densities are in the diagonal
				and the pair-wise joint densities are in the off-diagonal.
			}
		\end{figure}
	\end{columns}
\end{frame}

\begin{frame}{FUQ and validation}
	\begin{columns}
	\column{0.5\textwidth}
		\begin{itemize}
			\item To validate IUQ results,
				posterior samples are propagated using the GP surroate.
			\item A multivariate Gaussian copula is fitted to the MCMC samples.
				Synthetic samples are then generated using the copula,
				and then propagated using the GP surrogate.
			\item The fitted copula does a decent job,
				although it's not as accurate.
				The lower accuracy is possibly due to
				the usage of a Gaussian copula with uniform-like posteriors.
		\end{itemize}
	\column{0.5\textwidth}
		\begin{figure}[ht]
			\centering
			\begin{subfigure}{0.65\textwidth}
				\centering
				\includegraphics[width=\textwidth]
					{invuq/images/fuq_gp.png}
			\end{subfigure}
			\begin{subfigure}{0.65\textwidth}
				\centering
				\includegraphics[width=\textwidth]
					{invuq/images/fuq_gp_synth.png}
			\end{subfigure}
			\caption{
				Forward propagation using GP surrogate with
				(Top) posterior samples from MCMC
				and (Bottom) synthetic samples generated using copulas.
			}
		\end{figure}
	\end{columns}
\end{frame}

\section{U-Mo moment tensor potential}

\begin{frame}{Moment tensor potential (MTP)}
	\begin{columns}
	\column{0.5\textwidth}
		\begin{itemize}
			\item Item 1
			\item Item 2
			\item Item 3
		\end{itemize}
	\column{0.5\textwidth}
		\begin{figure}[ht]
			\centering
			\includegraphics[width=5cm]{example-image-a}
			\caption{
				An example caption!
			}
		\end{figure}
	\end{columns}
\end{frame}

\begin{frame}{Lattice constants}
	\begin{columns}
	\column{0.5\textwidth}
		\begin{itemize}
			\item Different U and Mo phases and intermetallics
				are simulated using the MTP.
				All simulations are run for 100 ps in NPT ensembles
				to get equilibrated supercell dimensions.
			\item The MTP can accurately predict lattice constants
				of different phases of U, bcc Mo, and U-Mo compounds.
			\item Noncubic systems are generally harder to replicate
				than cubic systems for potentials.
			\item The MTP doesn't struggle with
				$\alpha$U (orthorombic) and U$_2$Mo (tetragonal).
		\end{itemize}
	\column{0.5\textwidth}
		\begin{figure}[ht]
			\centering
			\begin{subfigure}{0.7\textwidth}
				\centering
				\includegraphics[width=\textwidth]
					{mtp/images/latConst_alphaU.pdf}
			\end{subfigure}

			\begin{subfigure}{0.7\textwidth}
				\centering
				\includegraphics[width=\textwidth]
					{mtp/images/latConst_gammaU.pdf}
			\end{subfigure}
			\caption{
				Lattice constants of $\alpha$U and $\gamma$U.
			}
		\end{figure}
	\end{columns}
\end{frame}

\begin{frame}{Lattice constants (cont.)}
	\begin{figure}[ht]
		\centering
		\begin{subfigure}{0.45\textwidth}
			\centering
			\includegraphics[width=\textwidth]
				{mtp/images/latConst_bccMo.pdf}
		\end{subfigure}
		\begin{subfigure}{0.45\textwidth}
			\centering
			\includegraphics[width=\textwidth]
				{mtp/images/latConst_u2mo.pdf}
		\end{subfigure}
		\caption{
			Lattice constants of bcc Mo and U$_2$Mo.
		}
	\end{figure}

	\begin{itemize}
		\item The lattice constants of bcc Mo and U$_2$Mo
			from MTP and ADP agree with each other
			for the entire temperature range.
	\end{itemize}
\end{frame}

\begin{frame}{Radial distribution functions}
	\begin{columns}
	\column{0.5\textwidth}
		\begin{itemize}
			\item The radial distribution functions (RDFs)
				are calculated using the last five timesteps
				of the simulations of 100 ps runs with a 1 fs timestep size.
			\item RDFs computed with MTP and ADP are similar in all cases
				up to 10 \AA.
			\item There are slight discrepancies in the peak heights,
				but general trends are correct.
		\end{itemize}
	\column{0.5\textwidth}
		\begin{figure}[ht]
			\centering
			\begin{subfigure}{0.7\textwidth}
				\centering
				\includegraphics[width=\textwidth]
					{mtp/images/rdf_alphaU_500K.pdf}
			\end{subfigure}

			\begin{subfigure}{0.7\textwidth}
				\centering
				\includegraphics[width=\textwidth]
					{mtp/images/rdf_gammaU_950K.pdf}
			\end{subfigure}
			\caption{
				RDFs of $\alpha$U and $\gamma$U.
			}
		\end{figure}
	\end{columns}
\end{frame}

\begin{frame}{Radial distribution functions (cont.)}
	\begin{figure}[ht]
		\centering
		\begin{subfigure}{0.45\textwidth}
			\centering
			\includegraphics[width=\textwidth]
				{mtp/images/rdf_bccMo_1600K.pdf}
		\end{subfigure}
		\begin{subfigure}{0.45\textwidth}
			\centering
			\includegraphics[width=\textwidth]
				{mtp/images/rdf_u2mo_500K.pdf}
		\end{subfigure}
		\caption{
			RDFs of bcc Mo and U$_2$Mo.
		}
	\end{figure}

	\begin{itemize}
		\item Most noticeable discrepancy occurs for U$_2$Mo.
			The first and second nearest neighbor densities for U-U pairs
			are predicted differently by MTP and ADP.
	\end{itemize}
\end{frame}

\begin{frame}{Elastic constants}
	\begin{columns}
	\column{0.5\textwidth}
		\begin{itemize}
			\item Elastic constants $C_{11}$, $C_{12}$, and $C_{44}$
				are calculated for all systems.
			\item In all cases, the MTP overpredicts the elastic constants
				by two to three times compared to the ADP.
			\item Thus, the simulated systems are more rigid
				than they are supposed to be.
			\item This will lead to systems with
				inaccurate mechanical properties, defect energies,
				and unrealistic phase stability.
		\end{itemize}
	\column{0.5\textwidth}
		\begin{figure}[ht]
			\centering
			\begin{subfigure}{0.7\textwidth}
				\centering
				\includegraphics[width=\textwidth]
					{mtp/images/elConst_alphaU.pdf}
			\end{subfigure}

			\begin{subfigure}{0.7\textwidth}
				\centering
				\includegraphics[width=\textwidth]
					{mtp/images/elConst_gammaU.pdf}
			\end{subfigure}
			\caption{
				Elastic constants of $\alpha$U and $\gamma$U.
			}
		\end{figure}
	\end{columns}
\end{frame}

\begin{frame}{Elastic constants (cont.)}
	\begin{figure}[ht]
		\centering
		\begin{subfigure}{0.45\textwidth}
			\centering
			\includegraphics[width=\textwidth]
				{mtp/images/elConst_bccMo.pdf}
		\end{subfigure}
		\begin{subfigure}{0.45\textwidth}
			\centering
			\includegraphics[width=\textwidth]
				{mtp/images/elConst_u2mo.pdf}
		\end{subfigure}
		\caption{
			Elastic constants of bcc Mo and U$_2$Mo.
		}
	\end{figure}

	\begin{itemize}
		\item Apart from the constants being higher generally,
			the temperature dependence of elastic constants
			are also predicted differently by the MTP
			for bcc Mo and U$_2$Mo.
	\end{itemize}
\end{frame}

\section{Conclusions}

\begin{frame}{Conclusions}
	\begin{columns}
	\column{0.5\textwidth}
		\begin{itemize}
			\item Item 1
			\item Item 2
			\item Item 3
		\end{itemize}
	\column{0.5\textwidth}
		\begin{figure}[ht]
			\centering
			\includegraphics[width=5cm]{example-image-a}
			\caption{
				An example caption!
			}
			% \label{fig:template}
		\end{figure}
	\end{columns}
\end{frame}

\begin{frame}{Future work}
	\begin{columns}
	\column{0.5\textwidth}
		\begin{itemize}
			\item Item 1
			\item Item 2
			\item Item 3
		\end{itemize}
	\column{0.5\textwidth}
		\begin{figure}[ht]
			\centering
			\includegraphics[width=5cm]{example-image-a}
			\caption{
				An example caption!
			}
			% \label{fig:template}
		\end{figure}
	\end{columns}
\end{frame}

\section*{Thanks}

\begin{frame}
	\centering \Large
	\emph{Thank you!}
\end{frame}

\end{document}
