\documentclass[10pt]{beamer}
\usepackage{booktabs, subcaption}
\usepackage{amsmath, amssymb, bm}

\usetheme{Boadilla}
\usecolortheme{seahorse}
\setbeamertemplate{caption}[numbered]
\setbeamerfont{caption}{size=\scriptsize}

\title[Computational methods for $\gamma$U-Mo]
{Application of computational methods for the evaluation of
\texorpdfstring{$\gamma$}{gamma}U-Mo monolithic fuel}
\author{ATM Jahid Hasan}
\institute[NCSU]{Department of Nuclear Engineering\\
North Carolina State University}
\date{April 4, 2025}

\begin{document}
\frame{\titlepage}

\begin{frame}{Table of Contents}
	\tableofcontents
\end{frame}


% \section{sectiontitle}
% \begin{frame}{frametitle}
% 	\begin{columns}
% 	\column{0.5\textwidth}
% 		\begin{itemize}
% 			\item Item 1
% 			\item Item 2
% 			\item Item 3
% 		\end{itemize}
% 	\column{0.5\textwidth}
% 		\begin{figure}[ht]
% 			\centering
% 			\includegraphics[width=5cm]{example-image-a}
% 			\caption{
% 				An example caption!
% 			}
% 			\label{fig:template}
% 		\end{figure}
% 	\end{columns}
% \end{frame}


\section{Introduction}
\begin{frame}{Introduction}
	\begin{columns}
	\column{0.5\textwidth}
		\begin{itemize}
			\item Put empty slides with a certain template.
			\item Populate with all the figures after that.
		\end{itemize}
	\column{0.5\textwidth}
		\begin{figure}[ht]
			\centering
			\includegraphics[width=5cm]{sfigs/umo_design.jpg}
			\caption{
				Depiction of monolithic fuel cross-section
				(Meyer et al.
				{\color{blue}\url{https://doi.org/10.5516/NET.07.2014.706}}).
			}
			\label{fig:design}
		\end{figure}
		\begin{itemize}
			\item \textbf{Goals:}
				\begin{itemize}
					\item This is a good style!
				\end{itemize}
		\end{itemize}
	\end{columns}
\end{frame}


\section{Grain boundary diffusion in \texorpdfstring{$\gamma$}{gamma}U-Mo}
\begin{frame}{GB diffusion}
	\begin{columns}
	\column{0.5\textwidth}
		\begin{itemize}
			\item Item 1
			\item Item 2
			\item Item 3
		\end{itemize}
	\column{0.5\textwidth}
		\begin{figure}[ht]
			\centering
			\includegraphics[width=5cm]{example-image-a}
			\caption{
				An example caption!
			}
			\label{fig:diffusion}
		\end{figure}
	\end{columns}
\end{frame}


\section{Re-solution of Xe gas bubbles
in \texorpdfstring{$\gamma$}{gamma}U-10Mo}
\begin{frame}{Bubble re-solution}
	\begin{columns}
	\column{0.5\textwidth}
		\begin{itemize}
			\item Fission gases can diffuse to grain boundaries
				to form intergranular bubbles
				where they can grow by addition of gas and vacancies.
			\item Fission also strongly limits bubble size
				and maintains a substantial fission-gas atom population
				in the fuel lattice by re-solution of gas bubbles.
			\item Individual gas atoms can be ejected from the gas bubble
				through collisions with energetic fission fragments
				or recoil atoms (PKAs).
			\item Bubbles can also be destroyed
				by the passage of fission tracks
				where local temperature can be higher than
				the fuel melting temperature.
		\end{itemize}
	\column{0.5\textwidth}
		\begin{figure}[ht]
			\centering
			\begin{subfigure}{0.6\textwidth}
				\centering
				\caption{}
				\includegraphics[width=\textwidth]{sfigs/homogeneous.png}
			\end{subfigure}

			\begin{subfigure}{0.6\textwidth}
				\centering
				\caption{}
				\includegraphics[width=\textwidth]{sfigs/heterogeneous.png}
			\end{subfigure}
			\caption{
				Illustration of
				(a) homogeneous and (b) heterogeneous re-solution.
			}
			\label{fig:schematic}
		\end{figure}
	\end{columns}
\end{frame}

\begin{frame}{Re-solution setup}
	\begin{itemize}
		\item Two types of simulations are performed:
			primary knock-on atom (PKA)
			and thermal spike simulations (fission track)
			for low- and high-energy interactions.
			These simulations account for re-solution
			through ballistic and thermal processes.
		\item For the PKA model supercell is $160 \times 160 \times 160$
			\r{A}$^3$, $\sim$ 8 million atoms in the system.
			For the thermal spike model,
			the supercell is $160 \times 160 \times 80$ \r{A}$^3$,
			$\sim$ 4 million atoms.
			The simulations are performed in NVE ensembles,
			with an NVT region around the edges.
		\item For the PKA model, PKA energies up to 500 keV,
			and for the thermal spike model, we impart up to 30 keV/nm
			to a cylindrical region containing the gas bubble.
		\item A void is first created in the middle of the supercell,
			then Xe atoms are inserted up to certain Xe/vac ratios.
			Xe atoms that traverse more than 1 nm
			from the surface of the spherical void region
			are counted as re-solved Xe atoms.
	\end{itemize}
\end{frame}

\begin{frame}{PKA simulation progression}
	\begin{columns}
	\column{0.5\textwidth}
		\begin{itemize}
			\item In the PKA model, the imparted energy propagates quickly
				as a shock wave and creates many point defects,
				the majority of which eventually annihilate.
			\item The gas bubble is seen to deform a little
				in the beginning stage of the shock wave,
				however, minimal re-solution is observed.
			\item The re-solution rates, considering uncertainty,
				are effectively zero,
				and are neglected from further examination.
				This is similar to observations in UO$_2$.
		\end{itemize}
	\column{0.5\textwidth}
		\begin{figure}[ht]
			\centering
			\begin{subfigure}{0.49\textwidth}
				\centering
				\caption{}
				\includegraphics[width=\textwidth]{resol/images/pka1.png}
			\end{subfigure}
			\begin{subfigure}{0.49\textwidth}
				\centering
				\caption{}
				\includegraphics[width=\textwidth]{resol/images/pka2.png}
			\end{subfigure}

			\begin{subfigure}{0.49\textwidth}
				\centering
				\caption{}
				\includegraphics[width=\textwidth]{resol/images/pka3.png}
			\end{subfigure}
			\begin{subfigure}{0.49\textwidth}
				\centering
				\caption{}
				\includegraphics[width=\textwidth]{resol/images/pka4.png}
			\end{subfigure}
			\caption{
				Snapshots of a PKA (500 keV) simulation
				at (a) 0, (b) 1.5, (c) 44.5, and (d) 114.5 ps.
				U, Mo, and Xe atoms are shown in red, blue, and black.
			}
			\label{fig:pka}
		\end{figure}
	\end{columns}
\end{frame}

\begin{frame}{Thermal spike progression}
	\begin{columns}
	\column{0.5\textwidth}
		\begin{itemize}
			\item The thermal spike imparts a large amount of kinetic energy,
				which propagates out.
			\item Thermal spikes induce significant re-solution,
				seeming to break apart the entire bubble
				into a loose collection of Xe atoms.
			\item The Xe atoms coalesce to reform a gas bubbles,
				but a significant portion remain outside
				the initial spherical region.
				These re-solved atoms may be isolated
				or form small Xe clusters.
			\item Initial bubble radii are varied from 5 \r{A} to 40 \r{A}.
		\end{itemize}
	\column{0.5\textwidth}
		\begin{figure}
			\begin{subfigure}{0.49\textwidth}
				\centering
				\caption{}
				\includegraphics[width=\textwidth]{resol/images/spike1.png}
			\end{subfigure}
			\begin{subfigure}{0.49\textwidth}
				\centering
				\caption{}
				\includegraphics[width=\textwidth]{resol/images/spike2.png}
			\end{subfigure}

			\begin{subfigure}{0.49\textwidth}
				\centering
				\caption{}
				\includegraphics[width=\textwidth]{resol/images/spike3.png}
			\end{subfigure}
			\begin{subfigure}{0.49\textwidth}
				\centering
				\caption{}
				\includegraphics[width=\textwidth]{resol/images/spike4.png}
			\end{subfigure}
			\caption{
				Snapshots of a thermal spike (30 keV/nm) simulation
				at (a) 0, (b) 1.5, (c) 44.5, and (d) 114.5 ps.
				U, Mo, and Xe atoms are shown in red, blue, and black.
			}
			\label{fig:spike}
		\end{figure}
	\end{columns}
\end{frame}

\begin{frame}{Bubble size dependence}
	\begin{columns}
	\column{0.5\textwidth}
		\begin{itemize}
			\item The fraction of re-solved atoms is a convenient way
				to plot different bubble sizes for comparisons.
			\item Error bars denote the standard deviations
				calculated from 5 bubbles of each radius.
			\item Data can be fit to an exponentially saturating function:
				$ \chi_0 = 1 - \exp[-\alpha S_{e,eff}] $
			\item $S_{e,eff}$ is the effective energy
				transferred to the lattice
				and $\alpha$ is the saturation factor.
			\item The saturation factor itself is a function of radius:
				$ \alpha = \frac{5.1}{R_{bubble}^{2.2}} $
		\end{itemize}
	\column{0.5\textwidth}
		\begin{figure}[ht]
			\centering
			\begin{subfigure}{\textwidth}
				\centering
				\includegraphics[height=3.0cm]{resol/images/resolutionVsRadius.pdf}
			\end{subfigure}
			\begin{subfigure}{\textwidth}
				\centering
				\includegraphics[height=2.5cm]{resol/images/saturationFactor.pdf}
			\end{subfigure}
			\caption{
				(Top) Fraction of re-solved Xe atoms
				as a function of the energy deposited to the lattice.
				(Bottom) Saturation factor as a function of bubble radius.
			}
			\label{fig:frac}
		\end{figure}
	\end{columns}
\end{frame}

\begin{frame}{Off-centered thermal spike}
	\begin{columns}
	\column{0.5\textwidth}
		\begin{itemize}
			\item Off-centered thermal spikes are simulated as well,
				with variable degree of off-centeredness.
				By varying the amount of overlap,
				a spatial distribution can be generated.
			\item At an off-center distance of $r_c := R_{bubble} + R_{spike}$,
				the re-solution stops.
			\item Normalized fraction of re-solved Xe atoms
				as a function of normalized off-centered distance
				can be modeled with a logistic equation:
				$
					\label{eq:off}
					\frac{\chi}{\chi_0}
						= \frac{1.058}{1 + \exp \big[8.168
						\big(\frac{r}{r_c}\big) - 3.331 \big]}
				$
		\end{itemize}
	\column{0.5\textwidth}
		\begin{figure}[ht]
			\centering
			\includegraphics[height=4.5cm]{resol/images/offcentered.pdf}
			\caption{
				Xe re-solution caused by
				off-centered thermal spikes (15 keV/nm).
			}
			\label{fig:off}
		\end{figure}
	\end{columns}
\end{frame}

\begin{frame}{Effect of bubble pressure}
	\begin{columns}
	\column{0.5\textwidth}
		\begin{itemize}
			\item Xe/vac ratio determines bubble pressure.
				Bubbles with different Xe/vac ratios were simulated
				to evaluate the effect of pressure.
			\item The number of re-solved Xe atoms is apparently invariant
				with respect to Xe/vac ratio.
			\item With increasing thermal spike energy,
				more atoms are re-solved as expected.
			\item Similar trends were observed in bubbles
				with radii of 15 \r{A} and 35 \r{A}.
		\end{itemize}
	\column{0.5\textwidth}
		\begin{figure}[ht]
			\centering
			\includegraphics[width=\textwidth]{resol/images/xevac.pdf}
			\caption{
				Number of re-solved Xe atoms from 25 \r{A} radius bubbles
				as a function of Xe/vacancy ratio.
			}
			\label{fig:pres}
		\end{figure}
	\end{columns}
\end{frame}

\begin{frame}{Pressure and size invariance}
	\begin{columns}
	\column{0.5\textwidth}
		\begin{itemize}
			\item The number of re-solved atoms from identical-radius bubbles
				with different pressures were averaged.
			\item The number of re-solved atoms also appears consistent
				across bubbles of various sizes.
			\item Thermal spikes create low-density regions
				around the Xe gas bubble,
				and these regions facilitate the separation of Xe atoms
				from the bubble.
			\item Thermal spike energy dictates
				the volume of these low-energy regions,
				and thus affects the number of re-solved atoms.
		\end{itemize}
	\column{0.5\textwidth}
		\begin{figure}[ht]
			\centering
			\includegraphics[width=\textwidth]{resol/images/r2dep.pdf}
			\caption{
				Number of re-solved Xe atoms against bubble radius.
				Data from identical-radius bubbles
				with different Xe/vacancy ratios were averaged.
			}
			\label{fig:NvsRad}
		\end{figure}
	\end{columns}
\end{frame}

\begin{frame}{Calculation of re-solution rate}
	\begin{columns}
	\column{0.5\textwidth}
		\begin{itemize}
			\item We need to sum up all the contributions
				from all the fission products originating
				at different distances from a bubble and different offsets
				by means of a volume integral.
			\item The integral can be expressed explicitly
				in terms of fission product yields,
				and $\zeta = S_{e,eff} / S_{e}$,
				the fraction of FP energy that is imparted to the lattice.
			\item
				$
					f(R_{bubble}, \dot{F}) $
				$
					= \sum_{i=1}^2 \int_{x} \int_{r}
						\chi_i \dot{F} 2 \pi r dr dx $
				$
					= \dot{F} \sum_{i=1}^2 \int_{r}
						\bigg( \frac{\chi_i}{\chi_{0,i}} \bigg)
						2 \pi r dr \int_{x} \chi_{0,i} dx $
		\end{itemize}
	\column{0.5\textwidth}
		\begin{figure}[ht]
			\centering
			\includegraphics[height=5cm]{resol/images/coordSystem.pdf}
			\caption{
				Cylindrical coordinate system for calculating
				the heterogeneous re-solution rate.
			}
			\label{fig:coord}
		\end{figure}
	\end{columns}
\end{frame}

\begin{frame}{Electronic stopping power}
	\begin{columns}
	\column{0.5\textwidth}
		\begin{itemize}
			\item The electronic stopping power of the fission products (FPs)
				Xe and Sr in U-10Mo are calculated using SRIM.
			\item Ion irradiation is simulated 2000 times
				for each fission product.
			\item Equations are fit to the data, which can be integrated,
				allowing for the calculation of the re-solution rate.
			\item
				$
					S_{e,Xe} = 21.3 \exp(-0.239 x^{1.78})
						+ 5.23 \exp(-4.67 \times 10^{-8} x^{11}) $
			\item
				$
					S_{e,Sr} = 19.7 \exp(-0.00273 x^{3.71})
						+ 6.8 \exp(-0.424 x^{1.45}) $
		\end{itemize}
	\column{0.5\textwidth}
		\begin{figure}[ht]
			\centering
			\includegraphics[height=4.5cm]{resol/images/elec_stopping.pdf}
			\caption{
				Total electronic stopping power ($S_e$)
				of Xe-140 and Sr-94 in $\gamma$U-10Mo,
				as calculated by the SRIM software as a function of
				distance traversed by the fission product
				from the location of the fission reaction.
			}
			\label{fig:elec}
		\end{figure}
	\end{columns}
\end{frame}

\begin{frame}{Re-solution rate at nominal pressure}
	\begin{columns}
	\column{0.5\textwidth}
		\begin{itemize}
			\item Item 1
			\item Item 2
			\item Item 3
		\end{itemize}
	\column{0.5\textwidth}
		\begin{figure}[ht]
			\centering
			\includegraphics[width=6cm]{resol/images/resRate_withDart.pdf}
			\caption{
				An example caption!
			}
			\label{fig:resdart}
		\end{figure}
	\end{columns}
\end{frame}

\begin{frame}{Final re-solution rate}
	\begin{columns}
	\column{0.5\textwidth}
		\begin{itemize}
			\item With all this information, we can provide
				an equation for the re-solution rate,
				which account for both bubble size and pressure.
			\item
				$ b_{het}(R_{bubble}, \dot{F}, \phi) $
				$ = f(R_{bubble}, \dot{F}) \cdot g(\phi) $
				$ = \bigg[ \frac{a}{1+(R_{bubble}/c)^d} 10^{-14} \dot{F} \bigg]
					\bigg( \frac{0.2}{\phi} \bigg) $
			\item Here, $a$, $c$, and $d$  depend on the value of $\zeta$.
			\item Re-solution rate decreases with
				both bubble size and pressure.
		\end{itemize}
	\column{0.5\textwidth}
		\begin{figure}[ht]
			\centering
			\includegraphics[trim={4cm 2cm 1cm 2cm}, clip, width=\textwidth]
				{resol/images/3d.pdf}
			\caption{
				Xe gas bubble re-solution rate in $\gamma$U-10Mo
				as a function of bubble radius and Xe/vacancy ratio
				at a fission rate of $10^{14}$ fiss/cm$^3$/s for $\zeta=0.01$.
			}
			\label{fig:3d}
		\end{figure}
	\end{columns}
\end{frame}


\section{IUQ of \texorpdfstring{$\gamma$}{gamma}U-10Mo
fission-gas-behavior parameters}
\begin{frame}{IUQ}
	\begin{columns}
	\column{0.5\textwidth}
		\begin{itemize}
			\item Item 1
			\item Item 2
			\item Item 3
		\end{itemize}
	\column{0.5\textwidth}
		\begin{figure}[ht]
			\centering
			\includegraphics[width=5cm]{example-image-a}
			\caption{
				An example caption!
			}
			% \label{fig:template}
		\end{figure}
	\end{columns}
\end{frame}


\section{U-Mo moment tensor potential}
\begin{frame}{MTP}
	\begin{columns}
	\column{0.5\textwidth}
		\begin{itemize}
			\item Item 1
			\item Item 2
			\item Item 3
		\end{itemize}
	\column{0.5\textwidth}
		\begin{figure}[ht]
			\centering
			\includegraphics[width=5cm]{example-image-a}
			\caption{
				An example caption!
			}
			% \label{fig:template}
		\end{figure}
	\end{columns}
\end{frame}


\section{Conclusions}
\begin{frame}{Conclusions}
	\begin{columns}
	\column{0.5\textwidth}
		\begin{itemize}
			\item Item 1
			\item Item 2
			\item Item 3
		\end{itemize}
	\column{0.5\textwidth}
		\begin{figure}[ht]
			\centering
			\includegraphics[width=5cm]{example-image-a}
			\caption{
				An example caption!
			}
			% \label{fig:template}
		\end{figure}
	\end{columns}
\end{frame}

\begin{frame}{Future work}
	\begin{columns}
	\column{0.5\textwidth}
		\begin{itemize}
			\item Item 1
			\item Item 2
			\item Item 3
		\end{itemize}
	\column{0.5\textwidth}
		\begin{figure}[ht]
			\centering
			\includegraphics[width=5cm]{example-image-a}
			\caption{
				An example caption!
			}
			% \label{fig:template}
		\end{figure}
	\end{columns}
\end{frame}


\section*{Thanks}
\begin{frame}
	\centering \Large
	\emph{Thank you!}
\end{frame}


\end{document}
