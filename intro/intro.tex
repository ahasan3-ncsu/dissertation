\chapter{Introduction}

\section{Motivation}

Research reactors (RRs) are nuclear facilities
that act as high neutron and/or gamma radiation sources
for use in training and education, production of radioisotopes,
material testing and characterization, transmutation, activation
or other irradiation services \cite{martens1967, mishra2023}.
% the sentence below may be incorrect
In general, RRs operate close to atmospheric pressure
and at relatively low temperatures
(typically with a fuel temperature of less than $250^{\circ}$C
and a coolant temperature of less than $50^{\circ}$C,
compared to a fuel temperature of $1000$--$1500^{\circ}$C
in $350^{\circ}$C coolant for power reactors)
\cite{couturier2019, mishra2023}.
However, RRs normally require fuel
that is much more enriched in U-235 than the fuel used in power reactors,
and have historically employed highly enriched uranium (HEU)
for their operation.
HEU fuels are ususally defined
as those containing greater than $20 \%$ U-235 content,
whereas low enriched uranium (LEU) fuels contain less than 20\% U-235 content
\cite{keiser2003}.

Early research reactor fuels included monolithic alloys like U-Al,
chosen for their structural similarity to aluminum cladding.
However, to achieve higher uranium loadings required
for lengthening the reactor cycle, dispersion fuels were developed.
Dispersion fuels consist of fine fuel particles
dispersed within an inert matrix, typically aluminum or an aluminum alloy.
This design contains most fission products within the fuel particles,
with only a fraction escaping into the matrix by recoil \cite{jamison2022}.

Due to the stable performance of UO$_2$ in power reactors,
UO$_2$-Al was the first developed dispersion fuel,
offering high uranium loading than monolithic U-Al.
However, a significant reaction between UO$_2$ and Al during hot rolling
in the fabrication process posed challenges.
During irradiation, an interaction layer of UAl$_4$ and Al$_2$O$_3$ forms,
even at low temperatures.
To avoid this problem, the U$_3$O$_8$-Al dispersion fuel was evaluated,
and it showed better fabrication characteristics than UO$_2$-Al.
Thus, it was adopted for reactors
like the High Flux Isotope Reactor (HFIR) in the US.
Meanwhile, the UAl$_x$ dispersion fuel,
where UAl$_x$ represents a mixture of UAl$_2$, UAl$_3$, and UAl$_4$,
was adopted by reactors like
the Advanced Test Reactor (ATR) and the Belgian Reactor 2 (BR2)
due to its excellent performance,
although its uranium density is lower than
that of U$_3$O$_8$ \cite{jamison2022}.

With the inception of
the Reduced Enrichment for Research and Test Reactors (RERTR) program in 1978,
there had been an effort to replace
HEU fuels in high-power research reactors with LEU fuels \cite{travelli1980}.
This transition was driven by the need
to move away from weapons-grade nuclear material
and reduce proliferation risks.
To compensate for the lower enrichment,
a higher-density uranium fuel was required
to sustain sufficient power and flux levels in research reactors
\cite{snelgrove1997, wilson2020}.
U$_3$Si$_2$ was developed and qualified in the US for the RERTR program.
The U$_3$Si$_2$ dispersion fuel drastically increased the uranium density
compared to the oxide and UAl$_x$-based fuels.
It became the first LEU fuel intended for the conversion of HEU reactors,
and still has the highest uranium density among the qualified dispersion fuels.
Since the 1980s, the UAl$_x$ fuel has generally been replaced with U$_3$Si$_2$.
All fuels developed and used before U$_3$Si$_2$ were practically based on HEU
\cite{jamison2022, mishra2023}.

% TODO: mention figure
\begin{figure}
	\centering
	\includegraphics[width=12cm]{sfigs/historic_fuels.png}
	\caption{
		Uranium density of research reactor fuel
		vs time of first use.
		\textit{Reproduced from Jamison et al. \cite{jamison2022}.}
	}
\end{figure}

U$_3$Si$_2$ by itself is insufficient for the conversion
of High-Performance Research Reactors (HPRRs) from HEU to LEU based fuels.
This conversion is the aim
of the United States High-Performance Research Reactor (USHPRR) program.
HPRRs require fuel capable of operating at low temperatures
while maintaining high fission density at high specific power.
The fission density for a candidate fuel
falls between $\num{3e21}$ and $\num{6e21}$ fission/cm$^3$.
Only a few fuels, such as stable $\gamma$U alloys, combine high uranium density
with stable behavior at such high burnup levels.
The $\gamma$ phase of uranium has a body-centered cubic (bcc) structure,
avoiding the anisotropic swelling issues
seen in orthorhombic $\alpha$U \cite{hofman1990, mahbuba2021}.
Alloying $\gamma$ phase with molybdenum produces a metastable $\gamma$ phase,
which transforms sluggishly into equilibrium phases
(namely $\alpha$U and $\gamma$'U$_2$Mo)
upon cooling \cite{saller1955, dwight1960},
allowing $\gamma$ phase retention at lower temperatures
than expected from the phase diagram.
Additionally, $\gamma$U-Mo alloys can revert to the $\gamma$ phase
from equilibrium phases under irradiation \cite{meyer2014, willard1965},
offering a large region of $\gamma$ phase metastability
under reactor conditions.

Therefore, $\gamma$U-Mo alloys, particularly the ones with $7$--$10$ wt.\% Mo,
have emerged as promising candidates for LEU conversion of HPRRs
\cite{meyer2014}.
In the dispersion form, $\gamma$U-Mo alloy particles
are dispersed in an aluminum matrix, similar to other dispersion fuels.
However, irradiation testing revealed a pattern of breakaway swelling
at intermediate burnup, associated with the formation of
a molybdenum-stabilized high aluminum intermetallic phase.
To address the issues with dispersion fuel, the monolithic form was developed
to minimize the interfacial area
between the $\gamma$U-Mo fuel meat and the aluminum cladding.
A typical design involves a solid $\gamma$U-10Mo fuel meat
bonded to Al-6061 cladding with a zirconium diffusion barrier
to improve bonding and prevent undesirable interactions
\cite{jamison2022, mishra2023}.
Consequently, the USHPRR program has selected
the $\gamma$U-10Mo monolithic fuel for qualification
\cite{robinson2009, miller2021, cole2016, wilson2019}.

% TODO: mention figure
\begin{figure}
	\centering
	\includegraphics[width=12cm]{sfigs/umo_design.jpg}
	\caption{
		Depiction of monolithic fuel cross-section (not to scale).
		\textit{Reproduced from Meyer et al. \cite{meyer2014}.}
	}
\end{figure}

There exists a knowledge gap in the microstructural properties of $\gamma$U-Mo,
which constrains the evaluation of its fuel performance.
This work aims to bridge this gap using computational methods,
enhancing the accuracy of engineering-scale modeling of $\gamma$U-Mo.

\section{Literature Review}

\subsection{\texorpdfstring{$\gamma$}{gamma}U-Mo Irradiation Behavior}

% TODO: mention figure
\begin{figure}
	\centering
	\includegraphics[width=14cm]{sfigs/umo_burnup_2.png}
	\caption{
		Scanning electron micrographs showing
		evolution of fuel microstructure with burnup.
		The darkest spots are fission gas bubbles.
		The images were taken from the fracture surfaces
		of irradiated U-10Mo fuel samples.
		\textit{Reproduced from Kim et al. \cite{kim2011}.}
	}
\end{figure}

Under irradiation, $\gamma$U-Mo fuel undergoes
a series of microstructural evolution that impact its performance.
Initially, at low fission densities
of less than $2.5$--$\num{3.5e21}$ fissions/cm$^3$
or a burnup of $6$--$7.2 \%$ fission per initial metal atoms (FIMA)
and temperatures typically below $250^{\circ}$C,
fuel swelling exhibits a quasi-linear trend \cite{kim2011, ye2015}.
The initial swelling is primarily attributed to
the accumulation of fission products
either in solid solution within the fuel lattice or gas bubbles.
A key feature during this early stage
is the formation of an ordered array of nanobubbles,
known as the gas bubble superlattice (GBS).
$\gamma$U-Mo is the only nuclear fuel in which a GBS has been observed to form
\cite{salvato2024}.
This GBS is typically a face-centered cubic (fcc) superlattice
registered on the underlying
body-centered cubic (bcc) $\gamma$U-Mo crystal lattice,
with nanobubbles around 4 nm in diameter
and a lattice parameter of approximately 12 nm \cite{salvato2024, meyer2014}.
The concentration of fission gas in the GBS
progressively increases with fission density,
pointing to its ability to retain fission gases.

As irradiation continues and the fuel reaches higher fission densities,
a significant microstructural transformation---grain refinement
or recrystallization---occurs.
Grain refinement refers to the conversion of the original, larger grains
to a multitude of much smaller grains \cite{leenaers2016}.
This leads to the development of the high burnup structure (HBS) of the fuel,
which is also observed in other fuels, such as UO$_2$, MOX, UC, and UN
\cite{rondinella2010}.
The HBS is characterized by
the formation of sub-micron sized grains ($200$--$300$ nm)
and a large number of intergranular pores.
The GBS collapses at this stage, and most of the fission gas inventory
collects into the relatively large intergranular bubbles
that nucleate at the grain boundaries of the newly formed refined grains.
HBS formation is common in fuel types operating at temperatures
low enough to reduce the probabilities of defects to anneal
\cite{salvato2024, leenaers2016}.
In $\gamma$U-Mo, HBS is usually first observed
at the high-angle grain boundaries.
The driving mechanisms for grain refinement
are proposed to be polygonization initially,
followed by dynamic recrystallization at a critical local fission density,
marked by the transformation of low-angle to high-angle grain boundaries.
The formation of HBS leads to accelerated fuel swelling
since the increased grain boundary area per unit volume
in the recrystallized fuel
enhances swelling mediated by fission gas bubble \cite{salvato2024}.

\subsection{Diffusion in \texorpdfstring{$\gamma$}{gamma}U-Mo}

Knowledge of microstructure evolution under irradiation
is crucial for designing nuclear fuels.
For $\gamma$U-Mo, fission gas bubble formation and its impact on fuel swelling
need to be quantified to ensure predictable fuel performance.
Mechanistic models are being developed to evaluate
microstructure-based fuel performance,
such as fuel swelling, fuel creep,
and degradation of mechanical and thermal properties
based on fuel parameters and irradiation conditions.
Accurate calculation of fuel swelling requires
diffusion coefficients of the related species in the fuel.
Furthermore, creep modeling also requires diffusion coefficients
to determine creep rates and evaluate the evolving microstructure.
Therefore, it is essential to understand
the diffusion behavior of the $\gamma$U-Mo fuel.

The bulk diffusion coefficients of U, Mo, and Xe in $\gamma$U-Mo fuel
have already been calculated \cite{smirnova2015, park2021}
or measured \cite{huang2013}.
However, the diffusion coefficients of the relevant species
in $\gamma$U-Mo grain boundaries (GBs) are yet unknown.
As a consequence, models of fuel swelling, gas bubble evolution,
irradiation creep, and other properties needed for fuel performance evaluation
either utilize estimated GB diffusion values
or make the diffusion coefficients adjustable parameters.
The current assumptions of the GB diffusion coefficients
are $10^2$ to $10^7$ times greater than
the bulk diffusion coefficients \cite{annualreport2021, ye2015}.
This much uncertainty in the estimated GB diffusion coefficients
can have a significant impact
on the predicted fuel swelling behavior \cite{annualreport2022}.
The fuel swelling model of $\gamma$U-Mo
in the Dispersion Analysis Research Tool \cite{dart}
needs the GB diffusion coefficients
to advise the GB enhancement factor \cite{cui2015, annualreport2021}.
The phase-field models of gas bubble evolution in $\gamma$U-Mo
require the diffusivity of gas atoms \cite{hu2021, annualreport2021}.
The GB diffusion coefficients are also needed for the irradiation creep model
of the $\gamma$U-Mo fuel \cite{annualreport2022}.

In the literature, there are many examples
of the use of molecular dynamics (MD) to compute GB diffusion coefficients
in nuclear fuels.
Vincent-Aublant et al. \cite{vincent2009} calculated
the self-diffusion of UO$_2$ near GBs using MD.
Govers et al. investigated
GB diffusion in nano-polycrystalline UO$_2$ \cite{govers2013}.
Nishina et al. studied the GB diffusion of actinides and oxygen in oxide fuels
using MD \cite{nishina2011}.
MD has also been used by Beeler et al. to compute GB energies in $\gamma$U-Mo
and U$_3$Si$_2$ fuels \cite{beeler2018gb, beeler2019}.
Besides nuclear fuels, MD has also been used to study
GB diffusion in other bcc materials, such as bcc iron \cite{yang2018},
bcc tungsten \cite{fu2021}, etc.

\subsection{Fission Gas Bubble Re-Solution}

To understand the fuel's behavior under irradiation,
meso-scale and engineering-level fuel performance models require
knowledge of the fundamental mechanistic behavior of fission products
within the fuel to describe key phenomena,
such as swelling \cite{beeler2018gb, annualreport2021}.
Specifically, understanding the progression of Xe gas bubbles in the fuel
is crucial for optimizing reactor performance and safety.
These Xe gas bubbles act as a sink for individual Xe atoms,
trapping them and causing the bubbles to grow after absorption.
Under irradiation, the Xe atoms in the gas bubble are reintroduced
into the fuel matrix through fission-product-induced cascades
and thermal spikes---a process known as re-solution.
The relative rates of the re-solution affect the overall size and density
of the bubbles \cite{ye2023, olander2006re, parfitt2008},
in turn impacting bubble evolution and subsequent fuel swelling.

Re-solution of fission gas in nuclear fuels involves
two commonly accepted mechanisms:
homogeneous re-solution and heterogeneous re-solution \cite{olander2006re}.
In the homogeneous model proposed by Nelson \cite{nelson1968},
atoms from the gas bubbles are ejected individually
through collisions with fission products
or the recoil atoms that traverse the bubbles.
These atomic collision cascades are primarily
governed by the nuclear stopping power of the material.
In the heterogeneous model proposed by Turnbull \cite{turnbull1971},
a portion of gas bubbles are dissolved
by a passing fission fragment (FF) in the vicinity.
The driving mechanism is
the local heating of the material containing the gas bubbles,
through the electronic stopping of the FFs \cite{setyawan2018}.
The fact that both these mechanisms occur on a short timescale
makes it challenging to conduct experiments for determining re-solution rates
that contribute to the fission gas release models.
Thus, atomistic-scale modeling is necessary
for determining the re-solution rate
and for elucidating the fundamental mechanism
behind re-solution in U-10Mo.

In the literature up to this point,
atomistic simulations have been widely used
to evaluate the re-solution rate in various nuclear materials.
For instance, in 2008, Parfitt et al. \cite{parfitt2008}
used simulations of primary knock-on atoms (PKAs) in uranium dioxide (UO$_2$)
to assess the re-solution of helium gas bubbles.
In 2009, Schwen et al. \cite{schwen2009md} investigated
the homogeneous re-solution of Xe gas bubbles in UO$_2$,
using binary collision approximation (BCA) and molecular dynamics (MD).
The following year, Huang et al. \cite{huang2010md}
examined the impact of thermal spikes on Xe re-solution in UO$_2$.
In 2012, Govers et al. \cite{govers2012}
performed PKA and thermal spike simulations of Xe gas bubbles in UO$_2$
and proposed a mathematical model for the re-solution rate.
However, the most comprehensive work on Xe gas bubble re-solution in UO$_2$
was conducted in 2018 by Setyawan et al. \cite{setyawan2018}.
They reconciled the inconsistencies found in the conclusions of previous works
on Xe bubble re-solution in UO$_2$
and evaluated the re-solution rate as a function of bubble radius.
Their findings suggest that heterogeneous re-solution of gas bubbles
is the dominant method of re-solution in UO$_2$.
In addition to UO$_2$,
the re-solution rate of fission gas bubbles has also been evaluated
in uranium carbide (UC) by Matthews et al. \cite{matthews2015} in 2015
and in uranium zirconium (U-Zr) alloys by Mao et al. \cite{mao2025} in 2025,
using BCA.
Thermal spikes are not expected to occur in UC and U-Zr systems
due to higher electronic conductivity and thermal diffusivity
compared to UO$_2$ \cite{matthews2015, ronchi1986, mao2025}.
Thus, only heterogeneous re-solution has been studied in UC and U-Zr.
In summary, BCA and MD simulations were both used
to determine the re-solution rate in nuclear fuels.

In MD simulations of homogeneous re-solution,
a regular lattice atom is typically endowed with high kinetic energy
to emulate a PKA.
The PKA then interacts ballistically with other atoms,
initiating a collision cascade near the gas bubble
and inducing disorder \cite{parfitt2008, govers2012}.
One alternative approach in MD is to simulate
only a portion of the cascade (i.e., a subcascade)
by imparting energy to a random gas atom within the bubble.
In doing so, the simulation avoids unnecessary cascade events
that may not significantly influence the re-solution process.
However, BCA must be utilized in this approach
to first obtain an energy spectrum of the gas atom PKAs \cite{schwen2009md}.
One challenge in using MD simulations to model homogeneous re-solution
is the channeling of PKAs or their recoils over long distances,
without any collisions \cite{jarrin2021}.
This can make collecting statistics
on the interactions between PKAs and gas bubble atoms computationally demanding,
especially when the PKA direction is random.
A potential solution is to direct the PKAs
toward a high index lattice direction \cite{stoller2000}.
Moreover, collision cascades end up in heat spikes due to nuclear stopping.
This can lead to damage assisted re-solution
due to the high concentration of defects caused by these cascades.
To identify atoms that are re-solved ballistically, a threshold atomic speed,
above which it is improbable to find atoms in thermal equilibrium,
can be utilized \cite{parfitt2008}.

For MD simulations of heterogeneous re-solution due to swift heavy ions,
the thermal spike model is normally employed.
This model is useful for describing
the interaction between the FFs and the fuel.
These interactions occur primarily via
electronic stopping of the energetic particles
that initially raise the electronic subsystem temperature.
The energy deposited in the electronic subsystem can then transfer
to the lattice as thermal energy via electron-phonon coupling.
Finally, the energy is transferred among the atoms,
leading to a rapid increase in lattice temperature
within a cylindrical zone of typically a few nm in radius.
This increase is known as a thermal spike
\cite{wang1994, toulemonde2002, patra2019}.
In MD simulations, electronic interactions cannot be treated directly.
However, the thermal spike process can be emulated
either by raising the temperature of atoms within a cylindrical region
\cite{govers2012, setyawan2018}
or by coupling the two-temperature model (TTM) with MD
\cite{duffy2006, huang2010md}.

For qualification of U-10Mo fuel,
the ability to accurately predict the fission gas atom evolution
under various operational and transient conditions is crucial.
To that end, the Dispersion Analysis Research Tool (DART),
a meso-scale code developed by Argonne National Laboratory \cite{ye2023},
has been equipped with the ability to calculate
fission gas swelling in U-10Mo under different operational situations.
One of the many parameters required to model the swelling behavior
is the re-solution rate of fission gas bubbles.
DART employs a re-solution model
that includes a piecewise function to account for
both intergranular and intragranular bubbles.
The parameters in this function are calibrated
by fitting the computed swelling value to experimental data,
meaning that we only have a rough estimation of the re-solution rate.
A physics-based model of the re-solution rate of fission gas bubbles would make
the swelling calculations of higher-length-scale models more rigorous.

% FIX: this subsection needs renaming and rewriting
% we can probably go into a lot more details about the parameters
\subsection{Other Fission-Gas-Behavior Parameters}

Dispersion Analysis Research Tool (DART) is a computational code
capable of modeling the swelling behavior of Uranium-Molybdenum ($\gamma$U-Mo)
under the operating conditions of high-power research and test reactors (RTRs).
DART is based on a rate-theory-based mechanistic model for fission gas behavior
and a phase-field method for grain-size-specific recrystallization kinetics.
It was originally developed to simulate irradiation-induced phenomena
in dispersion fuels (U$_3$Si$_2$-Al and U-Mo/Al).
In 2018, a computational module for $\gamma$U-10Mo was added
to separate its calculations from those of dispersion systems.
The fission gas induced swelling, one of the main causes of fuel swelling,
is simulated using this module,
named Gas Release and Swelling Subroutine (GRASS).
GRASS tracks bubble nucleation, re-solution, and growth processes
both within the grains and on grain boundaries
by solving a series of non-linear differential equations.
The output from DART includes
total porosity, intra- and inter-granular bubble size distributions
at various locations (bulk, grain face, grain edge, and grain corner).
The DART code has been validated against experimental data
and has been used to study the effects of various operational and
microstructural parameters on fuel swelling \cite{annualreport2021, ye2023}.

% TODO: mention figure
\begin{figure}
	\centering
	\includegraphics[trim={0 14cm 0 0}, clip, width=14cm]
		{sfigs/dart_schematic.png}
	\caption{
		Schematic of intergranular gas bubble morphology
		and its evolution with fission density.
		\textit{Reproduced from Ye et al. \cite{ye2023}.}
	}
\end{figure}

The key challenge in simulating fission gas swelling with a mechanistic model
is obtaining key material properties related to gas bubble behavior,
as many of them cannot be measured experimentally
with the currently available techniques.
Some of the parameters can be calculated using atomic-scale simulation methods.
For the parameters that do not have measurement data
or atomic-scale simulation results,
they are usually estimated by either fitting to measured bubble morphology
or by borrowing from other similar fuel systems where the data is available.
The fission-gas-behavior parameters used in the GRASS module of DART
were calibrated in a previous study,
using the bubble size distributions
measured from irradiated $\gamma$U-10Mo dispersion fuel particles.
However, this set of parameters needs recalibration
because new atomic-scale data have become available
since the previous calibration \cite{annualreport2021, ye2023}.

% TODO: what fission-gas-behavior parameters?
% need to talk about them here; maybe an overview

% NOTE: should we have a DART subsection here?

\clearpage
\section{Objectives}

Based on the identified knowledge gaps,
this dissertation pursues the evaluation of microstructural properties
of the U-Mo monolithic fuel using various computational methods:

\begin{itemize}
\item
The diffusivities of U, Mo, and Xe in $\gamma$U-Mo GBs
are computed using classical MD simulations.
Different types of symmetric tilt, asymmetric tilt, and twist GBs
are utilized for the calculations.
The effect of fuel composition on diffusion is also examined.
The results from this study will inform
the rate-theory-based fuel swelling models,
the phase-field models of gas bubble evolution,
and the irradiation creep models.
The study of GB diffusion coefficients is described in detail
in \textbf{Chapter \ref{chap:diff}}.

\item
BCA and MD simulations are employed to investigate
the re-solution of Xe gas bubbles in the U-10Mo fuel%
---considering both the homogeneous and heterogeneous re-solution mechanisms
by simulating collision cascades and thermal spikes.
The behavior of representative fission products is profiled with BCA so as to
make the simulation of fission product and gas bubble interactions tractable,
whereas the two-temperature model (TTM) is utilized along with MD
to emulate thermal spikes.
Finally, an overall re-solution rate of Xe gas bubbles
as a function of bubble radius, bubble pressure, and fission rate is derived.
The specific details of this work are provided
in \textbf{Chapter \ref{chap:xeres}}.

\item
Inverse uncertainty quantification (IUQ) is performed to obtain
probability distributions of unknown fission-gas-behavior parameters
used in DART.
First, the data compiled from high-throughput DART calculations are visualized.
Afterward, surrogate models built with the data
is used for global sensitivity analysis.
Bayesian inference methods are then applied
to obtain the posterior probability distributions of the DART parameters.
Finally, the surrogate model are used again along with the parameter posteriors
to compute the forward propagation for validation.
The IUQ study is detailed in \textbf{Chapter \ref{chap:iuq}}.
\end{itemize}

Besides presenting technical results,
this dissertation also briefly discusses the necessary theoretical background
of the studies in \textbf{Chapter \ref{chap:theory}}.
Overall conclusions and recommended future work
feature in \textbf{Chapter \ref{chap:conclude}}.
