\chapter{Introduction}

\section{Motivation}

\section{Objectives}

\section{Literature Review}

\subsection{Diffusion in \texorpdfstring{$\gamma$}{gamma}U-Mo}

The United States High-Performance Research Reactor (USHPRR) program
is pursuing a replacement of highly enriched uranium (HEU) fuels
in high-power research reactors to low enriched uranium (LEU) fuels.
This effort originates from the necessity
of moving away from weapons-grade nuclear material
and minimizing potential proliferation threats.
To achieve this feat, a new fuel with increased uranium density is required
to offset the reduction in enrichment \cite{snelgrove1997, wilson2020},
while maintaining adequate power and flux levels in research reactors.

High-performance research reactors need fuel
that can operate at low temperatures
and provide high fission density at high specific power.
The fission density requirement of a candidate fuel
is in the range of $3 \times 10^{21}$ fission/cm$^3$
to $6 \times 10^{21}$ fission/cm$^3$.
Only a few fuels, such as stable $\gamma$U alloys,
have the suitable combination of high uranium density
and stable behavior at such a high burnup \cite{meyer2014}.
The $\gamma$ phase of uranium has a body-centered cubic (bcc) structure,
eliminating the problem of anisotropic swelling behavior
observed in orthorhombic $\alpha$U \cite{hofman1990, mahbuba2021}.
Alloying the $\gamma$ phase with molybdenum
leads to a metastable $\gamma$ phase that shows sluggish transformation
to the equilibrium phases (namely $\alpha$U and $\gamma$'U$_2$Mo)
upon cooling \cite{saller1955, dwight1960},
allowing for the presence of the $\gamma$ phase at lower temperatures
than predicted by the phase diagram.
Also, $\gamma$U-Mo alloys exhibit phase reversion
to the $\gamma$ phase from the equilibrium phases
under irradiation \cite{meyer2014, willard1965}.
$\gamma$U-Mo alloys thus provide the largest region
of $\gamma$ phase metastability under reactor conditions.
As a result, the USHPRR program is pursuing fuel designs with $\gamma$U-Mo
as the monolithic fuel foil with a zirconium diffusion barrier
in aluminum cladding \cite{robinson2009, cole2016, miller2021}.

Knowledge of microstructure evolution under irradiation
is crucial for designing nuclear fuels.
For $\gamma$U-Mo, fission gas bubble formation and its impact on fuel swelling
need to be quantified to ensure predictable fuel performance.
Mechanistic models are being developed to evaluate
microstructure-based fuel performance,
such as fuel swelling, fuel creep,
and degradation of mechanical and thermal properties
based on fuel parameters and irradiation conditions.
Accurate calculation of fuel swelling requires
diffusion coefficients of the related species in the fuel.
Furthermore, creep modeling also requires diffusion coefficients
to determine creep rates and evaluate the evolving microstructure.
Therefore, it is essential to understand
the diffusion behavior of the $\gamma$U-Mo fuel.

The bulk diffusion coefficients of U, Mo, and Xe in $\gamma$U-Mo fuel
have already been calculated \cite{smirnova2015, park2021}
or measured \cite{huang2013}.
However, the diffusion coefficients of the relevant species
in $\gamma$U-Mo grain boundaries (GBs) are yet unknown.
As a consequence, models of fuel swelling, gas bubble evolution,
irradiation creep, and other properties needed for fuel performance evaluation
either utilize estimated GB diffusion values
or make the diffusion coefficients adjustable parameters.
The current assumptions of the GB diffusion coefficients
are $10^2$ to $10^7$ times greater than
the bulk diffusion coefficients \cite{annualreport2021, ye2015}.
This much uncertainty in the estimated GB diffusion coefficients
can have a significant impact
on the predicted fuel swelling behavior \cite{annualreport2022}.
The fuel swelling model of $\gamma$U-Mo
in the Dispersion Analysis Research Tool \cite{dart}
needs the GB diffusion coefficients
to advise the GB enhancement factor \cite{cui2015, annualreport2021}.
The phase-field models of gas bubble evolution in $\gamma$U-Mo
require the diffusivity of gas atoms \cite{hu2021, annualreport2021}.
The GB diffusion coefficients are also needed for the irradiation creep model
of the $\gamma$U-Mo fuel \cite{annualreport2022}.

In the literature, there are many examples
of the use of molecular dynamics (MD) to compute GB diffusion coefficients
in nuclear fuels.
Vincent-Aublant et al. \cite{vincent2009} calculated
the self-diffusion of UO$_2$ near GBs using MD.
Govers et al. investigated
GB diffusion in nano-polycrystalline UO$_2$ \cite{govers2013}.
Nishina et al. studied the GB diffusion of actinides and oxygen in oxide fuels
using MD \cite{nishina2011}.
MD has also been used by Beeler et al. to compute GB energies in $\gamma$U-Mo
and U$_3$Si$_2$ fuels \cite{beeler2018gb, beeler2019}.
Besides nuclear fuels, MD has also been used to study
GB diffusion in other bcc materials, such as bcc iron \cite{yang2018},
bcc tungsten \cite{fu2021}, etc.

In this work, the diffusivities of U, Mo, and Xe in $\gamma$U-Mo GBs
are computed using classical MD simulations.
Different types of symmetric tilt, asymmetric tilt, and twist GBs
are utilized for the calculations.
The effect of fuel composition on diffusion is also examined.
The results from this study will inform
the rate-theory-based fuel swelling models,
the phase-field models of gas bubble evolution,
and the irradiation creep models.

\subsection{Fission Gas Bubble Re-solution}

A $\gamma$U-10Mo alloy-based monolithic fuel design was identified
as the fuel type for converting U.S. High-Performance Research Reactors (HPRRs)
\cite{meyer2014} from high enriched fuel to low enriched fuel.
To understand the fuel's behavior under irradiation,
mesoscale and engineering-level fuel performance models
require knowledge of the fundamental mechanistic behavior
of fission products within the fuel to describe key phenomena,
such as swelling \cite{beeler2018gb, annualreport2021}.
Specifically, understanding the progression of Xe gas bubbles in the fuel
is crucial for optimizing reactor performance and safety.
These Xe gas bubbles act as a sink for individual Xe atoms,
trapping them and causing the bubbles to grow after absorption.
Under irradiation, the Xe atoms in the gas bubble are reintroduced
into the fuel matrix through fission-product-induced cascades
and thermal spikes---a process known as re-solution.
The relative rates of the re-solution affect the overall size and density
of the bubbles \cite{ye2023, olander2006re, parfitt2008},
in turn impacting bubble evolution and subsequent fuel swelling.
Re-solution of fission gas in nuclear fuels involves
two commonly accepted mechanisms: homogeneous re-solution
and heterogeneous re-solution.
In homogeneous re-solution, atoms from the gas bubbles are ejected individually
through collisions with fission products
or the recoil atoms that traverse the bubbles.
These atomic collision cascades are primarily governed
by the nuclear stopping power of the material.
In the heterogeneous model, a portion of gas bubbles are dissolved
by a passing fission fragment in the vicinity.
The driving mechanism is the local heating of the material
containing the gas bubbles,
through the electronic stopping of the fission fragments \cite{setyawan2018}.
The fact that both these mechanisms occur on a short timescale
makes it challenging to conduct experiments for determining re-solution rates
that contribute to the fission gas release models.
Thus, atomistic-scale modeling is necessary
for determining the re-solution rate
and for elucidating the fundamental mechanism
behind re-solution in $\gamma$U-10Mo.

In the literature up to this point, atomistic simulations have been widely used
to evaluate the re-solution rate in various nuclear materials.
For instance, in 2008, Parfitt et al. \cite{parfitt2008} used simulations
of primary knock-on atoms (PKAs) in uranium dioxide (UO$_2$)
to assess the re-solution of helium gas bubbles.
In 2009, Schwen et al. \cite{schwen2009md} investigated
the homogeneous re-solution of Xe gas bubbles in UO$_2$,
using a binary collision model and molecular dynamics (MD) simulations.
The following year, Huang et al. \cite{huang2010md} examined
the impact of thermal spikes on Xe re-solution in UO$_2$.
In 2012, Govers et al. \cite{govers2012} performed
PKA and thermal spike simulations of Xe gas bubbles in UO$_2$
and proposed a mathematical model for the re-solution rate.
However, the most comprehensive work on Xe gas bubble re-solution in UO$_2$
was conducted  in 2018 by Setyawan et al. \cite{setyawan2018}.
They reconciled the inconsistencies found in the conclusions of previous works
on Xe bubble re-solution in UO$_2$
and evaluated the re-solution rate as a function of bubble radius.
Their findings suggest that heterogeneous re-solution of gas bubbles
is the dominant method of re-solution in UO$_2$.
In addition to UO$_2$, the re-solution rate of fission gas bubbles
was also evaluated in uranium carbide (UC)
by Matthews et al. \cite{matthews2015diss}, using binary collision methods.
The thermal spike model was not employed in UC
because it was assumed that the local heating does not exceed
the melting temperature \cite{matthews2015diss, ronchi1986}.
In summary, the binary collision model and MD simulations were used
to determine the re-solution rate in nuclear fuels.

In MD simulations of homogeneous re-solution,
a regular lattice atom is typically endowed with high kinetic energy
to emulate a PKA.
The PKA then interacts ballistically with other atoms,
initiating a collision cascade near the gas bubble
and inducing disorder \cite{parfitt2008, govers2012}.
One alternative approach in MD is to simulate only a portion of the cascade
(i.e., a subcascade) by imparting energy
to a random gas atom within the bubble.
In doing so, the simulation avoids unnecessary cascade events
that may not significantly influence the re-solution process.
However, a binary collision model must be utilized in this approach
to first obtain an energy spectrum of the gas atom PKAs \cite{schwen2009md}.
One challenge in using MD simulations to model homogeneous re-solution
is the channeling of PKAs or their recoils over long distances,
without any collisions \cite{jarrin2021}.
This can make collecting statistics on the interactions
between PKAs and gas bubble atoms computationally demanding,
especially when the PKA direction is random.
A potential solution
is to direct the PKAs toward a high index lattice direction \cite{stoller2000}.
Moreover, collision cascades end up in heat spikes due to nuclear stopping.
In that regard, the simulation of PKAs
also encompasses the heterogeneous re-solution.
To identify atoms that are re-solved ballistically, a threshold atomic speed,
above which it is improbable to find atoms in thermal equilibrium,
can be utilized \cite{parfitt2008}.
For MD simulations of heterogeneous re-solution due to swift heavy ions,
the thermal spike model is normally employed.
This model is useful for describing the interaction
between the fission fragments and the fuel.
These interactions occur primarily
via electronic stopping of the energetic particles
that initially raise the electronic subsystem temperature.
The energy deposited in the electronic subsystem can then transfer
to the lattice as thermal energy via electron-phonon coupling.
Finally, the energy is transferred among the atoms,
leading to a rapid increase in lattice temperature within a cylindrical zone
of typically a few nm in radius.
This increase is known as
a thermal spike \cite{wang1994, toulemonde2002, patra2019}.
In MD simulations, electronic interactions cannot be treated directly.
However, the final step described above can be emulated
by raising the temperature of atoms within a cylindrical region.

% Normally, heat spikes and thermal spikes are used interchangeably. However,
% in this work, we use "heat spike" to mean the spike due to nuclear stopping
% at the end of collision (sub)cascades. The word "thermal spike" is reserved
% for describing local heating due to heavy electronic stopping.

For qualification of $\gamma$U-Mo fuel,
the ability to accurately predict the fission gas atom evolution
under various operational and transient conditions is crucial.
The Dispersion Analysis Research Tool (DART),
developed by Argonne National Laboratory \cite{ye2023},
is a mesoscale code that can calculate fission gas swelling
in $\gamma$U-Mo under different operational situations.
One of the many parameters required to model swelling behavior
is the re-solution rate of fission gas bubbles.
DART employs a simple re-solution model
that includes a piecewise function to account for the bubble radius.
The parameters in this function are calibrated
by fitting the computed swelling value to experimental data,
meaning that we can only roughly estimate the re-solution rate.
A physics-based re-solution rate for fission gas bubbles would make
the swelling calculations of higher-length-scale models more rigorous.

In the present study, we utilized MD simulations to investigate
the re-solution of Xe gas bubbles in $\gamma$U-Mo fuel---considering
both the homogeneous and heterogeneous re-solution mechanisms
by simulating PKAs and thermal spikes.
We also calculated the electronic stopping power
of representative fission products
so as to determine the energy imparted on gas bubbles
by fission events at a certain distance.
Finally, we evaluated an overall re-solution rate of Xe gas bubbles
as a function of bubble radius, bubble pressure, and fission rate.

\subsection{Inverse Uncertainty Quantification}

% TODO: need to make it Inverse UQ review

Dispersion Analysis Research Tool (DART) is a computational code
capable of modeling the swelling behavior of Uranium-Molybdenum ($\gamma$U-Mo)
under the operating conditions of high-power research and test reactors (RTRs).
DART is based on a rate-theory-based mechanistic model for fission gas behavior
and a phase-field method for grain-size-specific recrystallization kinetics.
The DART code has been validated against experimental data
and has been used to study the effects of various operational and
microstructural parameters on fuel swelling \cite{annualreport2021, ye2023}.

The key challenges in simulating fuel swelling with a mechanistic model
are obtaining key material properties related to gas bubble behavior,
as many of them cannot be measured experimentally
with the currently available techniques.
Some of the parameters can be calculated using atomic-scale simulation methods.
For the parameters that do not have measurement data
or atomic-scale simulation results,
they are usually estimated by either fitting to measured bubble morphology
or by borrowing from other similar fuel systems where the data is available.
The fission-gas-behavior parameters used in the GRASS module of DART
were calibrated in a previous study,
using the bubble size distributions
measured from irradiated $\gamma$U-10Mo dispersion fuel particles.
However, this set of parameters needs recalibration
because new atomic-scale data have become available
since the previous calibration \cite{annualreport2021, ye2023}.

Inverse uncertainty quantification (IUQ) is a process
to quantify uncertainties in the input parameters of a computer model,
such as DART, given experimental data.
IUQ is an essential step in computational model validation
because it provides a concrete and quantifiable measure of uncertainty
in model predictions
\cite{wu2018inverse, nagel2019bayesian, wu2021comprehensive}.
The concept of IUQ is in contrast to forward uncertainty quantification (FUQ),
which quantifies the uncertainty in the output of a computer model
given the input parameters.
FUQ is often used to predict the uncertainty in the results of a simulation,
while IUQ is used to quantify the uncertainty in the model itself.
IUQ is often used in engineering and scientific applications
where there is a high degree of uncertainty in the input parameters.
By quantifying the uncertainty in the input parameters,
IUQ can help to improve the accuracy and reliability
of the model predictions \cite{wu2021comprehensive, xie2024functional}.

In this work, IUQ is performed to obtain probability distributions
of a few fission-gas-behavior parameters used in DART.
First, the data compiled from some DART calculations is visualized.
Afterward, surrogate models are built using the data.
A variety of linear and non-linear models are used for this purpose.
Afterward, Bayesian inference methods are applied
to find the posterior probability distributions of the DART parameters.
Finally, the surrogate model is used again along with the parameter posteriors
to compute the forward propagation as FUQ and validation.
