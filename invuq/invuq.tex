\chapter{Inverse UQ}

\section{Surrogate modeling}

Figure \ref{fig:linear} depicts the surrogate prediction vs actual test data
for three linear surrogate models: OLS, Lasso, and Ridge.
The performance of the linear surrogate models are assessed
with the test data.
The R$^2$ scores for all of them are at least 0.997,
and the root mean squared errors (RMSEs)
and the mean absolute errors (MAEs)
are about 0.25 and 0.17 swelling\%, respectively.
Table \ref{tab:surrogates} details a few performance metrics
for all the surrogate models.
Among the three linear surrogate models,
the Lasso model provided the most noteworthy insight.
The regression coefficients for the parameters
"vResol", "DatomFissGBx", "fNucleate", and "aAtomDifFiss"
are equal to 0 in the Lasso surrogate.
Therefore, it is possible to ignore these four parameters
without losing any accuracy in the prediction.

\begin{figure}[!ht]
\begin{subfigure}{0.32\textwidth}
	\centering
	\includegraphics[width=\textwidth]{invuq/images/ols.png}
\end{subfigure}
\begin{subfigure}{0.32\textwidth}
	\centering
	\includegraphics[width=\textwidth]{invuq/images/lasso.png}
\end{subfigure}
\begin{subfigure}{0.32\textwidth}
	\centering
	\includegraphics[width=\textwidth]{invuq/images/ridge.png}
\end{subfigure}
\caption{
	Surrogate prediction vs actual test data for linear surrogates.
}
\label{fig:linear}
\end{figure}

The prediction vs true test data for three non-linear surrogate models:
GP, NN, and SVR, are shown in Figure \ref{fig:nonlin}.
The NN and SVR are trained with 7 parameters
and 70\% of the total data (the train split),
similar to the linear models.
The GP model is trained with only 3 significant parameters
(as identified by Lasso) and 14\% of the total data (448 samples).
However, this model is evaluated with the same test data as others.
The rationale behind using fewer training samples for the GP model
is to reduce the computational cost.
It is apparent from Figure \ref{fig:nonlin}
that the GP model predict swelling values with high accuracy.
This is also reflected by its higher R$^2$ score
and lower RMSE and MAE scores than other models.
As a result, the GP model is utilized in the MCMC sampling for IUQ.

\begin{figure}[!ht]
\begin{subfigure}{0.32\textwidth}
	\centering
	\includegraphics[width=\textwidth]{invuq/images/gp.png}
\end{subfigure}
\begin{subfigure}{0.32\textwidth}
	\centering
	\includegraphics[width=\textwidth]{invuq/images/nn.png}
\end{subfigure}
\begin{subfigure}{0.32\textwidth}
	\centering
	\includegraphics[width=\textwidth]{invuq/images/svr.png}
\end{subfigure}
\caption{
	Surrogate prediction vs actual test data for non-linear surrogates.
}
\label{fig:nonlin}
\end{figure}

\begin{table}[ht]
\centering
\caption{
	Comparison of surrogate performance on test data.
}
\label{tab:surrogates}
\begin{tabular}{lccc}
\toprule
Surrogate Model           & R$^2$   & RMSE  & MAE   \\
\midrule
Ordinary Least Squares    & 0.99708 & 0.253 & 0.174 \\
Lasso                     & 0.99706 & 0.255 & 0.174 \\
Ridge                     & 0.99708 & 0.253 & 0.174 \\
Gaussian Processes        & 0.99981 & 0.065 & 0.044 \\
Neural Network            & 0.99366 & 0.374 & 0.279 \\
Support Vector Regression & 0.99365 & 0.374 & 0.172 \\
\bottomrule
\end{tabular}
\end{table}


\section{MCMC sampling}

Figure \ref{fig:trace} displays the trace plots of the MCMC samples
of the three fission-gas-behavior parameters.
Two chains of 100,000 samples are overlaid on the trace plots.
The cumulative averages of the blue and orange traces
are shown in black and red, respectively.
After an initial period, the cumulative averages of both chains converge.
The density plots of the parameter posterior samples
are shown in Figure \ref{fig:hist}.
Again, the density plots from both chains show similar trends.
The posterior of "dGrainHBS" is reminiscent of a Gaussian distribution,
whereas the posteriors of the other two parameters
are more akin to a uniform distribution.

\begin{figure}[ht]
	\centering
	\includegraphics[width=\textwidth]{invuq/images/trace.png}
	\caption{
		Trace plots of the posterior samples
		of the fission-gas-behavior parameters from MCMC sampling.
	}
	\label{fig:trace}
\end{figure}

\begin{figure}[ht]
	\centering
	\includegraphics[width=\textwidth]{invuq/images/hist.png}
	\caption{
		Density plots of the posterior samples
		of the fission-gas-behavior parameters from MCMC sampling.
	}
	\label{fig:hist}
\end{figure}

The acceptance rate of the MCMC sampling was about 24\% for both chains,
which is close to the optimal value for mixing
for a random walk Metropolis algorithm.
To be sure of the convergence of the MCMC samples,
the Gelman-Rubin statistic and the effective sample size (ESS)
of the MCMC samples are also calculated and listed in Table \ref{tab:conv}.
The Gelman-Rubin statistic
is either 1 or very close to 1 for the posterior samples,
which indicates that no convergence issues were detected.
For all three parameters, the ESS is about 2000,
which is sufficient for stable estimates.
The MCMC samples were thus thinned by picking every 100th value
to reduce auto-correlation.
The remaining 2000 posterior samples are investigated for the IUQ process.

\begin{table}[ht]
\centering
\caption{
	Convergence diagnostics for the MCMC samples.
}
\label{tab:conv}
\begin{tabular}{lccc}
\toprule
Statistic              & dGrainHBS & FaceCovMax & SwellLink \\
\midrule
Effective Sample Size  & 2019      & 1878       & 2122      \\
Gelman-Rubin statistic & 1.0       & 1.0        & 1.001     \\
\bottomrule
\end{tabular}
\end{table}


\section{IUQ}

The diagonal subfigures in Figure \ref{fig:iuq}
shows the posterior distributions of the fission-gas-behavior parameters
after the IUQ process.
The distributions in this figure are from the thinned samples
unlike the distributions shown in Figure \ref{fig:hist}.
Thinning the samples did not change the characteristics of the distributions.
The means, standard deviations, and the 95\% credible intervals
of the posterior distributions are reported in Table \ref{tab:iuq}.

Besides the posteriors,
the correlations among the posteriors are also displayed
as contour plots in the off-diagonal of Figure \ref{fig:iuq}.
Even though the prior distributions of the parameters
are assumed to be independent,
Bayesian IUQ can still identify the correlations
among the parameter posteriors through MCMC sampling.
For instance, the posteriors of "dGrainHBS" and "FaceCovMax"
have a strong positive correlation ($\rho = 0.9$),
and the posteriors of "dGrainHBS" and "SwellLink"
have a weak positive correlation ($\rho = 0.25$).
Consequently, the correlation among the parameter posteriors
need to be taken into account when generating new samples.

\begin{figure}[ht]
	\centering
	\includegraphics[width=0.8\textwidth]{invuq/images/iuq.png}
	\caption{
		Posterior distributions of 3 fission-gas-behavior parameters.
		Marginal densities are in the diagonal
		and the pair-wise joint densities are in the off-diagonal.
	}
	\label{fig:iuq}
\end{figure}

\begin{table}[ht]
\centering
\caption{
	Posterior means, standard deviations, and 95\% credible intervals
	for all the fission-gas-behavior parameters.
}
\label{tab:iuq}
\begin{tabular}{lccc}
\toprule
Parameter  & Mean
		   & Std. deviation
		   & 95\% credible interval               \\
\midrule
dGrainHBS  & \num{3.72e-5}
		   & \num{3.42e-6}
		   & {[} \num{3.11e-5}, \num{4.37e-5} {]} \\
FaceCovMax & \num{7.57e-1}
		   & \num{8.82e-2}
		   & {[} \num{6.09e-1}, \num{8.99e-1} {]} \\
SwellLink  & \num{2.53e-2}
		   & \num{5.22e-3}
		   & {[} \num{1.66e-2}, \num{3.37e-2} {]} \\
\bottomrule
\end{tabular}
\end{table}


\section{FUQ and validation}

To validate the results from IUQ,
we propagated the posterior samples using two surrogate models: GP and Lasso.
The results from the forward propagation are shown in Figure \ref{fig:fuq}.
Both surrogate models predict
fuel swelling aligning closely with the experimental data,
even though IUQ was performed only with the GP surrogate.
This process does not lead to a proper validation of the IUQ results
since we are using the same surrogate for both IUQ and FUQ.
A more rigorous validation process would be generating new samples
from the joint posteriors of the parameters
and propagating them directly through DART.

\begin{figure}[!ht]
\begin{subfigure}{0.49\textwidth}
	\centering
	\caption{GP}
	\includegraphics[width=\textwidth]{invuq/images/fuq_gp.png}
\end{subfigure}
\begin{subfigure}{0.49\textwidth}
	\centering
	\caption{Lasso}
	\includegraphics[width=\textwidth]{invuq/images/fuq_las3.png}
\end{subfigure}
\caption{
	Forward propagation using (a) GP, and (b) Lasso surrogate models.
}
\label{fig:fuq}
\end{figure}
