\chapter{Inverse Uncertainty Quantification
of U-10Mo Fission-Gas-Behavior Parameters}
\label{chap:iuq}

\section{Problem Definition}

The United State Department of Energy initiated the RERTR program in 1978
with the goal of developing the technology needed
to eliminate HEU in civilian applications.
The eponymous irradiation tests performed by the program thus focused on
high density LEU fuels, such as U-Mo.
In the RERTR-5 experiment,
fuel plates containing atomized or machined particles were supplied by
either the Korea Atomic Energy Research Institue (KAERI)
or the Atomic Energy Canada Limited (AECL).
The composition of the fuel alloys ranged from $6$ wt.\% Mo to $10$ wt.\% Mo.
The irradiation of the fuel plates was performed in the ATR in 2001
\cite{wachs2006}.
One of the irradiated plates was \textit{V6018G},
which had a burnup of $36 \%$ after $116$ effective full power days (EFPD)
of operation.
The end-of-life fission density in the fuel plate was
$F_d = \num{2.31e21}$ fsn/cm$^3$,
with a life-averaged fission rate density of $\num{2.3e14}$ fsn/cm$^3$/s
and a life-averaged fuel temperature of $116.3^{\circ}$ C \cite{kim2013}.
In RERTR-12,
U-Mo monolithic fuel plates with a Zr diffusion barrier were irradiated
to investigate the effects of fission density and fission rate on fuel behavior.
One of the plates investigated was \textit{L1P755},
which had a fission density of $\num{5.34e21}$ fsn/cm$^3$,
with a life-averaged fission rate and temperature
of $\num{6.86e14}$ fsn/cm$^3$/s and $133.6^{\circ}$ C \cite{rice2015}.

DART is a rate-theory-based meso-scale model
capable of predicting fuel swelling behavior
based on microstructural properties of the U-Mo fuel \cite{ye2023}.
However, the main obstacle in predicting fuel swelling is that
many of the microstructural properties of U-Mo are currently unknown.
Good estimates of these unknown parameters are thus needed
to provide accurate predictions of swelling necessary for fuel evaluation.
Simulations have been performed in DART
with $3,200$ different combinations of the parameter values
to predict the fuel swelling in the operational conditions
of the plates mentioned above.
The goal here is to find reasonable values of the uknown parameters
to match the experimentally observed swelling values
in the \textit{RERTR5-V6018G} and \textit{RERTR12-L1P755} fuel plates
of $10.7 \%$ and $32 \%$, respectively.
An analysis of fuel swelling in $74$ irradiated U-Mo test fuel plates
over a range of irradiation conditions
reveals that the measurement uncertainty in these experiments can be assumed
to have a standard deviation of $2.64 \%$
\cite{rabin2017preliminary, robinson2021}.
The observed swelling values, along with the measurement uncertainty,
are plotted in Figure \ref{fig:mvn_expt}.

\begin{figure}[ht]
	\centering
	\includegraphics[width=10cm]{iuq2/images/mvn_expt.pdf}
	\caption{
		Experimentally observed fuel swelling
		of \textit{RERTR5-V6018G} at $F_d = \num{2.32e21}$ fsn/cm$^3$
		and \textit{RERTR12-L1P755} at $F_d = \num{5.34e21}$ fsn/cm$^3$
		with a measurement uncertainty of $2.64 \%$.
	}
	\label{fig:mvn_expt}
\end{figure}

The unknown parameters perturbed in the DART simulations
are listed in Table \ref{tab:param}.
The table shows the name of the parameters, their reference values,
and their ranges.
The orders of magnitude of the parameters vary considerably.
The parameter \texttt{dGrainHBS} denotes the high burunup
structure grain diameter.
\texttt{FaceCovMax} means the fraction of grain face coverage
needed for interlinkage.
\texttt{SwellLink} determines the grain edge swelling starting interlinkage.
\texttt{rResolBulk} is the thickness of the destructed outer shell
in gas bubbles due to re-solution.
\texttt{DatomFissGBx} is the grain boundary diffusion enhancement factor.
\texttt{StickProb} is the probability that
two atoms stick to form a bubble nucleus.
\texttt{fNucleate} indicates the adjustment factor
for the probability of bubble nucleation on the grain boundary.
\texttt{vResol} is the probability that
a gas bubble interacts with fission fragments.
Finally , \texttt{rResolGBB} is the average re-solution distance
an atom is ejected from the grain boundary gas bubbles
\cite{annualreport2021, ye2023, annualreport2022}.

% TODO: can we put a unit column here?
\begin{table}[ht]
\centering
\caption{
	Fission-gas-behavior parameters and their ranges.
}
\label{tab:param}
\begin{tabular}{lccc}
\toprule
Parameter             & Minimum value & Maximum value & Reference value \\
\midrule
\texttt{dGrainHBS}    & \num{1e-5 }   & \num{1e-4 }   & \num{4e-5 }   \\
\texttt{FaceCovMax}   & \num{0.5  }   & \num{0.907}   & \num{0.907}   \\
\texttt{SwellLink}    & \num{0.01 }   & \num{0.08 }   & \num{0.025}   \\
\texttt{rResolBulk}   & \num{6e-10}   & \num{5e-9 }   & \num{3e-9 }   \\
\texttt{DatomFissGBx} & \num{6e3  }   & \num{5e4  }   & \num{3e4  }   \\
\texttt{StickProb}    & \num{2e-10}   & \num{4e-7 }   & \num{2e-7 }   \\
\texttt{fNucleate}    & \num{3e-10}   & \num{3e-9 }   & \num{6e-10}   \\
\texttt{vResol}       & \num{1e-18}   & \num{4e-18}   & \num{2e-18}   \\
\texttt{rResolGBB}    & \num{2e-7 }   & \num{1e-6 }   & \num{5e-7 }   \\
\bottomrule
\end{tabular}
\end{table}

\section{Methodology}

\subsection{Formulation of the Inverse Problem}

Let's define the unknown reality or true value of an output as $y^R(x)$,
where $x$ denotes the vector of design variables
specifying experimental conditions.
A computer model simulation can predict that reality only as an approximation:
\begin{align}
	\label{eq:real}
	\bm{y}^R(\bm{x}) = \bm{y}^M(\bm{x}, \bm{\theta}^*) + \delta(\bm{x})
\end{align}
where $\bm{\theta}^*$ is a vector of true but unknown values
of calibration parameters $\bm{\theta}$
and $\delta(\bm{x})$ is the model uncertainty/discrepancy
due to an incomplete understanding of the underlying physics of the model.

To learn the reality $\bm{y}^R(\bm{x})$,
experiments may be performed to obtain an observation $\bm{y}^E(\bm{x})$.
However, the measurement process also introduces uncertainty:
\begin{align}
	\label{eq:exp}
	\bm{y}^E(\bm{x}) = \bm{y}^R(\bm{x}) + \bm{\epsilon}
\end{align}
where $\bm{\epsilon} \sim \mathcal{N} (\bm{\mu}, \bm{\Sigma}_{exp})$
indicates the measurement error/noise.
It is typical to assume $\bm{\mu} = 0$
and $\bm{\Sigma}_{exp} = \sigma_{exp}^2 \bm{I}$ for experiments
having no systematic bias and having homoscedastic experimental errors.
Combining equations \ref{eq:real} and \ref{eq:exp},
we can obtain the ``model updating equation"
\cite{wu2021comprehensive, kennedy2001bayesian, arendt2012quant}:
\begin{align}
	\label{eq:upd}
	\bm{y}^E(\bm{x})
	= \bm{y}^M(\bm{x}, \bm{\theta}^*) + \delta(\bm{x}) + \bm{\epsilon}
\end{align}
Equation \ref{eq:upd} is the starting point for Bayesian IUQ.

Assuming experimental uncertainty is Gaussian,
$\bm{\epsilon}
= \bm{y}^E(\bm{x}) - \bm{y}^M(\bm{x}, \bm{\theta}^*) - \delta(\bm{x})$
follows a multi-dimensional Gaussian distribution
with a mean of zero and a covariance matrix of $\bm{\Sigma}$.
The posteriors of the true parameters $p(\bm{\theta}^* | \bm{y}^E, \bm{y}^M)$
can then be written as:
\begin{align}
	p(\bm{\theta}^* | \bm{y}^E, \bm{y}^M)
	&\propto p(\bm{\theta}^*) \cdot p(\bm{y}^E, \bm{y}^M | \bm{\theta}^*) \\
	&\propto p(\bm{\theta}^*) \cdot \frac{1}{\sqrt{|\bm{\Sigma}|}}
		\exp \bigg[-\frac{1}{2} (\bm{y}^E-\bm{y}^M-\delta)^T
		\bm{\Sigma}^{-1} (\bm{y}^E-\bm{y}^M-\delta) \bigg]
\end{align}
where $p(\bm{\theta}^*)$ is the prior distribution provided by expert opinion,
and $p(\bm{y}^E, \bm{y}^M | \bm{\theta}^*)$ is the likelihood function.
$\bm{\Sigma}$ is the covariance of the likelihood consisting of three parts:
\begin{align}
	\bm{\Sigma} &= \bm{\Sigma}_{exp} + \bm{\Sigma}_{bias} + \bm{\Sigma}_{code}
\end{align}
where $\bm{\Sigma}_{exp}$ represents experimental uncertainty
due to measurement error,
$\bm{\Sigma}_{bias}$ is the model uncertainty
due to an inherent bias in the model,
and $\bm{\Sigma}_{code}$ is the code/interpolation uncertainty
due to the use of surrogate models to reduce the computational cost.
In this work, $\bm{\Sigma}_{bias}$ and $\bm{\Sigma}_{code}$ are not considered
due to the limited amount of experimental data available for comparison
and the use of surrogate models that do not have inherent uncertainty measures.

\subsection{Surrogate Modeling}

Surrogate models, also called metamodels, response surfaces or emulators,
are approximations of the input/output relation of the original computer model.
They are built from a limited number of full model runs (training data)
and a learning algorithm.
Metamodels usually take much less computational time than the full model
while maintaining the input/output relation to a desirable accuracy.
Once validated, metamodels can be used in uncertainty, sensitivity,
validation and optimization studies,
for which the original computer model can incur an excessive computational burden
as hundreds or thousands of computer model simulations are needed
\cite{wu2018inverse}.

Any machine learning model that can learn from the training data
and predict an output relatively quickly compared to the original computer model
can be used as a surrogate model.
Typical examples of surrogate models include Moving Least-Squares (MLS),
Radial Basis Functions (RBF), Neural Networks (NN),
Support Vector Machines (SVM), Gaussian Process (GP),
Polynomial Chaos Expansion (PCE), etc.,
with GP being the most commonly used surrogate
\cite{wu2018inverse, wu2021comprehensive}.

Considering $X$ as an input matrix and $y$ as an output vector,
a linear regression model would have the following formulation:
\begin{align}
	y = \beta X + \varepsilon
\end{align}
where $\beta$ is the vector of unknown coefficients
and $\varepsilon$ is the vector of random errors.
In Ordinary Least-Squares (OLS), the estimated coefficients $\hat{\beta}$
are determined by minimizing the sum of squared residuals (the loss function).
\begin{align}
	\hat{\beta} = \arg \min_{\beta} || y - \beta X ||^2
\end{align}

GPs are a popular choice for surrogate models
because they provide a straightforward estimation of the code uncertainty.
A GP is a probabilistic model that describes
a distribution over possible functions.
Given a set of points, a GP can be used to predict
the value of an unknown function at any other point.
Additionally, GP can be used to represent the uncertainty
in the output of the computer model,
which is important for Bayesian inference and calibration
\cite{wu2018inverse, wang2020}.
The mathematical form of a GP model can be written as follows:
\begin{align}
	y = f^T(x) \beta + z(x)
\end{align}
where $y$ is the output prediction at a general location $x$.
The first term here is a linear regression of the data
modeling the drift of the mean.
The set of basis functions $f$ is chosen by the user
and $\beta$ are the regression coefficients.
$z(x)$ is a stationary Gaussian random process with zero mean and covariance
$Cov[z(x^{(i)}), z(x^{(j)})] = \sigma^2 \mathcal{R} (x^{(i)}, x^{(j)})$,
where $\sigma^2$ is the process variance
and $\mathcal{R} (\cdot, \cdot)$ is the correlation function or kernel.
A kernel in GP is a function that defines the similarity
between pairs of data points.
There are many different kernels that can be used with GP,
each of which has its own advantages and disadvantages.
Some of the most commonly used kernel functions include the RBF kernel,
the Mat\'ern kernel, and the rational quadratic kernel \cite{wang2020}.

NNs are a flexible class of machine learning models
inspired by biological neurons.
These models are highly effective
for approximating complex, high-dimensional functions.
NN models consist of layers of interconnected nodes, also called neurons,
that transform input data through a series of linear and non-linear operations.
The layers are usually differentiated as one \textit{input} layer,
\textit{hidden} layers with non-linear activation functions,
and one \textit{output} layer for regression.
Given a set of training data,
the network learns to recognize the underlying patterns
by adjusting the weights of the connections between the nodes
to minimize a predefined loss function.
This architecture allows NNs to serve as universal function approximators,
capable of capturing intricate dependencies that traditional models might miss
\cite{de2013neural, lecun2015, goodfellow2016}.
The mathematical operation of a single layer in a neural network 
can be written as follows:
\begin{align}
	y &= \sigma (W x + b)
\end{align}
where $x$ represents the input vector,
and $W$ is the weight matrix that determines
the strength of the connections between neurons.
$\sigma(\cdot)$ is a non-linear activation function,
and the vector $b$ is the bias, which allows the model
to shift the activation function to better fit the input.
The activation function in NNs defines the output of a node.
There are many choices for activation functions,
each of which affects the gradient flow and convergence of the model.
Some of the popular activation functions include
the sigmoid function, the rectified linear unit (ReLU),
and the hyperbolic tangent ($\tanh$) \cite{goodfellow2016}.

In this work, we have used GP and NN models as surrogates.
All the surrogate models have been implemented
using the scikit-learn Python library \cite{sklearn}.

\section{Results}

\subsection{Surrogate Modeling}

First, the data is visualized in the form of scatter plots
of fuel swelling against all the fission-gas-behavior parameters.
in Figures \ref{fig:scat_lo} and \ref{fig:scat_hi}.
At low $F_d$, only the parameter \texttt{rResolBulk}
seems to have a strong correlation with fuel swelling.
The swelling values range from about $9 \%$ to $22 \%$.
At high $F_d$, \texttt{dGrainHBS} shows a positive correlation with swelling,
and \texttt{SwellLink} and \texttt{rResolBulk} show a very weak correlation.
In this case, the swelling ranges from about $20 \%$ to $140 \%$.
Many of the parameters do not seem to have any correlation with fuel swelling
at both low and high fission densities.

\begin{figure}[!ht]
	\centering
	\includegraphics[width=\textwidth]{iuq2/images/scatter_lo.pdf}
	\caption{
		Scatter plots of fuel swelling
		against the fission-gas-behavior parameters
		at $F_d = \num{2.31e21}$ fsn/cm$^3$.
	}
	\label{fig:scat_lo}
\end{figure}

\begin{figure}[!ht]
	\centering
	\includegraphics[width=\textwidth]{iuq2/images/scatter_hi.pdf}
	\caption{
		Scatter plots of fuel swelling
		against the fission-gas-behavior parameters
		at $F_d = \num{5.34e21}$ fsn/cm$^3$.
	}
	\label{fig:scat_hi}
\end{figure}

The data sets were split into two groups, $70 \%$ for training
and $30 \%$ for testing.
The feature vectors of the training set
were then transformed using a min-max scaler.
Afterward, the testing set was transformed
with the aforementioned min-max scaler fitted with the training data.
This is to avoid preprocessing the testing set along with the training set
since doing so can leak information from the testing set.
The training split was then used to build the GP and NN surrogates.
All the surrogate models take $9$ parameters as inputs.
During the construction of the surrogates,
the training set was further split into two groups,
$75 \%$ for actual training and $25 \%$ for validation.
A few kernels, such as RBF, Mat\'ern, rational quadratic, and linear,
and their combinations were used in the GP model to find the best fit.
A GP with a Mat\'ern kernel was found to perform the best
with $R^2$ scores of $0.98$ and $0.97$,
and mean absolute errors of $0.24 \%$ and $2.17 \%$,
for emulating low and high $F_d$, respectively.
The NN surrogate model was optimized by tuning its hyperparameters
with a grid search.
The model has $4$ hidden layers with $250$ neurons each,
and $1$ output layer with one neuron to represent swelling values.
The NN model performed better than the GP model on the test split,
having $R^2$ scores of $0.98$ and $0.98$,
and mean absolute errors of $0.27 \%$ and $1.54 \%$
for low $F_d$ and high $F_d$.
Figure \ref{fig:surr} displays the actual vs predicted plots
for the validation of the surrogate models.
% The GP model has noticeable weakness in the higher swelling range
% for both the low $F_d$ and the high $F_d$ data sets.
% Meanwhile, the NN model predictions are slightly spread out
% only for the low $F_d$ data set.
Since the NN surrogate model performs better overall,
we chose to utilize this model for the rest of the IUQ process.

\begin{figure}[!ht]
	\begin{subfigure}{0.495\textwidth}
		\centering
		\includegraphics[width=\textwidth]{iuq2/images/gp_lo.pdf}
	\end{subfigure}
	\begin{subfigure}{0.495\textwidth}
		\centering
		\includegraphics[width=\textwidth]{iuq2/images/gp_hi.pdf}
	\end{subfigure}
	\begin{subfigure}{0.495\textwidth}
		\centering
		\includegraphics[width=\textwidth]{iuq2/images/nn_lo.pdf}
	\end{subfigure}
	\begin{subfigure}{0.495\textwidth}
		\centering
		\includegraphics[width=\textwidth]{iuq2/images/nn_hi.pdf}
	\end{subfigure}
	\caption{
		Surrogate predictions vs actual test data for Gaussian process (GP)
		and neural network (NN) surrogate models.
		\textit{RERTR5-V6018G} and \textit{RERTR12-L1P755} represent
		low and high fission density data sets, respectively.
	}
	\label{fig:surr}
\end{figure}

\FloatBarrier
\subsection{Sensitivity Analysis}

To quantify the influence of
various fission-gas-behavior parameters on swelling,
a global sensitivity analysis was conducted by calculating Sobol' indices
\cite{sobol2001}.
Figure \ref{fig:sobol} displays
both the first-order ($S_i$) and total-effect ($S_{T_i}$) Sobol' indices.
This provides a breakdown of how individual parameters and their interactions
contribute to the overall variance in the swelling predictions.
It can be observed that the sensitivity of the model is concentrated
in a few key parameters.

\begin{figure}[!ht]
	\begin{subfigure}{\textwidth}
		\centering
		\caption{$F_d = \num{2.31e21}$ fsn/cm$^3$ (\textit{RERTR5-V6018G}).}
		\includegraphics[width=0.8\textwidth]{iuq2/images/sobol_lo.pdf}
	\end{subfigure}
	\begin{subfigure}{\textwidth}
		\centering
		\caption{$F_d = \num{5.34e21}$ fsn/cm$^3$ (\textit{RERTR12-L1P755}).}
		\includegraphics[width=0.8\textwidth]{iuq2/images/sobol_hi.pdf}
	\end{subfigure}
	\caption{
		First-order ($S_i$) and total-effect ($S_{T_i}$) Sobol' indices
		calculated using the neural network surrogate models.
	}
	\label{fig:sobol}
\end{figure}

% NOTE: talk about physical significance
Figure \ref{fig:sobol}a depicts how the parameter \texttt{rResolBulk}
accounts for most of the variability in the model output at low fission density.
Both the first-order and total-effect indices of \texttt{rResolBulk}
are about $0.7$.
The second most important parameter here is \texttt{vResol},
which has indices of about $0.2$.
A distinct shift is observed at high fission density,
where \texttt{dGrainHBS} becomes the most influential parameter
($S_i, S_{T_i} \approx 0.65$).
At this fission density, there are also some minor influences
from \texttt{rResolBulk}, \texttt{SwellLink} and \texttt{FaceCovMax}.
Other parameters, such as \texttt{StickProb} and \texttt{fNucleate},
consistently show negligible sensitivity.
The results also reveal that
the first-order indices are almost identical to the total-effect indices.
This close agreement suggests that the effects of the parameters are additive,
with almost no complex interactions between the parameters.

\FloatBarrier
\subsection{MCMC Sampling}

MCMC sampling is performed by proposing candidate parameter values,
running surrogate models with them,
and then accepting or rejecting the candidate values
according to the acceptance ratio.
Figure \ref{fig:mcmc_trace} displays the trace plots of the MCMC samples
of all the fission-gas-behavior parameters.
Two chains of $200,000$ samples are overlaid on the trace plots.
The cumulative averages of the blue and orange traces
are shown in black and red, respectively.
After an initial period, the cumulative averages of both chains converge.
The acceptance rate of the MCMC sampling was about $26 \%$ for both chains,
which is close to the optimal value
for mixing for a random walk Metropolis algorithm \cite{gelman1997}.
To be sure of the convergence of the MCMC samples,
the Gelman-Rubin statistic \cite{gelman1992}
and the effective sample size (ESS) \cite{kish1965}
of the MCMC samples are also calculated.
The Gelman-Rubin statistic is either $1$ or very close to $1$
for all the posterior samples,
which indicates that no convergence issues were detected.
For all $9$ parameters, the ESS is at least $2100$,
which is sufficient for stable estimates.

\begin{figure}[!ht]
	\centering
	\includegraphics[width=0.9\textwidth]{iuq2/images/mcmc_trace.png}
	\caption{
		Trace plots of the posterior samples
		of the fission-gas-behavior parameters from MCMC sampling.
	}
	\label{fig:mcmc_trace}
\end{figure}

Except for \texttt{dGrainHBS}, the MCMC samples span the entire range
defined by the parameter priors.
The averages of the MCMC samples
for \texttt{FaceCovMax}, \texttt{SwellLink}, and \texttt{rResolBulk}
are slightly higher than the averages from the uniform prior distributions.
For the other parameters, the sample averages are noninformative.
This corroborates the findings from the global sensitivity analysis.

\FloatBarrier
\subsection{IUQ}

\begin{figure}[!ht]
	\centering
	\includegraphics[width=\textwidth]{iuq2/images/iuq.png}
	\caption{
		Posterior distributions of $9$ fission-gas-behavior parameters.
		Marginal densities are in the diagonal
		and the pair-wise joint densities are in the off-diagonal.
	}
	\label{fig:iuq}
\end{figure}

The MCMC samples were thinned by picking every 100th value
to reduce auto-correlation.
The remaining $4000$ posterior samples are investigated for the IUQ process.
The posterior distributions obtained from these samples
are displayed in Figure \ref{fig:iuq}.
The diagonal subplots in Figure \ref{fig:iuq} show
the marginal posterior distributions of the fission-gas-behavior parameters.
\texttt{dGrainHBS} has a defined posterior resembling a normal distribution.
\texttt{FaceCovMax}, \texttt{SwellLink}, and \texttt{rResolBulk}
have left skewed marginal posteriors.
All the other parameters have almost uniform posterior distirbutions.
In addition to the marginal posteriors, the off-diagonal subplots
present the pairwise correlations as contour plots.
Although the prior distributions were specified as independent,
the Bayesian IUQ process captures dependencies among parameters
through MCMC sampling.
Thus, when generating new samples from the posteriors,
the correlation between the parameters should be taken into account.
In our case, the posterior of \texttt{dGrainHBS} shows positive correlations
with that of \texttt{FaceCovMax}, \texttt{SwellLink}, and \texttt{rResolBulk}.
Also, \texttt{rResolBulk} and \texttt{vResol} have a weak negative correlation.
There does not seem to be
any other noticeable correlations between the parameters.

\begin{table}[ht]
\centering
\caption{
	Posterior means, standard deviations, and 95\% credible intervals (CIs)
	of the means for all the fission-gas-behavior parameters.
}
\label{tab:iuq}
\begin{tabular}{lccc}
\toprule
Parameter				& Mean
						& Std. deviation
						& 95\% CI of the mean               \\
\midrule
\texttt{dGrainHBS}		& \num{3.23e-5}
						& \num{8.67e-6}
						& {[} \num{1.67e-5}, \num{5.15e-5} {]} \\
\texttt{FaceCovMax}		& \num{7.20e-1}
						& \num{1.17e-1}
						& {[} \num{5.13e-1}, \num{8.98e-1} {]} \\
\texttt{SwellLink}		& \num{4.92e-2}
						& \num{2.08e-2}
						& {[} \num{1.24e-2}, \num{7.88e-2} {]} \\
\texttt{rResolBulk}		& \num{3.03e-9}
						& \num{1.16e-9}
						& {[} \num{9.09e-10}, \num{4.90e-9} {]} \\
\texttt{DatomFissGBx}	& \num{2.82e4}
						& \num{1.29e4}
						& {[} \num{7.00e3}, \num{4.90e4} {]} \\
\texttt{StickProb}		& \num{2.10e-7}
						& \num{1.16e-7}
						& {[} \num{1.11e-8}, \num{3.92e-7} {]} \\
\texttt{fNucleate}		& \num{1.69e-9}
						& \num{7.81e-10}
						& {[} \num{3.64e-10}, \num{2.93e-9} {]} \\
\texttt{vResol}			& \num{2.58e-18}
						& \num{8.47e-19}
						& {[} \num{1.10e-18}, \num{3.93e-18} {]} \\
\texttt{rResolGBB}		& \num{6.28e-7}
						& \num{2.18e-7}
						& {[} \num{2.68e-7}, \num{9.83e-7} {]} \\
\bottomrule
\end{tabular}
\end{table}

The results indicate that there is not enough information in the DART data set
to infer meaningful posteriors for some of the parameters.
These seemingly noninfluential parameters might be irrelevant for fuel swelling
or their ranges in the data set may have been too narrow.
The sensitivity analysis, the trace plots of the MCMC samples,
and now the posterior distributions all tell a consistent story.
Summary statistics, including the means, standard deviations,
and $95 \%$ credible intervals (CIs) of the posterior means
of all the fission-gas-behavior parameters, are provided in Table \ref{tab:iuq}.

\FloatBarrier
\subsection{Forward Uncertainty Quantification and Validation}

To assess the IUQ results,
we performed forward uncertainty quantification (FUQ)
by propagating the posterior samples through the NN surrogate models.
The joint swelling distribution resulting from the forward propagation
is shown in Figure \ref{fig:fuq_mvn}a.
Unlike the distribution of observed swelling (Figure \ref{fig:mvn_expt}),
the distribution from FUQ is asymmetric.
The swelling density at low $F_d$ decreases steeply around $8 \%$.
To better understand this phenomenon,
the marginal distributions are also plotted
in Figures \ref{fig:fuq_mvn}b and \ref{fig:fuq_mvn}c.
At high $F_d$, the swelling density from FUQ matches the observation well.
However, the density at low $F_d$ is significantly narrower
than that of the observation.
The density is also asymmetric and right skewed.
This is due to the fact that the lowest predicted swelling at low $F_d$
is about $9 \%$ (Figure \ref{fig:scat_lo}) in the DART data set.
Thus, any combination of the fission-gas-behavior parameters
within the provided range will not lead to
a swelling value low enough to fit the observation.
Also, the surrogate model used in this work for low $F_d$
has a mean absolute error of about $0.27$,
which explains why the distribution from FUQ extends slightly beyond $9 \%$.

\begin{figure}[!ht]
	\begin{subfigure}{\textwidth}
		\centering
		\caption{}
		\includegraphics[width=0.7\textwidth]{iuq2/images/fuq_mvn.pdf}
	\end{subfigure}
	\begin{subfigure}{0.49\textwidth}
		\centering
		\caption{}
		\includegraphics[width=\textwidth]{iuq2/images/fuq_lo.pdf}
	\end{subfigure}
	\begin{subfigure}{0.49\textwidth}
		\centering
		\caption{}
		\includegraphics[width=\textwidth]{iuq2/images/fuq_hi.pdf}
	\end{subfigure}
	\caption{
		(a) Joint density plot of swelling predictions calculated by
		the forward propagation of the thinned MCMC samples.
		Marginal density plots of swelling predictions calculated by
		the forward propagation of the thinned MCMC samples for
		(b) $F_d = \num{2.31e21}$ fsn/cm$^3$,
		and (c) $F_d = \num{5.34e21}$ fsn/cm$^3$.
	}
	\label{fig:fuq_mvn}
\end{figure}

Although the distribution from forward propagation
does not exactly match that of the observation specially at low $F_d$,
it shows reasonable overall agreement.
To show the effects of the IUQ process,
the experimental observations, swelling ranges from the DART data sets,
and swelling ranges from the forward propagation
are plotted in Figure \ref{fig:iuq_comp}.
At low $F_d$,
the mean of the values from the forward propagation matches the observation,
but the standard deviation is narrower than that of the observation
due to the reason explained above.
At high $F_d$,
the range of swelling values is significantly reduced
from the DART predictions to the forward propagation,
matching the observation almost exactly.
However, this procedure does not constitute
a rigorous validation of the IUQ results,
as the same surrogate models are used for both the IUQ and FUQ processes.
A more robust validation would involve generating new samples
from the joint posterior distributions
and propagating them directly through DART.

\begin{figure}[ht]
	\centering
	\includegraphics[width=10cm]{iuq2/images/aftermath.pdf}
	\caption{
		Comparison of experimental observations with
		swelling predictions from the DART data set
		and swelling predictions from the forward propagation
		of the surrogate models using the MCMC samples.
	}
	\label{fig:iuq_comp}
\end{figure}
