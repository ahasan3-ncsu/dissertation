\chapter{Conclusions}

\section{Contributions to
\texorpdfstring{$\gamma$}{gamma}U-Mo Fuel Development}

With the application of various computational methods,
we quantified many crucial material properties of $\gamma$U-Mo fuel.
This will assist the higher-length-scale models of the fuel
in making more robust predictions,
and thus help the fuel qualification process in general.
Overall, this dissertation achieves the following:

\begin{itemize}
\item
Fully quantifies the GB diffusion coefficients of U, Mo, and Xe in $\gamma$U-Mo
for various compositions, temperatures, and GB orientations.
The findings provide critical insight into the anisotropy of GB diffusion.
This quantification enables comparison between bulk and GB diffusion
and in turn effective diffusivity calculations in the fuel.

\item
Provides a mathematical model for re-solution rate of Xe gas bubbles,
taking into account both bubble size and pressure.
The model also considers
both the heterogeneous and the homogeneous mechanisms of re-solution.
This analytical model is readily usable
in mesoscale- and engineering-scale codes
trying to predict gas bubble evolution and swelling in $\gamma$U-Mo.

\item
Performs IUQ on still unknown fission-gas-behavior parameters of $\gamma$U-Mo
as implemented in DART.
Three significant parameters are identified and subsequently optimized
for better meso-scale simulation of the U-Mo fuel.
In the process, it also sets a precedent for parameter optimization
as more data and material properties become available in the future.
\end{itemize}

\section{Future Work}

Some of the studies included in this report is still unfinished.
Future work will attempt the completion of these studies
by achieving the following:

\begin{itemize}
\item
Futher investigation is needed to evaluate the re-solution process of Xe.
Preliminary findings suggest that the homogeneous mechanism of re-solution
might not be negligible compared to the heterogeneous mechanism.
The two-temperature model will be employed
to quantify the amount of energy transferred
from the electronic subsystem to the ionic subsystem.

\item
The IUQ process will be expanded to incorporate newly generated data,
which takes into account variations in fission rate and grain size.
Surrogate models will be developed for the new dataset,
enabling a refined IUQ to better quantify
fission-gas-behavior parameter distributions
applicable to a wide range of reactor conditions.

\item
MLIPs of U and Mo will be developed and evaluated.
Provided MLIPs can achieve reasonable accuracy compared to the existing ADP,
additions of other elements to the potential will be explored.
Successful implementation of this
will enable MD simulations U-Mo fuel with liners and/or claddings.
\end{itemize}
