\chapter{Conclusions}

\section{GB Diffusion}

In the present work, MD simulations are performed using an ADP potential
to compute the GB diffusion coefficients of U, Mo, and Xe
in $\gamma$U-Mo alloys \cite{starikov2018}.
First, the GB region is identified by locating the lattice point jumps
of atoms using the atomic trajectories.
It is observed that the GB width increases almost linearly
from around 6 \r{A} to around 12 \r{A} with temperature
in the range 600 K - 1200 K.
The MSDs of the U, Mo, and Xe atoms are then used to calculate
the GB diffusivities of the species in several $\gamma$U-Mo alloys
($\gamma$U-7Mo, $\gamma$U-10Mo, and $\gamma$U-12Mo)
as a function of temperature and misorientation angle.
The GB diffusion coefficients are typically
on the order of $10^{-14}$ to $10^{-11}$ m$^2$s$^{-1}$.
The U GB diffusivity is always higher than the Mo GB diffusivity
for a specific GB at a given temperature.
The Xe GB diffusivity is closer to the Mo GB diffusivity around 600 K
and it steadily approaches the U GB diffusivity with increasing temperature.
The GB diffusion coefficients parallel to the tilt axis of the GB plane
are generally larger than the coefficients perpendicular to the tilt axis,
with some specific GBs showing significant anisotropy.
The orientation-averaged GB diffusion coefficients parallel to the tilt axis
are used to evaluate the effect of composition.
There is a significant negative correlation
between the GB diffusivities and the concentration of Mo in the alloy.
The orientation-averaged diffusion coefficients of U, Mo, and Xe
are between three to fifteen orders of magnitude faster
than the intrinsic/self-diffusion coefficients of the species in the bulk,
and GB diffusion dominates the effective diffusion of the material
at lower temperatures.
The diffusional acceleration by GBs is the most prominent for Xe,
and a quantitative account of this will improve
the understanding of the fission gas swelling behavior.
The calculated diffusion coefficients will also inform
models of creep mechanisms.
Overall, the results obtained from this work will provide insight
into the various phenomena related to the fuel performance
of the $\gamma$U-Mo fuels.
