\chapter{Conclusions}
\label{chap:conclude}

\section{Summary of Results}

\subsection{Grain Boundary Diffusion Coefficients
in \texorpdfstring{$\gamma$}{gamma}U-Mo}

MD simulations were performed using an ADP potential
to compute the GB diffusion coefficients of U, Mo, and Xe
in $\gamma$U-Mo alloys \cite{starikov2018}.
First, the GB region was identified by locating the lattice point jumps
of atoms using the atomic trajectories.
It was observed that the GB width increases almost linearly
from around 6 \r{A} to around 12 \r{A} with temperature
in the range 600 K - 1200 K.
The MSDs of the U, Mo, and Xe atoms were then used to calculate
the GB diffusivities of the species in several $\gamma$U-Mo alloys
($\gamma$U-7Mo, $\gamma$U-10Mo, and $\gamma$U-12Mo)
as a function of temperature and misorientation angle.
The GB diffusion coefficients are typically
on the order of $10^{-14}$ to $10^{-11}$ m$^2$s$^{-1}$.
The U GB diffusivity is always higher than the Mo GB diffusivity
for a specific GB at a given temperature.
The Xe GB diffusivity is closer to the Mo GB diffusivity around 600 K
and it steadily approaches the U GB diffusivity with increasing temperature.
The GB diffusion coefficients parallel to the tilt axis of the GB plane
are generally larger than the coefficients perpendicular to the tilt axis,
with some specific GBs showing significant anisotropy.
The orientation-averaged GB diffusion coefficients parallel to the tilt axis
were used to evaluate the effect of composition.
There is a significant negative correlation
between the GB diffusivities and the concentration of Mo in the alloy.
The orientation-averaged diffusion coefficients of U, Mo, and Xe
are between three to fifteen orders of magnitude faster
than the intrinsic/self-diffusion coefficients of the species in the bulk,
and GB diffusion dominates the effective diffusion of the material
at lower temperatures.
The diffusional acceleration by GBs is the most prominent for Xe,
and a quantitative account of this will improve
the understanding of the fission gas swelling behavior.
The calculated diffusion coefficients will also inform
models of creep mechanisms.

\subsection{Xe Gas Bubble Re-solution in U-10Mo Nuclear Fuel}

BCA and MD simulations were utilized
to determine the re-solution rate of Xe gas bubbles in U-10Mo nuclear fuel.
Both the homogeneous and heterogeneous re-solution mechanisms were investigated.
Thermal spikes initiated by electronic stopping cannot cause re-solution.
Thus, the occurrence of heterogeneous re-solution in U-10Mo is not probable.
Homogeneous re-solution, which is brought about by nuclear stopping,
was found to be the only mechanism of re-solution in U-10Mo.
Therefore, the re-solution rate was calculated
by first profiling FF behavior in the fuel,
then evaluating the FF interactions with Xe gas bubbles,
and finally putting all the information together in a physical model.
The computed re-solution rate $b$ is
$\num{4.4e-26} \dot{F}$ s$^{-1}$ for the largest intergranular bubble
and $\num{8.8e-25} \dot{F}$ s$^{-1}$ for the smallest intragranular bubble,
where the unit of $\dot{F}$ is fission/m$^3$/s.
Furthermore, BCA simulations with varying Xe number density in the bubble
revealed that
the re-solution rate is inversely proportional to the Xe number density.
Thus, higher bubble pressure leads to a lower re-solution rate.
The results of this study provide a physics-based description of
the Xe gas bubble re-solution rate in U-10Mo.

\subsection{Inverse Uncertainty Quantification
of U-10Mo Fission-Gas-Behavior Parameters}

This work presented a methodology for performing IUQ on
unknown fission-gas-behavior parameters of U-10Mo utilized in DART.
The analysis was based on two data sets generated from $6,400$ DART simulations,
one for simulating the low $F_d$ conditions of \textit{RERTR5-V6018G}
and another for simulating the high $F_d$ conditions of \textit{RERTR12-L1P755}.
GP and NN surrogate models were built to emulate the data sets.
The NN surrogates were then used for global sensitivity analysis.
It was observed that the parameter \texttt{rResolBulk}
is the most influential parameter at low fission density,
whereas \texttt{dGrainHBS} is the major driver of variance in swelling
at high fission density.
MCMC sampling with the NN surrogates then led to
converged posterior probability distributions for all the parameters.
Only the parameter \texttt{dGrainHBS} has a well-defined
Gaussian-like posterior distribution.
The posteriors of other parameters span the parameter ranges fully,
leading to almost uniform distribusions.
Forward propagation was then performed with the posteriors,
providing validation of the IUQ results.

\section{Contributions to U-Mo Fuel Development}

With the application of various computational methods,
we assessed crucial microstructural properties of the U-Mo fuel.
This will assist the higher-length-scale models of the fuel
in making more robust predictions,
and thus help the fuel qualification process in general.
Overall, this dissertation achieves the following:

\begin{itemize}
\item
Fully quantifies the GB diffusion coefficients of U, Mo, and Xe in $\gamma$U-Mo
for various compositions, temperatures, and GB orientations.
The findings provide critical insight into the anisotropy of GB diffusion.
This quantification enables comparison between bulk and GB diffusion
and in turn effective diffusivity calculations in the fuel.

\item
Provides a mathematical model for re-solution rate of Xe gas bubbles,
taking into account both bubble size and pressure.
The model also considers
both the heterogeneous and the homogeneous mechanisms of re-solution.
This analytical model is readily usable
in meso- and engineering-scale codes
trying to predict gas bubble evolution and swelling in U-Mo.

\item
Performs IUQ on still unknown fission-gas-behavior parameters of U-Mo
as implemented in DART.
A few significant parameters are identified and subsequently optimized
for better meso-scale simulation of the U-Mo fuel.
In the process, it also sets a precedent for parameter optimization
as more data and material properties become available in the future.
\end{itemize}

\section{Future Work}

Based on the insight gained during the research phase of this dissertation,
the following topics would make for good research:

\begin{itemize}
\item
Assessment of blistering mechanism in U-Mo:
The blisters are observed as a raised region on the fuel-plate surface,
resulting from the plastic deformation of the aluminium cladding
due to large fission gas pore pressure in the fuel meat.
Verifying the possible causes of fuel blistering
can provide the fuel developers with a better understanding
of the fuel behavior under extreme conditions.

\item
Incorporation of cladding elements in the U-Mo potential:
Fuel-cladding interactions need to be understood well for fuel evaluation.
An accurate MLIP of U, Mo, and Al/Zr would be extremely helpful
for atomistic simulations of the U-Mo fuel with liners and/or claddings.

\item
IUQ of fission-gas-behavior parameters:
The IUQ work should be extended
to incorporate quantities of interest other than fuel swelling.
This may help quantify some parameters that seemed noninfluential in our study.
Also, more DART simulations of swelling at different operational conditions
can be performed to enable the evaluation of model bias.
\end{itemize}
