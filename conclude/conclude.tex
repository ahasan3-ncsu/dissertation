\chapter{Conclusions}

\section{Contributions to
\texorpdfstring{$\gamma$}{gamma}U-Mo Fuel Development}

With the application of various computational methods,
we quantified many crucial material properties of $\gamma$U-Mo fuel.
This will assist the higher-length-scale models of the fuel
in making more robust predictions,
and thus help the fuel qualification process in general.
Overall, this dissertation achieves the following:

\begin{itemize}
\item
Fully quantifies the GB diffusion coefficients of U, Mo, and Xe in $\gamma$U-Mo
for various compositions, temperatures, and GB orientations.
This quantification enables comparison between bulk and GB diffusion
and in turn effective diffusivity calculations in the fuel.

\item
Provides a mathematical model for re-solution rate of Xe gas bubbles,
taking into account both bubble size and pressure.
The model is readily usable in mesoscale- and engineering-scale codes
trying to predict gas bubble evolution and swelling in $\gamma$U-Mo.

\item
Performs IUQ on still unknown fission-gas-behavior parameters of $\gamma$U-Mo.
In the process, it also sets a precedent for parameter optimization
as more data and material properties become available in the future.
\end{itemize}

\section{Future Work}

\begin{itemize}
\item
Xe re-solution study needs more work.
It is possible that the homogeneous mechanism of re-solution
is not negligible compared to the heterogeneous mechanism.
The two-temperature model will be employed
to quantify the amount of energy transferred
from the electronic subsystem to the ionic subsystem.

\item
The IUQ process will be extended to the newly generated data,
which takes into account variable fission rate and grain size.
Surrogate models will be built for the new dataset,
and IUQ will be performed again
to better quantify fission-gas-behavior parameter distributions
applicable to a broad range of reactor conditions.

% TODO: add blistering and mtp
\end{itemize}
