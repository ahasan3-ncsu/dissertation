\chapter{Methods}

Oh no!

\section{UMo GB diffusion}

Molecular dynamics simulations have been performed
using the LAMMPS software package \cite{lammps}.
An angular dependent potential (ADP) describing the interactions
among U, Mo, and Xe is used for the simulations
\cite{starikov2018, beelerUMoXe}.
Supercells with periodic boundaries and two GBs parallel to the (010) plane
are generated by dividing the supercell into two regions.
One GB is in the middle of the supercell, and another is along the edge.
For symmetric tilt GBs, each region has a tilted bcc lattice
with respect to the [001] tilt axis.
Figure \ref{fig:gb} shows an example of such a supercell.
Six symmetric tilt GBs are constructed,
having tilt planes of \{120\}, \{130\}, \{150\}, \{190\}, \{340\}, and \{350\}.
Asymmetric tilt and twist GBs are also generated for investigation.
For the asymmetric tilt systems, half of the supercell is rotated
with respect to the [001] tilt axis while the other half is kept fixed.
Four asymmetric tilt systems having tilt planes
\{110\}, \{130\}, \{190\}, and \{350\} are examined.
For twist systems, half of the supercell is rotated
with respect to the [010] tilt axis.
The examined twist planes are \{110\} and \{230\}.
Since this study explores bcc structures,
the misorientation angle can range from $0^{\circ}$ to $90^{\circ}$
due to the symmetry of the crystal structure.
The rotational planes in the symmetric tilt, asymmetric tilt,
and twist systems are chosen so that this range is explored uniformly.

\begin{figure}[!ht]
\centering
\includegraphics[height=3cm]{diffusion/images/configuration.png}
\caption{
	$\gamma$U-10Mo \{120\} symmetric tilt grain boundary
	at the initial configuration.
	Grain boundaries are in the middle and along the edges of the supercell.
	Red atoms are uranium and blue atoms are molybdenum.
}
\label{fig:gb}
\end{figure}

The dimensions of the supercells are at least
$50 \text{ \r{A}} \times 200 \text{ \r{A}} \times 50 \text{ \r{A}}$,
with the dimension perpendicular to the GB being the longest.
These dimensions are verified to be large enough to obtain a stable system,
from which converged energies can be extracted \cite{beeler2018gb}.
A single simulation consists of 35000 to 50000 atoms
depending on the type and misorientation angle of the GB.
A temperature range starting from 600 K to 1200 K has been probed
with an increment of 100 K.
Three different compositions are investigated for symmetric tilt GB systems:
$\gamma$U-7Mo (16 at.\% Mo), $\gamma$U-10Mo (22 at.\% Mo),
and $\gamma$U-12Mo (25 at.\% Mo).
For asymmetric tilt and twist GBs, only $\gamma$U-10Mo has been considered.

An initial validation study is performed
to ensure the correct construction of the GBs,
measured against the reproducibility of GB energies in the literature.
The GB energy is computed using the following equation.
\begin{equation}
	E_{gb} = \bigg( \frac{E - E_0}{A} \bigg) N
\end{equation}
where $E$ is the potential energy per atom of the system with GBs,
$E_0$ is the potential energy per atom of the $\gamma$U-Mo system without GBs,
$A$ is the area of the GBs,
and $N$ is the number of atoms in the system with GBs.
To obtain $E$, 25 simulations with different random number seeds
are performed for each GB system.
This allows for the evaluation of the statistical significance
of the results since the seeds determine the initial distribution of Mo atoms
and the velocities of both the U and Mo atoms in the supercell.
Similarly, 25 simulations of $\gamma$U-Mo systems without a GB are conducted
with different random number seeds to get an average reference energy $E_0$.
The systems are relaxed over 100 ps in an NPT ensemble
using a Nos\'e-Hoover thermostat at 1200 K,
with energies averaged over the final 50 ps.

After validating the structures of GB systems,
simulations are performed to extract the GB diffusion coefficients.
To that end, the supercells are first relaxed in an NPT ensemble
using a Nos\'e-Hoover thermostat and barostat
with a damping parameter of 0.1 ps.
100,000 timesteps are run with a timestep size of 1 fs,
making the NPT relaxation 100 ps in length.
Afterward, the simulation box lengths are fixed
at the equilibrated supercell lengths obtained during the NPT relaxation,
and the systems are further equilibrated for 100 ps in an NVT ensemble.
The NVT ensemble also uses a Nos\'e-Hoover thermostat
with a 0.1 ps damping parameter.
The production runs are then executed for 5,000,000 timesteps
with a timestep size of 2 fs,
yielding a 10 ns simulation for determining the diffusion coefficients.
For a single GB structure, 5 simulations are performed
with different random number seeds at a given temperature.

To examine the diffusion of Xe in $\gamma$U-Mo GBs,
the $\gamma$U-10Mo symmetric tilt GBs are generated as per the prior procedure,
but with a single Xe substitutional defect inserted in both GBs
(a total of two Xe atoms in the system).
For Xe simulations, the 100 ps NPT-ensemble relaxation
and a subsequent 50 ps NVT-ensemble relaxation are first performed
and then Xe substitutions are made.
After Xe insertion, the system is again relaxed in an NVT ensemble for 50 ps.
A timestep size of 1 fs is used throughout the relaxation period.
Afterward, a 100 ns data collection period begins with a timestep size of 5 fs.
Since there are only two Xe atoms per system,
collecting data over a longer period is necessary for diffusion calculations.
5 simulations with different random number seeds are conducted
for each GB system with Xe.
It is also verified through a few simulations
that changing the timestep size from 2 fs to 5 fs has minimal impact
on the Xe trajectory and the stability of the simulations.

The GB region of the supercell is determined
using the atomic trajectory images generated with OVITO \cite{ovito}.
Atomic trajectories show where the lattice point jumps occur.
Since almost all such jumps occur in the vicinity of the GB plane,
the GB region can be differentiated from the bulk
by determining the lattice points associated with the jumps.
Figure \ref{fig:def} depicts how the GB region is defined
as a rectangular cuboid in a system with symmetric tilt \{190\} GBs.
Please notice that the GB region also includes lattice points
that do not participate in the diffusion process.
Tracking only mobile GB atoms would overestimate the diffusion coefficients
and would not be representative of the GB structure.
Once the GB region of a system is defined,
the number of U and Mo atoms in that region is also recorded
using OVITO's expression selection tool.
To verify the GB width evaluated this way,
the potential energy of the system as a function of the $y$ coordinate
is also computed by averaging the potential energy
over thin layers parallel to the GB plane.
The peak widths of the potential energy graphs corroborate
the determination of GB width $\delta_{gb}$ using atomic trajectories.
Several atomic trajectories and potential energy peaks are analyzed
for each GB system using different portions of the simulation data
to detect any GB migration
since the migration of a GB would likely yield spurious results.
Only in a few high-temperature simulations was GB migration observed.
The simulations showing GB migration are not used
in the calculation of the diffusion coefficients.

\begin{figure}[!ht]
\centering
\includegraphics[height=6cm]{diffusion/images/gb_def.png}
\caption{
	Definition of the grain boundary region
	based on the trajectories of atoms in a $\gamma$U-10Mo supercell
	with symmetric tilt \{190\} grain boundary at 1100 K.
}
\label{fig:def}
\end{figure}

To calculate the mean squared displacement (MSD) of the U, Mo, and Xe atoms
in the systems, a buffer averaging scheme is deployed
as described in \cite{rapaport2004}.
In this scheme, multiple parallel MSD calculations are performed
over subsets of the full simulations.
For example, in the case of systems without Xe,
the first buffer computes the MSD from 2 ns to 7 ns,
and the second computes the MSD from 3 ns to 8 ns, and so on.
MSD values from 4 such buffers are then averaged.
While the average from only 4 buffers is enough to get
a smooth MSD versus time curve of U and Mo atoms,
MSD calculations of Xe need more buffers.
Therefore, a total of 34 buffers,
where each buffer calculates the MSD over 12.5 ns,
are employed for the systems with Xe.
The starting times of two successive buffers
are separated by 2.5 ns in this case.

The Einstein relation is used to compute the diffusion coefficients
from the buffer-averaged MSD values.
The following equations are used to extract GB diffusion coefficients
of U, Mo, and Xe perpendicular and parallel to the tilt axis,
\begin{equation}
	D^i_{\perp} = \frac{N_{i, gb}}{N_i} \frac{\langle x^2 \rangle}{2 \Delta t}
\end{equation}
\begin{equation}
	D^i_{\parallel} = \frac{N_{i, gb}}{N_i} \frac{\langle z^2 \rangle}{2 \Delta t}
\end{equation}
where $D^i_{\perp}$ is the GB diffusion coefficient of $i$ (=U, Mo, or Xe)
perpendicular to the tilt axis,
$D^i_{\parallel}$ is the GB diffusion coefficient of $i$
parallel to the tilt axis,
$\Delta t$ is the corresponding simulation time of the directional MSDs
$\langle x^2 \rangle$ or $\langle z^2 \rangle$ of the system,
$N_i$ is the number of $i$ atoms in the system,
and $N_{i, gb}$ is the number of $i$ atoms in the GB region.
It is assumed that the atoms outside the GB region
undergo negligible diffusion.
Generally, diffusion follows the Arrhenius equation as formulated below,
\begin{equation}
	D = D_0 \exp \bigg( -\frac{E_a}{k_B T} \bigg)
\end{equation}
where $D_0$ is the pre-exponential factor,
$E_a$ is the activation energy,
$T$ is the absolute temperature,
and $k_B$ is the Boltzmann constant.
Arrhenius fits to the GB diffusion coefficients
are calculated for all the simulated systems.

\section{Xe gas bubble re-solution}

The LAMMPS software package \cite{lammps} was utilized for the MD work,
in conjunction with a U-Mo-Xe angular-dependent potential (ADP)
\cite{smirnova2013, starikov2018, beelerADP}.
This ADP can accurately describe the body-centered cubic phase
of $\gamma$U-Mo alloys, reproducing their stable structure,
modulus of elasticity, room temperature density, and melting point.
Given that the present work focuses on high-energy interactions
pertaining to PKAs and thermal spike events,
the distance between the ions and atoms can become extremely small.
On its own, the classical interatomic potential
does not accurately describe these interactions.
To address this issue in all the simulations,
the Ziegler, Biersack, and Littmark (ZBL) \cite{ziegler1985}
repulsive potential,
which provides a realistic depiction of ion-ion repulsive interactions,
was combined with the ADP.
The inner and outer cutoff distances for the ZBL potential
were chosen to be 1 \r{A} and 2 \r{A}, respectively.

For both the PKA and thermal spike simulations,
the system was equilibrated at 400 K at a pressure of 0 bar in an NPT ensemble,
using the Nos\'e-Hoover barostat and thermostat.
Utilizing the resulting configurations,
spherical bubbles of radius $R_{bubble}$ were created
by removing U and Mo atoms and depositing gaseous Xe atoms
at a given density within the resulting void.
The system was then further equilibrated in the NPT ensemble.
The simulations were ultimately performed using an NVE ensemble
with a canonical sampling thermostat \cite{bussi2007}
at the edges of the simulation box to act as a heat sink.
This enabled the resulting local heating to slowly dissipate
from the simulation by modeling a constant long-range temperature.
Electronic stopping was incorporated into the simulation
by applying a frictional force to each atom.
This frictional force was calculated
based on the stopping power of the related atom species in the fuel.
These values were determined using the Stopping and Range of Ions in Matter
(SRIM) software \cite{ziegler2010srim}.
The frictional force due to electronic stopping is applied to atoms
whose kinetic energy exceeds a specified threshold.
This threshold is often set to 10 eV or double the cohesive energy for metals \cite{nordlund1998, duffy2006}.
In the present work, electronic stopping was applied
only when an atom's kinetic energy exceeded 9 eV,
which is double the U-U cohesive energy of 4.5 eV.
The cohesive energy was evaluated by Beeler et al. \cite{beeler2018disp}
using the same ADP potential applied in the present work.
A variable timestep size was also implemented for all simulations,
such that any atom's maximum displacement between two successive timesteps
was less than or equal to 0.01 \r{A}.

For the PKA model, a $160 \times 160 \times 160$ \r{A}$^3$ supercell
was utilized, holding around 8 million atoms.
Such a large supercell was needed to prevent the self-interaction of cascades
through their periodic images.
The thermostatted sink was implemented in all three directions.
Figure \ref{fig:struct} gives a cross section of such a supercell.
Within a 20 \r{A} radius gas bubble,
the investigation was conducted on 200 and 400 Xe atoms,
corresponding to pressures of 124 MPa and 787 MPa, respectively.
The bubbles with 200 Xe atoms were underpressurized,
whereas those with 400 Xe atoms led to an overpressurized bubble.
For reference,
the equilibrium pressure of such a bubble is 436 MPa \cite{beelerADP}.
The cascade was initiated by selecting a single PKA,
located approximately 5.5 nm from the bubble's center.
The PKA was directed toward the bubble at high velocity.
PKAs featuring kinetic energies of up to 500 keV were simulated.
These simulations were executed to model the initial 120 ps of the cascade,
ensuring that the total temperature of all the examined systems
fell below 600 K.
For each bubble pressure and PKA energy value,
five simulations were performed---using different initial atomic configurations
and PKA directions---to gather statistical data.

\begin{figure}[ht]
	\centering
	\includegraphics[height=6cm]{resol/images/struct.png}
	\caption{
		Initial configuration for an MD simulation of the PKA.
		The supercell is sliced through the middle
		to show the 20 \r{A} radius Xe gas bubble (black).
		The red and blue dots represent U and Mo atoms, respectively.
	}
	\label{fig:struct}
\end{figure}

In the thermal spike model, a $120 \times 120 \times 50$ \r{A}$^3$ supercell
was simulated, containing approximately 1.5 million atoms.
The thermal spike axis was aligned
with the shortest dimension of the supercell,
and the thermostatted sink was applied only in directions
perpendicular to the thermal spike axis.
Atoms within a cylindrical zone of 35 \r{A} radius along the thermal spike axis
were excited via high kinetic energy
in order to simulate local heating due to electronic stopping.
This was achieved by rescaling the atom velocities.
To understand the role played independently by bubble size,
the Xe/vacancy ratio was at first kept constant.
The Xe/vacancy ratio here is defined as the ratio of the number of Xe atoms
in the bubble to the number of lattice sites occupied by the bubble.
This ratio determines the bubble pressure.
A Xe/vacancy ratio of 0.2 was chosen for this purpose,
since it ensures an equilibrium bubble pressure, as per Beeler et al.
\cite{beeler2020improved}.
The thermal spike's position within the supercell was varied
to account for both on- and off-centered collisions
between the gas bubble and thermal spike.
For on-centered thermal spikes, gas bubbles
with radii of 5--40 \r{A} were simulated,
with the thermal spike energies varying from 5 to 30 keV/nm.
For off-centered thermal spikes, gas bubbles of radii 15
\r{A}, 25 \r{A}, and 35 \r{A} were examined,
with a thermal spike energy of 15 keV/nm.
The off-centered distance was varied---at an interval of 10 \r{A}---up to
a specific cutoff value (discussed later).
Bubbles with different Xe/vacancy ratios were also simulated
to account for the bubble pressure's effect on re-solution.
Xe/vacancy ratios of 0.1, 0.3, and 0.5 were used for bubbles of
radii 15 \r{A}, 25 \r{A}, and 35 \r{A}.
Only on-centered thermal spikes were employed in this case,
and their energies were varied over a range of 5--30 keV/nm.

\begin{figure}[ht]
	\centering
	\begin{subfigure}{0.49\textwidth}
		\centering
		\caption{}
		\includegraphics[height=6cm]{resol/images/temp_time.pdf}
	\end{subfigure}
	\begin{subfigure}{0.49\textwidth}
		\centering
		\caption{}
		\includegraphics[height=6cm]{resol/images/resol_time.pdf}
	\end{subfigure}
	\caption{
		(a) System temperature vs. time for 40 \r{A} radius Xe gas bubble
		simulations featuring different thermal spike energies.
		(b) Number of re-solved Xe atoms vs. time.
		The vertical dotted lines represent 450 ps.
	}
	\label{fig:cooldown}
\end{figure}

The thermal spike simulations were run for at least 450 ps,
which is sufficient for all systems to cool down to under 800 K.
Figure \ref{fig:cooldown} shows the system temperature
and the number of re-solved Xe atoms over time
in regard to a few 40 \r{A} radius bubble simulations
featuring various thermal spike energies.
As seen in the figure, the number of re-solved atoms stabilizes
well before 450 ps.
For further verification, longer simulations were also conducted
to allow the systems to cool down to under 500 K.
The re-solution behavior remains unchanged
throughout the cooldown from 800 to 500 K.
Thus, a minimum simulation time of 450 ps was chosen
so as to optimize the use of computing resources.
Cluster analysis was performed on the Xe atoms in the system,
using OVITO \cite{ovito}.
The cutoff distance among clusters was chosen as 10 \r{A}.
Initially, all the Xe atoms in the system formed a single cluster,
representing the original Xe bubble.
However, after introducing the PKAs or thermal spikes,
multiple Xe clusters could be identified due to re-solution of Xe atoms.
The cluster containing the most Xe atoms
was considered the original Xe gas bubble,
and all Xe atoms in the other clusters were classified as re-solved atoms.
When only a single atom remained in the original bubble,
that gas bubble was considered completely re-solved.

\section{Inverse UQ}

\subsection{Problem definition}

Table \ref{tab:uqparam} shows the DART parameters of interest and their ranges.
The orders of magnitude of the parameters vary considerably.
The parameter "dGrainHBS" denotes the high burunup structure grain diameter.
"FaceCovMax" dictates the grain face coverage needed for interlinkage.
"SwellLink" determines the swelling starting interlinkage of the grain edges.
"vResol" is the probability that a gas bubble interacts with fission fragments.
"DatomFissGBx" is the grain boundary diffusion enhancement factor.
"fNucleate" indicates the adjustment factor
for the probability of bubble nucleation on the grain boundary.
Finally, 'aAtomDifFiss" is the linear coefficient
of radiation-driven gas atom diffusivity
\cite{annualreport2021, ye2023, annualreport2022}.

% TODO: need to talk about dataset generation

There are 3200 samples relating the 7 fission-gas-behavior parameters
with fuel swelling at a fission density of $7 \times 10^{21}$ fiss/cm$^3$.

\begin{table}[ht]
\centering
\caption{
	Fission gas behavior parameters and their ranges.
}
\label{tab:uqparam}
\begin{tabular}{lccc}
\toprule
Parameter    & Minimum value & Maximum value & Reference value \\
\midrule
dGrainHBS    & \num{2.4e-5 } & \num{5.6e-5 } & \num{4e-5   }   \\
FaceCovMax   & \num{0.6    } & \num{0.907  } & \num{0.907  }   \\
SwellLink    & \num{0.016  } & \num{0.034  } & \num{0.025  }   \\
vResol       & \num{1.3e-18} & \num{2.7e-18} & \num{2e-18  }   \\
DatomFissGBx & \num{19641  } & \num{40530  } & \num{3e4    }   \\
fNucleate    & \num{3.5e-10} & \num{8.3e-10} & \num{6e-10  }   \\
aAtomDifFiss & \num{3.2e-31} & \num{7.2e-31} & \num{5.1e-31}   \\
\bottomrule
\end{tabular}
\end{table}

The goal of this IUQ study is to find parameter distributions
that lead to fuel swelling predictions
in agreement with the experimental observations.
There have been a few experimental studies
on the fuel swelling of $\gamma$U-Mo fuel.
Robinson et al. have developed a predictive swelling correlation
using the experimental data collected on
74 irradiated $\gamma$U-10Mo monolithic test fuel plates
over a range of irradiation conditions
\cite{rabin2017preliminary, robinson2021}.
The correlation is as follows.
\begin{align}
	\% Swelling = 6.13 \times 10^{-43} F_d^2 + 4 \times 10^{-21} F_d
\end{align}
where $F_d$ is the fission density.
The 95\% confidence interval of the swelling prediction
at $7 \times 10^{21}$ fiss/cm$^3$ spans about 5 swelling\%.
This swelling correlation is utilized as the experimental observation
for IUQ purposes in this work.

\begin{figure}[ht]
	\centering
	\includegraphics[width=0.6\textwidth]{invuq/images/target.png}
	\caption{
		Histogram of the fuel swelling prediction from DART
		at $7 \times 10^{21}$ fiss/cm$^3$
		along with the experimental observation.
	}
	\label{fig:target}
\end{figure}

\subsection{Formulation of the inverse problem}

Let's define the unknown reality or true value of an output as $y^R(x)$,
where $x$ denotes the vector of design variables
specifying experimental conditions.
A computer model simulation can predict that reality only as an approximation:
\begin{align}
	\label{eq:real}
	y^R(x) = y^M(x, \theta^*) + \delta(x)
\end{align}
where $\theta^*$ is a vector of true but unknown values
of calibration parameters $\theta$
and $\delta(x)$ is the model uncertainty/discrepancy
due to an incomplete understanding of the underlying physics of the model.

To learn the reality $y^R(x)$,
experiments may be performed to obtain an observation $y^E(x)$.
However, the measurement process also introduces uncertainty:
\begin{align}
	\label{eq:exp}
	y^E(x) = y^R(x) + \epsilon
\end{align}
where $\epsilon \sim \mathcal{N} (\mu, \Sigma_{\epsilon})$
indicates the measurement error/noise.
It is typical to assume $\mu = 0$
and $\Sigma_{\epsilon} = \sigma_{\epsilon}^2 I$ for experiments
having no systematic bias and having homoscedastic experimental errors.
Combining equations \ref{eq:real} and \ref{eq:exp},
we can obtain the "model updating equation"
\cite{wu2021comprehensive, kennedy2001bayesian, arendt2012quant}:
\begin{align}
	\label{eq:upd}
	y^E(x) = y^M(x, \theta^*) + \delta(x) + \epsilon
\end{align}
Equation \ref{eq:upd} is the starting point for Bayesian IUQ.
Assuming experimental uncertainty is Gaussian,
$\epsilon = y^E(x) - y^M(x, \theta^*) - \delta(x)$
follows a multi-dimensional Gaussian distribution
with a mean of zero and a covariance matrix of $\Sigma_{\epsilon}$.
The posteriors of the true parameters $p(\theta^* | y^E, y^M)$
can then be written as:
\begin{align}
	p(\theta^* | y^E, y^M)
		&\propto p(\theta^*) \cdot p(y^E, y^M | \theta^*) \\
		&\propto p(\theta^*) \cdot \frac{1}{\sqrt{|\Sigma|}}
			\exp \bigg[-\frac{1}{2} (y^E-y^M-\delta)^T
			\Sigma^{-1} (y^E-y^M-\delta) \bigg]
\end{align}
where $p(\theta^*)$ is the prior distribution provided by expert opinion,
and $p(y^E, y^M | \theta^*)$ is the likelihood function.
$\Sigma$ is the covariance of the likelihood consisting of three parts:
\begin{align}
	\Sigma &= \Sigma_{exp} + \Sigma_{bias} + \Sigma_{code}
\end{align}
where $\Sigma_{exp}$ represents experimental uncertainty
due to measurement error,
$\Sigma_{bias}$ means model uncertainty due to an inherent bias in the model,
and $\Sigma_{code}$ means code/interpolation uncertainty
due to the use of surrogate models to reduce the computational cost.
In this work, we assumed $\Sigma_{bias} = 0$ and $\Sigma_{code} = 0$
because of the lack of experimental observations
and high fidelity of the surrogate models.
To get the posterior distributions of the calibration parameters,
the Metropolis-Hastings algorithm is used
to generate Markov Chain Monte Carlo (MCMC) samples
\cite{robert2004metropolis, van2018simple}.

\subsection{Surrogate modeling}

Surrogate models, also called metamodels, response surfaces or emulators,
are approximations of the input/output relation of the original computer model.
They are built from a limited number of full model runs (training data)
and a learning algorithm.
Metamodels usually take much less computational time than the full model
while maintaining the input/output relation to a desirable accuracy.
Once validated, metamodels can be used in uncertainty, sensitivity,
validation and optimization studies,
for which the original computer model can incur
an excessive computational burden
as hundreds or thousands of computer model simulations are needed
\cite{wu2018inverse}.

Any machine learning model that can learn from the training data
and predict an output relatively quickly
compared to the original computer model can be used as a surrogate model.
Typical examples of surrogate models include Moving Least-Squares (MLS),
Radial Basis Functions (RBF), Neural Networks (NN),
Support Vector Machines (SVM), Gaussian Processes (GP),
Polynomial Chaos Expansion (PCE), etc.,
with GP being the most commonly used surrogate
\cite{wu2018inverse, wu2021comprehensive}.
In this work, we have used three linear
and three non-linear models as surrogates.
A brief discussion of these models follows.

Considering $X$ as an input matrix and $y$ as an output vector,
a linear model would have the following formulation:
\begin{align}
	y = \beta X + \varepsilon
\end{align}
where $\beta$ is the vector of unknown coefficients
and $\varepsilon$ is the vector of random errors.
In Ordinary Least-Squares (OLS),
the estimated coefficients $\hat{\beta}$ is determined
by minimizing the sum of squared residuals (the loss function).
\begin{align}
	\hat{\beta} = \arg \min_{\beta} || y - \beta X ||^2
\end{align}
In Least Absolute Shrinkage and Selection Operator (Lasso),
$\hat{\beta}$ is determined
by adding an $L1$ penalty term to the loss function.
\begin{align}
	\hat{\beta} = \arg \min_{\beta} (|| y - \beta X ||^2
					+ \lambda \sum |\beta_i|)
\end{align}
where $\lambda$ is the regularization parameter
that adds a penalty for the magnitude of the coefficients $\beta$.
The benefit in using Lasso is that the added penalty
shrinks some coefficients to zero.
As a result, this regression model also acts as a selection operator.
Ridge regression is different from Lasso
because it uses an $L2$ regularization term to the loss function
\cite{yang2018ridge, thevaraja2019recent}.
The estimated coefficients then become:
\begin{align}
	\hat{\beta} = \arg \min_{\beta} (|| y - \beta X ||^2
					+ \lambda ||\beta||^2)
\end{align}

Gaussian processes (GP) are a popular choice for surrogate models
because they provide a straightforward estimation of the code uncertainty.
A GP is a probabilistic model
that describes a distribution over possible functions.
Given a set of points, a GP can be used to predict the value
of an unknown function at any other point.
Additionally, GP can be used to represent the uncertainty
in the output of the computer model,
which is important for Bayesian inference and calibration
\cite{wu2018inverse, wang2020}.
The mathematical form of a GP model is as follows:
\begin{align}
	y = f^T(x) \beta + z(x)
\end{align}
The first term here is a linear regression of the data
modeling the drift of the mean.
The set of basis functions $f$ is chosen by the user
and $\beta$ are the regression coefficients.
$z(x)$ is a stationary Gaussian random process with zero mean and covariance
$Cov[z(x^{(i)}), z(x^{(j)})] = \sigma^2 \mathcal{R} (x^{(i)}, x^{(j)})$,
where $\sigma^2$ is the process variance
and $\mathcal{R} (\cdot, \cdot)$ is the correlation function or kernel.
A kernel in GP is a function that defines
the similarity between pairs of data points.
There are many different kernels that can be used with GP,
each of which has its own advantages and disadvantages.
Some of the most commonly used kernel functions include the RBF kernel,
the Mat\'ern kernel, and the rational quadratic kernel \cite{wang2020}.

NN models consist of one input layer,
hidden layers with non-linear activation functions,
and one output layer for regression.
All these layers have nodes (also called neurons)
that connect to the nodes of the previous layer.
Each connection between two nodes has an optimizable weight
and each node has an associated bias and an activation function.
The number of hidden layers in the model,
the number of nodes in each layer,
and the activation functions of the nodes are to be tuned
to find an optimal model for a particular dataset
\cite{goodfellow2016, de2013neural}.

Support Vector Regressor (SVR) is a regression model
that uses SVMs, aiming to find a function $f(X)$
that fits the data while keeping the errors within a predefined margin.
\begin{align}
	f(X) = w^T X + b
\end{align}
where $w$ is the weight vector and $b$ is the bias.
SVR uses an $\varepsilon$-insensitive loss function,
where the model allows errors within a certain margin $\varepsilon$.
SVRs can handle non-linear relationships through the use of kernels,
which map input data to a higher-dimensional space
\cite{cortes1995support, gunn1997support}.
In this work, all the surrogate models are implemented
using the sklearn python library \cite{sklearn}.
